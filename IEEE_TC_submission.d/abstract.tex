A class of Boolean functions, 
called \textbf{nested canalyzing functions} (NCFs),
has been used to model certain biological phenomena.
Researchers have studied various properties of these functions
such as sensitivity and stability.
The purpose of this note is to bring out some relationships between NCFs, symmetric 
Boolean functions and a generalization of symmetric Boolean functions,
which we call $r$-symmetric functions (where $r$ is the symmetry level).
Using a normalized representation for NCFs, we develop a 
characterization of when two variables of an NCF are symmetric.
Using this characterization, we show 
that the symmetry level of an NCF $f$
can be easily computed given a standard representation of $f$.
(In contrast, we prove that even approximating the symmetry level of
a general Boolean function to within any factor $\geq 1$ is \cnp-hard.) 
We also present an efficient algorithm for testing whether 
a given $r$-symmetric function is an NCF.
Further, we show that for any NCF $f$ with $n$ variables, the notion of
strong asymmetry considered in the literature is equivalent to
the property that $f$ is $n$-symmetric. 
We use this result to derive a closed form expression for the
number of $n$-variable Boolean functions 
that are NCFs and strongly asymmetric.
We also identify all the Boolean functions that are NCFs 
and symmetric. 
