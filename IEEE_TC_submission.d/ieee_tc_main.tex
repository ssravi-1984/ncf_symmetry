\documentclass[10pt,journal,compsoc]{IEEEtran}
%
% If IEEEtran.cls has not been installed into the LaTeX system files,
% manually specify the path to it like:
% \documentclass[10pt,journal,compsoc]{../sty/IEEEtran}

% Some very useful LaTeX packages include:
% (uncomment the ones you want to load)


% *** CITATION PACKAGES ***
%
\ifCLASSOPTIONcompsoc
  % IEEE Computer Society needs nocompress option
  % requires cite.sty v4.0 or later (November 2003)
  \usepackage[nocompress]{cite}
\else
  % normal IEEE
  \usepackage{cite}
\fi

% *** GRAPHICS RELATED PACKAGES ***
%
\ifCLASSINFOpdf
  % \usepackage[pdftex]{graphicx}
  % declare the path(s) where your graphic files are
  % \graphicspath{{../pdf/}{../jpeg/}}
  % and their extensions so you won't have to specify these with
  % every instance of \includegraphics
  % \DeclareGraphicsExtensions{.pdf,.jpeg,.png}
\else
  % or other class option (dvipsone, dvipdf, if not using dvips). graphicx
  % will default to the driver specified in the system graphics.cfg if no
  % driver is specified.
  % \usepackage[dvips]{graphicx}
  % declare the path(s) where your graphic files are
  % \graphicspath{{../eps/}}
  % and their extensions so you won't have to specify these with
  % every instance of \includegraphics
  % \DeclareGraphicsExtensions{.eps}
\fi


% *** MATH PACKAGES ***
%
\usepackage{amsmath}
\usepackage{amsfonts}
\usepackage{latexsym}
\usepackage{amsmath}
%\usepackage[latin1]{inputenc}
%\usepackage{varioref}
%\usepackage{cite}
% \usepackage{times}


% *** PDF, URL AND HYPERLINK PACKAGES ***
%
\usepackage{url}
% url.sty was written by Donald Arseneau. It provides better support for
% handling and breaking URLs. url.sty is already installed on most LaTeX
% systems. The latest version and documentation can be obtained at:
% http://www.ctan.org/pkg/url
% Basically, \url{my_url_here}.


% correct bad hyphenation here
\hyphenation{op-tical net-works semi-conduc-tor}

%% Some theorem-like environments needed.

\newtheorem{theorem}{Theorem}[section]
\newtheorem{lemma}[theorem]{Lemma}
\newtheorem{corollary}[theorem]{Corollary}
\newtheorem{fact}[theorem]{Fact}
\newtheorem{claim}[theorem]{Claim}
\newtheorem{observation}[theorem]{Observation}
\newtheorem{definition}[theorem]{Definition}
\newtheorem{proposition}[theorem]{Proposition}
\newtheorem{example}[theorem]{Example}

%%% The following command controls the indent for description
%%% environmet.

%\renewcommand{\IEEEiedlistdecl}{\IEEEsetlabelindent{xx}}
\setlength{\IEEElabelindent}{-5pt}
\setlength{\labelsep}{3pt}
%\setlength{\IEEElabelindent}{0pt}

\begin{document}
%
% paper title
% Titles are generally capitalized except for words such as a, an, and, as,
% at, but, by, for, in, nor, of, on, or, the, to and up, which are usually
% not capitalized unless they are the first or last word of the title.
% Linebreaks \\ can be used within to get better formatting as desired.
% Do not put math or special symbols in the title.
%\begin{titlepage}
\title{Symmetry Properties of Nested Canalyzing Functions}
%\end{titlepage}


%
%
% author names and IEEE memberships
% note positions of commas and nonbreaking spaces ( ~ ) LaTeX will not break
% a structure at a ~ so this keeps an author's name from being broken across
% two lines.
% use \thanks{} to gain access to the first footnote area
% a separate \thanks must be used for each paragraph as LaTeX2e's \thanks
% was not built to handle multiple paragraphs
%
%
%\IEEEcompsocitemizethanks is a special \thanks that produces the bulleted
% lists the Computer Society journals use for "first footnote" author
% affiliations. Use \IEEEcompsocthanksitem which works much like \item
% for each affiliation group. When not in compsoc mode,
% \IEEEcompsocitemizethanks becomes like \thanks and
% \IEEEcompsocthanksitem becomes a line break with idention. This
% facilitates dual compilation, although admittedly the differences in the
% desired content of \author between the different types of papers makes a
% one-size-fits-all approach a daunting prospect. For instance, compsoc 
% journal papers have the author affiliations above the "Manuscript
% received ..."  text while in non-compsoc journals this is reversed. Sigh.

\iffalse
%%%%%%%%%%%%%%%%%%%%%%%%%%%%%%%%%%%%%%%%%%%%%%%%%%%%%%%%%%%%%
\author{Daniel J. Rosenkrantz,~\IEEEmembership{}
        Madhav V. Marathe,~\IEEEmembership{Fellow,~IEEE,}
        S. S. Ravi,~\IEEEmembership{}
        and~Richard E. Stearns,~\IEEEmembership{}% <-this % stops a space
\IEEEcompsocitemizethanks{
\IEEEcompsocthanksitem D. J. Rosenkrantz is with the Department
of Computer Science, University at Albany -- SUNY, Albany, NY, 12222.\protect\\
% note need leading \protect in front of \\ to get a newline within \thanks as
% \\ is fragile and will error, could use \hfil\break instead.
E-mail: drosenkrantz@gmail.com
\IEEEcompsocthanksitem M. V. Marathe is with the Biocomplexity Institute \& Initiative
and the Department of Computer Science at the  University of Virginia, Charlottesville,
VA, 22904.\protect\\ E-mail: marathe@virginia.edu
\IEEEcompsocthanksitem S. S. Ravi is with the Biocomplexity Institute \& Initiative
at the  University of Virginia, Charlottesville,
VA, 22904 and the Department of Computer Science, University at Albany -- SUNY,
Albany, NY, 12222.\protect\\ E-mail: ssravi0@gmail.com
\IEEEcompsocthanksitem R. E. Stearns is with the Department
of Computer Science, University at Albany -- SUNY, Albany, NY, 12222.\protect\\
% note need leading \protect in front of \\ to get a newline within \thanks as
% \\ is fragile and will error, could use \hfil\break instead.
E-mail: thestearns2@gmail.com
}
}% <-this % stops an unwanted space
%%%%%%%%%%%%%%%%%%%%%%%%%%%%%%%%%%%%%%%%%%%%%%%%%%%%%%%%%%%%%
\fi
\author{Daniel J. Rosenkrantz,~ Madhav V. Marathe,~ S.~S. Ravi~ and~
        Richard E. Stearns% 
   \thanks{D. J. Rosenkrantz and R. E. Stearns are with the Computer
           Science Department, University at Albany -- SUNY, Albany,
           NY 12222. Email: \{drosenkrantz,thestearns2\}@gmail.com}
   \thanks{M. V. Marathe is with the Biocomplexity Institute and Initiative
           and the Department of Computer Science, University of Virginia,
           Charlottesville, VA, 22904. Email: marathe@virgnia.edu.}
   \thanks{S. S. Ravi is with the Biocomplexity Institute and Initiative
           at the University of Virginia,
           Charlottesville, VA, 22904 and the Computer
           Science Department, University at Albany -- SUNY, Albany, NY 12222.
           Email: ssravi0@gmail.com}
}
\maketitle

\noindent
\thanks{Manuscript received xx, xxxx; revised xx, xxxx.}

% note the % following the last \IEEEmembership and also \thanks - 
% these prevent an unwanted space from occurring between the last author name
% and the end of the author line. i.e., if you had this:
% 
% \author{....lastname \thanks{...} \thanks{...} }
%                     ^------------^------------^----Do not want these spaces!
%
% a space would be appended to the last name and could cause every name on that
% line to be shifted left slightly. This is one of those "LaTeX things". For
% instance, "\textbf{A} \textbf{B}" will typeset as "A B" not "AB". To get
% "AB" then you have to do: "\textbf{A}\textbf{B}"
% \thanks is no different in this regard, so shield the last } of each \thanks
% that ends a line with a % and do not let a space in before the next \thanks.
% Spaces after \IEEEmembership other than the last one are OK (and needed) as
% you are supposed to have spaces between the names. For what it is worth,
% this is a minor point as most people would not even notice if the said evil
% space somehow managed to creep in.



% The paper headers
\markboth{IEEE Transactions on Computers,~Vol.~xx, No.~x, February~2019}%
{Rosenkrantz \MakeLowercase{\textit{et al.}}: Symmetry Properties of Nested 
Canalyzing Functions} 

% The only time the second header will appear is for the odd numbered pages
% after the title page when using the twoside option.
% 
% *** Note that you probably will NOT want to include the author's ***
% *** name in the headers of peer review papers.                   ***
% You can use \ifCLASSOPTIONpeerreview for conditional compilation here if
% you desire.


% The publisher's ID mark at the bottom of the page is less important with
% Computer Society journal papers as those publications place the marks
% outside of the main text columns and, therefore, unlike regular IEEE
% journals, the available text space is not reduced by their presence.
% If you want to put a publisher's ID mark on the page you can do it like
% this:
%\IEEEpubid{0000--0000/00\$00.00~\copyright~2015 IEEE}
% or like this to get the Computer Society new two part style.
%\IEEEpubid{\makebox[\columnwidth]{\hfill 0000--0000/00/\$00.00~\copyright~2015 IEEE}%
%\hspace{\columnsep}\makebox[\columnwidth]{Published by the IEEE Computer Society\hfill}}
% Remember, if you use this you must call \IEEEpubidadjcol in the second
% column for its text to clear the IEEEpubid mark (Computer Society jorunal
% papers don't need this extra clearance.)

% use for special paper notices
%\IEEEspecialpapernotice{(Invited Paper)}



% for Computer Society papers, we must declare the abstract and index terms
% PRIOR to the title within the \IEEEtitleabstractindextext IEEEtran
% command as these need to go into the title area created by \maketitle.
% As a general rule, do not put math, special symbols or citations
% in the abstract or keywords.
%% \IEEEtitleabstractindextext{%


\newcommand{\QED}{\hfill\rule{2mm}{2mm}}
\newcommand{\qed}{\hfill$\Box$}

\newcommand{\cpoly}{\textbf{P}}
\newcommand{\cnp}{\textbf{NP}}
\newcommand{\cnump}{\textbf{\#P}}
\newcommand{\wtg}{\mbox{$\mathcal{G}$}}

\newcommand{\arr}{\mbox{$\:\longrightarrow\:$}}

\smallskip

\begin{abstract}
%%The abstract goes here.
Many researchers have studied symmetry properties of 
various Boolean functions. 
A class of Boolean functions, 
called \textbf{nested canalyzing functions} (NCFs),
has been used to model certain biological phenomena.
We identify some interesting relationships between NCFs, symmetric 
Boolean functions and a generalization of symmetric Boolean functions,
which we call $r$-symmetric functions (where $r$ is the symmetry level).
Using a normalized representation for NCFs, we develop a 
characterization of when two variables of an NCF are symmetric.
Using this characterization, we show 
that the symmetry level of an NCF $f$
can be easily computed given a standard representation of $f$.
%% (In contrast, we prove that even approximating the symmetry level of
%% a general Boolean function to within any factor $\geq 1$ is \cnp-hard.) 
We also present an algorithm for testing whether 
a given $r$-symmetric function is an NCF.
Further, we show that for any NCF $f$ with $n$ variables, the notion of
strong asymmetry considered in the literature is equivalent to
the property that $f$ is $n$-symmetric. 
We use this result to derive a closed form expression for the
number of $n$-variable Boolean functions 
that are NCFs and strongly asymmetric.
We also identify all the Boolean functions that are NCFs 
and symmetric. 

\end{abstract}

% Note that keywords are not normally used for peerreview papers.
\begin{IEEEkeywords}
Boolean functions, Nested canalyzing functions, Symmetry.
\end{IEEEkeywords}
%%}

% make the title area
%% \maketitle  <--- Using peerreview mode.



% To allow for easy dual compilation without having to reenter the
% abstract/keywords data, the \IEEEtitleabstractindextext text will
% not be used in maketitle, but will appear (i.e., to be "transported")
% here as \IEEEdisplaynontitleabstractindextext when the compsoc 
% or transmag modes are not selected <OR> if conference mode is selected 
% - because all conference papers position the abstract like regular
% papers do.

%% \IEEEdisplaynontitleabstractindextext

% \IEEEdisplaynontitleabstractindextext has no effect when using
% compsoc or transmag under a non-conference mode.



% For peer review papers, you can put extra information on the cover
% page as needed:
% \ifCLASSOPTIONpeerreview
% \begin{center} \bfseries EDICS Category: 3-BBND \end{center}
% \fi
%
% For peerreview papers, this IEEEtran command inserts a page break and
% creates the second title. It will be ignored for other modes.

%% \IEEEpeerreviewmaketitle

%%% Newcommands needed for the paper.




%%\IEEEraisesectionheading{\section{Introduction}\label{sec:intro}}

% Computer Society journal (but not conference!) papers do something unusual
% with the very first section heading (almost always called "Introduction").
% They place it ABOVE the main text! IEEEtran.cls does not automatically do
% this for you, but you can achieve this effect with the provided
% \IEEEraisesectionheading{} command. Note the need to keep any \label that
% is to refer to the section immediately after \section in the above as
% \IEEEraisesectionheading puts \section within a raised box.

% The very first letter is a 2 line initial drop letter followed
% by the rest of the first word in caps (small caps for compsoc).
% 
% form to use if the first word consists of a single letter:
% \IEEEPARstart{A}{demo} file is ....
% 
% form to use if you need the single drop letter followed by
% normal text (unknown if ever used by the IEEE):
% \IEEEPARstart{A}{}demo file is ....
% 
% Some journals put the first two words in caps:
% \IEEEPARstart{T}{his demo} file is ....
% 
% Here we have the typical use of a "T" for an initial drop letter
% and "HIS" in caps to complete the first word.

%% \IEEEPARstart{T}{his} demo file is intended to serve as a ``starter file''
%% for IEEE Computer Society journal papers produced under \LaTeX\ using
%% IEEEtran.cls version 1.8b and later.

% You must have at least 2 lines in the paragraph with the drop letter
% (should never be an issue)

\section{Introduction} 
\label{sec:intro}

\subsection{Canalyzing and Nested Canalyzing Functions}
\label{sse:ncf_def}

\IEEEPARstart{T}{he} class of \textbf{canalyzing} Boolean functions, introduced 
in \cite{Kauffman-1969}, is defined as follows.

\begin{definition}\label{def:canalyzing}
Given a set $X = $ $\{x_1, x_2, \ldots, x_n\}$ of $n$  Boolean variables,
a Boolean function $f(x_1, x_2, \ldots, x_n)$ over $X$ is \textbf{canalyzing}
if there is a variable $x_i \in X$ and values $a$, $b$ $\in \{0,1\}$ such that
\[
f(x_1, \ldots, x_{i-1}, a, x_{i+1}, \ldots, x_n) ~=~ b, 
\]
for all combinations of values assigned to the variables in $X - \{x_i\}$.
\end{definition}

\medskip

\noindent
\textbf{Example 1:}~ Consider the function 
$f(x_1, x_2, x_3)$ $~=~$ $\overline{x_1} \wedge (x_2 \vee \overline{x_3})$.
This function is canalyzing since if we set to $x_1 = 1$,
$f(1, x_2, x_3) ~=~ 0$, for all combinations of values for 
$x_2$ and $x_3$. %\qed

\medskip

Another class of Boolean functions, called \textbf{nested canalyzing functions} (NCFs),
was introduced later in \cite{Kauffman-etal-2003} to carry out a detailed
analysis of the behavior of certain biological systems.
We follow the presentation in \cite{Layne-2011} in defining NCFs.
(For a Boolean value $b$,~ the complement is denoted by $\overline{b}$.)

\begin{definition}\label{def:nested_canalyzing}
Let $X = $ $\{x_1, x_2, \ldots, x_n\}$ denote a set of $n$  Boolean variables.
Let $\pi$ be a permutation of $\{1, 2, \ldots, n\}$.
A Boolean function $f(x_1, x_2, \ldots, x_n)$ over $X$ is \textbf{nested canalyzing}
in the variable order $x_{\pi(1)}, x_{\pi(2)}, \ldots, x_{\pi(n)}$ with
\textbf{canalyzing values} $a_1, a_2, \ldots, a_n$ and 
\textbf{canalyzed values} $b_1, b_2, \ldots, b_n$ 
if $f$ can be expressed in the following form:
\[
f(x_1, x_2, \ldots, x_n) ~=~ 
   \begin{cases}
       \:b_1 & \mathrm{if~~} x_{\pi(1)} ~=~ a_1 \\
       \:b_2 & \mathrm{if~~} x_{\pi(1)} ~\neq~ a_1 \mathrm{~~and~~}\\
             & x_{\pi(2)} ~=~ a_2 \\
       \:\vdots & \vdots \\ %\smallskip
       \:b_n & \mathrm{if~~} x_{\pi(1)} ~\neq~ a_1 \mathrm{~~and~~} \ldots \\
             & x_{\pi(n-1)} ~\neq~ a_{n-1} \mathrm{~~and~~}\\
             & x_{\pi(n)} ~=~ a_n \\ %\smallskip
       \:\overline{b_n} & \mathrm{if~~} x_{\pi(1)} ~\neq~ a_1 \mathrm{~~and~~} \ldots\\
                        & x_{\pi(n)} ~\neq~ a_n \\
   \end{cases}
\]
\end{definition}
%% \clearpage
For convenience, we will use a notation introduced in \cite{Stearns-etal-2018}
to represent NCFs.
For $1 \leq i \leq n$, line $i$ of this representation has the form

\medskip

\noindent
\hspace*{0.5in} $x_{\pi(i)}:~ a_i ~~\longrightarrow~~ b_i$

\medskip

\noindent 
where $x_{\pi(i)}$ is the \textbf{canalyzing variable} that is
\textbf{tested} in line $i$, 
and $a_i$ and $b_i$ are respectively the canalyzing and 
canalyzed values in line $i$,~ $1 \leq i \leq n$.
Each such line is called a \textbf{rule}.
When none of the conditions ``$x_{\pi(i)} ~=~ a_i$" 
is satisfied, we have line $n+1$ with the ``Default" rule
for which the canalyzed value is~ $\overline{b_n}$: 

\medskip

\noindent
\hspace*{0.5in} Default:~ $\overline{b_n}$

\medskip
\noindent
As in \cite{Stearns-etal-2018}, we will refer to the above specification
of an NCF as the \textbf{simplified representation} and assume
(without loss of generality) that each NCF is specified in this manner.
The simplified representation provides the following simple
computational view of an NCF. 
The lines that define an NCF are 
considered sequentially in a top-down manner.
The computation stops at the first line where the 
specified condition is satisfied, and the value of the function
is the canalyzed value on that line. 
We now present an example of an NCF using the two representations
mentioned above.


\medskip
\noindent
\textbf{Example 2:}~ Consider the function 
$f(x_1, x_2, x_3) ~=~ \overline{x_1} \wedge (x_2 \vee \overline{x_3})$
used in Example~1.
This function is nested canalyzing using the identity permutation $\pi$ on $\{1,2,3\}$
with canalyzing values $1,1,0$ and canalyzed values $0, 1, 1$.
We first show how this function can be expressed using the syntax of
Definition~\ref{def:nested_canalyzing}.

\[
f(x_1, x_2, x_3) ~=~ 
   \begin{cases}
       \:0 & \mathrm{if~~} x_{1} ~=~ 1 \\
       \:1 & \mathrm{if~~} x_{1} ~\neq~ 1 \mathrm{~~and~~}
            x_{2} ~=~ 1 \\
       \:1 & \mathrm{if~~} x_{1} ~\neq~ 1 \mathrm{~~and~~}
            x_{2} ~\neq~ 1 \mathrm{~~and~~}\\ 
           &x_{3} = 0 \\
       \:0 & \mathrm{if~~} x_{1} ~\neq~ 1 \mathrm{~~and~~} 
             x_{2} ~\neq~ 1 \mathrm{~~and~~}\\ 
           & x_{3} ~\neq~ 0 \\ 
   \end{cases}
\]

\medskip
\noindent
A simplified representation of the same function is as follows.

\bigskip

\noindent
\begin{tabular}{ll}
\hspace*{0.25in} & $x_1:~$  $1 ~\longrightarrow~ 0$ \\ [0.5ex]
\hspace*{0.25in} & $x_2:~$  $1 ~\longrightarrow~ 1$ \\ [0.5ex]
\hspace*{0.25in} & $x_3:~$  $0 ~\longrightarrow~ 1$ \\ [0.5ex]
\hspace*{0.25in} & Default:~ $0$ \\
\end{tabular}

\noindent
%\qed

\iffalse
%%%%%%%%%%%%%%%%%%%%%%%%%%%%%%%%%%%%%%%%%%%%%%%
\noindent
We will also consider a more general form of NCFs defined below.

\begin{definition}\label{def:generalized ncf}
A {\bf generalized NCF} is a function represented as either a constant
or an NCF representation of a subset (not necessarily proper) 
of the function's variables.
\end{definition}

\noindent
\textbf{Example 3:}~ For $b \in \{0,1\}$, the constant function which takes 
on the value $b$ for every input is a generalized NCF since 
it can be represented by the following rule: 

\smallskip

\hspace*{1.1in} Default:~ $b$ 

\smallskip

\noindent
The following generalized NCF specification for 
a function $f(x_1, x_2, x_3, x_4)$ indicates that the function
depends only on variables $x_1$ and $x_3$.

\medskip

\noindent
\begin{tabular}{ll}
\hspace*{1.1in} & $x_1:~$  $0 ~\longrightarrow~ 1$ \\ [1ex]
\hspace*{1.1in} & $x_3:~$  $1 ~\longrightarrow~ 0$ \\ [1ex]
\hspace*{1.1in} & Default:~ $1$ \\
\end{tabular}

\noindent
%\qed

%% \bigskip

%% \noindent
%% \textbf{Note:} If a generalized NCF is not a constant, 
%% there is an assignment to its variables
%% such that the value of the function is 0,
%% and there is an assignment to its variables
%% such that the value of the function is 1.
%%%%%%%%%%%%%%%%%%%%%%%%%%%%%%%%%%%%%%%%%%%%%%%
\fi

\subsection{Symmetric Boolean Functions}
\label{sse:symmetry}

A pair of variables of a Boolean function $f(x_1, x_2, \ldots, x_n)$ is
said to be \textbf{symmetric} if their values can be interchanged without
affecting the value of the function.
As a simple example, each pair of variables in the function 
$f_2(x_1, x_2, x_3)$ = $x_1 \oplus x_2 \oplus x_3$ is symmetric,
where `$\oplus$' represents the Exclusive-Or operator.
This form of interchange symmetry partitions the set of variables into a set of
\textbf{symmetry groups}, where the members of each group are pairwise symmetric.
A Boolean function $f$ is 
$r$-\textbf{symmetric} if it has at most $r$ symmetry groups.
In this case, the value of $f$
depends only on how many of the variables in each symmetry group have the value 1.
We say that $f$ is \textbf{properly} $r$-\textbf{symmetric} if
it is $r$-symmetric, but not $(r-1)$-symmetric.
For example, the function $f_3(x_1, x_2, x_3) = (x_1 \wedge x_2) \vee\, \overline{x_3}$~
is not 1-symmetric since $f_3(1, 0, 1) \neq f_3(1, 1, 0)$; however, 
it is 2-symmetric with the symmetric groups being $\{x_1, x_2\}$ and $\{x_3\}$.
For a Boolean function $f$, the integer $r$ such that $f$ is properly $r$-symmetric
will be referred to as the \textbf{symmetry level} of $f$.
Thus, the symmetry level of $f$ is the smallest integer $r$ such that
$f$ is $r$-symmetric.

In the literature (see e.g., \cite{Crama-Hammer-2011,HT-2016,Toth-etal-1977}),
a 1-symmetric function $f$ is referred to simply
as a \textbf{symmetric} function.
Thus, in any symmetric function, each pair of variables is symmetric.
As a simple example, the function $f_2(x_1, x_2, x_3)$ = 
$x_1 \oplus x_2 \oplus x_3$ (defined above) is symmetric.
If $f$ is a symmetric function, then for any
input $(a_1, a_2, \ldots, a_n)$ to $f$, where $a_i \in \{0,1\}$ for
$1 \leq i \leq n$, and any permutation $\pi$ of $\{1, 2, \ldots, n\}$,
$f(a_1, a_2, \ldots, a_n)$ = $f(a_{\pi(1)}, a_{\pi(2)}, \ldots, a_{\pi(n)})$.
The class of $r$-symmetric functions have also been
studied in the literature on discrete dynamical systems (see e.g., 
\cite{Barrett-etal-2007,Rosenkrantz-etal-2015,MR-2007}).

Other forms of symmetry which can be more general than the permutations 
corresponding to symmetry groups have been considered (see e.g.,
\cite{Maurer-2015,KS-2000}).
A Boolean function $f(x_1, x_2, \ldots, x_n)$
is \textbf{strongly asymmetric}
if for any permutation $\pi$ of $\{1, 2, \ldots, n\}$
\emph{except} the identity permutation,
there exists an input $(a_1, a_2, \ldots,  a_n)$
to $f$ such that $f(a_1, a_2, \ldots, a_n)$ $\neq$
$f(a_{\pi(1)}, a_{\pi(2)}, \ldots, a_{\pi(n)})$.
In general, there are functions with $n$ variables that are properly
$n$-symmetric, but not strongly asymmetric \cite{KS-2000}.
However, in the case of NCFs with $n$ variables, we will show 
(see Section~\ref{sec:ncf_and_symmetry}) that the notion of 
strong asymmetry coincides with that 
of being properly $n$-symmetric.

%\noindent

\subsection{Summary of Results}
\label{sse:contrib}

Our focus is on the relationships between 
NCFs and symmetric Boolean functions.
Using a normalized representation for NCFs 
(defined in Section~\ref{sec:prelim}), we develop a
characterization of when two variables of an NCF are symmetric.
Using this characterization, we show
that the symmetry level of an NCF $f$
can be easily computed given a normalized representation of $f$.
(In contrast, we show that 
one cannot even efficiently approximate the symmetry level of
a general Boolean function to within any factor $\geq 1$, 
unless \textbf{P} = \cnp.)
We also show that for any NCF $f$ with $n$ variables, the notion of
strong asymmetry considered in the literature is equivalent to
the property that $f$ is $n$-symmetric.
We use this result to derive a closed form expression for the
number of $n$-variable Boolean functions
that are NCFs and strongly asymmetric.
In addition, we identify all the Boolean functions that are 
canalyzing and symmetric as well as those that 
are NCFs and symmetric.

\subsection{Related Work}
\label{sse:related}

The term \textbf{canalization}, coined by
Waddington \cite{Waddington-1942}, is generally used to describe
the stability of a biological system with changes
in external conditions.
In 1969, Kauffman \cite{Kauffman-1969} introduced canalyzing Boolean functions
to explain the stability of gene regulatory networks.
The subclass of NCFs was
introduced later by Kauffman et al. \cite{Kauffman-etal-2003} 
to facilitate a rigorous analysis of the Boolean network model
for gene regulatory networks.
It is known that the class of NCFs coincides with that
of unate cascade Boolean functions \cite{Jarrah-etal-2007}.
In the literature on computational learning theory,
NCFs are referred to as $1$-\textbf{decision lists} \cite{KV-1994}.
Many researchers have pointed out the usefulness of NCFs 
in modeling biological phenomena 
(e.g., \cite{Layne-2011,
Layne-etal-2012,Li-etal-2011,Li-etal-2012,Li-etal-2013}).
Properties of NCFs such as sensitivity and 
stability have also been studied in the 
literature \cite{Kauffman-etal-2004, Layne-2011,Layne-etal-2012,
Li-etal-2011,Li-etal-2013,Klotz-etal-2013, Stearns-etal-2018}. 

Over the years, a considerable amount of research on detecting symmetry properties 
of Boolean functions has been reported in the literature
(e.g., \cite{BS-1968,Biswas-1970,TM-1996,KS-2000,Darga-etal-2008,Maurer-2015}).
Several references have observed the usefulness of detecting symmetries
in automated logic synthesis (e.g., \cite{AP-2008,Hu-etal-2008,Darga-etal-2008}).
Reference \cite{Maurer-2015} provides a thorough discussion of known
symmetry detection algorithms and presents a new universal algorithm 
for detecting any form of permutation-based symmetry in Boolean functions. 
To our knowledge, relationships between NCFs and symmetric
Boolean functions have not been addressed in the literature.
   %% Introduction (definitions, summary of results and related work)

\section{Other Definitions and Preliminary Results}
\label{sec:prelim}

\subsection{Overview}

We present some definitions and recall
a known result regarding NCFs which allow us
to define a normalized representation for NCFs.
This representation plays an important role in several
problems considered in later sections.
We also show (Theorem~\ref{thm:approx_sym_level_hard}) 
that, in general, the symmetry level
of a Boolean function cannot be efficiently approximated to within any
factor $\rho \geq 1$, unless \textbf{P} = \cnp.

\medskip

\subsection{Layers and Normalized Representation}
\label{sse:ncf_layer}

Recall that any NCF $f$ with $n$ variables,
denoted by $x_1$, $x_2$, $\ldots$, $x_n$,~
is specified using a variable ordering 
$x_{\pi(1)}$, $x_{\pi(2)}$, $\ldots$,  $x_{\pi(n)}$,
where $\pi$ is a permutation of $\{1, 2, \ldots, n\}$.
Throughout this paper, we will assume without loss of generality
that the variable ordering uses the identity permutation 
so that $x_i$ is the variable tested in line $i$, $1 \leq i \leq n$. 
We assume that an NCF $f$ with $n$
variables is specified using the simplified representation with $n$ lines. 
Unless otherwise mentioned, the ``Default" line (which is assigned the number $n+1$)
in the simplified representation of an NCF is \emph{not} considered
in the results stated in this section.
We will use the following result from \cite{Stearns-etal-2018}.

\begin{observation} \label{obs:ncf_transformations}
Let $f$ be an NCF with $n$ variables specified using $n$ lines
in the simplified representation. 
%\begin{description}
\begin{enumerate}
\item 
For any $q \geq 2$ and for any $i$, $1 \leq i \leq n-q+1$,
if the $q$ consecutive lines $i$, $i+1$, $\ldots$, $i+q-1$~
have the same canalyzed value, then the function remains
unchanged if these $q$ lines are permuted in any order
without changing the other lines.

\item 
Suppose lines $n-1$ and $n$ in the specification of $f$ 
have complementary canalyzed values.
Then, the function remains unchanged 
if the canalyzing value and canalyzed value in line $n$
are both complemented. 
(Here, the value on the ``Default" line is also complemented.)
\QED
\end{enumerate}
%\end{description}
\end{observation}

%%\noindent
\citet{Li-etal-2013} defined the concept of a 
{\bf layer} of an NCF in terms of
an algebraic representation of the NCF as an extended monomial.
For our purposes, we use the following definition based on the
simplified representation of NCFs.

\begin{definition}
\label{def:layer}
A {\bf layer} of an NCF representation is a maximal length sequence 
of lines with the same canalyzed value.
\end{definition}
%%Note that the ``Default" line in the representation of an NCF is
%%\emph{not} considered in the definition of a layer.

\noindent
\textbf{Example 4:}~ Consider the following function $f$ of six variables
$x_1$, $x_2$, $\ldots$, $x_6$. 

\bigskip

\noindent
\begin{tabular}{ll}
\hspace*{0.25in} & $x_1:~$  $1 ~\longrightarrow~ 0$ \\ [0.5ex]
\hspace*{0.25in} & $x_2:~$  $0 ~\longrightarrow~ 0$ \\ [0.5ex]
\hspace*{0.25in} & $x_3:~$  $0 ~\longrightarrow~ 0$ \\ [0.5ex]
\hspace*{0.25in} & $x_4:~$  $1 ~\longrightarrow~ 1$ \\ [0.5ex]
\hspace*{0.25in} & $x_5:~$  $1 ~\longrightarrow~ 1$ \\ [0.5ex]
\hspace*{0.25in} & $x_6:~$  $1 ~\longrightarrow~ 0$ \\ [0.5ex]
\hspace*{0.25in} & Default:~ $1$ \\
\end{tabular}

\medskip
\noindent
This function has 3 layers, with 
Layer~1 consisting of the lines with $x_1$, $x_2$ and $x_3$, 
Layer~2 consisting of the lines with $x_4$ and $x_5$, and
Layer~3 consisting of the line with $x_6$.  %\qed

\medskip
From Part~1 of Observation~\ref{obs:ncf_transformations},
it follows that the lines within the same layer of an NCF
representation can be permuted without changing the function.
Part~2 of Observation~\ref{obs:ncf_transformations} points
out that for any NCF $f$ with $n \geq 2$ variables, there is a simplified 
representation in which lines $n-1$ and $n$ are in the same layer.
We can now define the notion of a \textbf{default-normalized} 
representation for NCFs.

\begin{definition}
\label{def:normalized}
Let $f$ be an NCF with $n \geq 2$ variables specified using
the simplified representation with 
the variable ordering $\langle x_1, x_2, \ldots, x_n\rangle$.
We say that the representation is {\bf default-normalized} if
lines $n-1$ and $n$ have the same canalyzed value.
\end{definition}

\noindent
\textbf{Example 5:}~ 
Consider the NCF $f$ shown in Example~4. 
This representation is \emph{not} default-normalized since lines 5 and 6
have different canalyzed values.
If we change line 6 to ``$x_6$:~ 0 $\longrightarrow$ 1" and the
value on the ``Default" line to 0, then we obtain 
a default-normalized representation. 

\medskip
From Observation \ref{obs:ncf_transformations},
we note that in polynomial time, for any NCF representation with more than one line,
we can first modify the last line if necessary 
so that it has the same canalyzed value as the preceding line,
thereby obtaining an equivalent default-normalized NCF representation.
We state this formally below.

\begin{observation}\label{obs:normalization_poly}
Given an NCF representation for a Boolean function $f$, a default-normalized
representation for $f$ can be obtained in polynomial time. \QED
\end{observation}

In view of Observation~\ref{obs:normalization_poly},
we will assume from now on that any NCF is specified in default-normalized form.

\subsection{Complexity of Approximating the Symmetry Level of a 
General Boolean Function}
\label{sse:symmetry_level_hardness}

Recall that the symmetry level of a Boolean function $f$ is the smallest
integer $r$ such that the following condition holds:
the  inputs to $f$ can be partitioned into $r$ subsets 
(symmetry groups) so that the value of $f$ depends only on 
how many of the inputs in each group have the value 1.
We now show that given a Boolean function $f$
in the form of a Boolean expression, it is \cnp-hard to 
approximate\footnote{An algorithm approximates the symmetry
level of a Boolean function $f$ within the factor $\rho$ if it finds a
partition of the inputs to $f$ into at most $\rho \times r^*$ symmetry
groups, where $r^*$ is the symmetry level of $f$.}
the symmetry level of $f$ to within any factor $\rho \geq 1$.
This result holds even when the function is given as an expression
in conjunctive normal form (CNF), that is, in the form of a conjunction
of clauses where each clause is a disjunction of literals. 

\newcommand{\cala}{\mbox{$\mathcal{A}$}}

\begin{theorem}\label{thm:approx_sym_level_hard}
Unless \textbf{P} = \cnp,
for any $\rho \geq 1$, there is no polynomial time 
algorithm for approximating the 
symmetry level of a Boolean function $f$ 
specified as a Boolean expression to within the factor $\rho$.
This result holds even when $f$ is a CNF expression.
\end{theorem}

\noindent
\textbf{Proof:}~ For some $\rho \geq 1$, suppose there is an efficient algorithm \cala{} 
that approximates the symmetry level of a Boolean function
specified as a Boolean expression to within the factor $\rho$.
Without loss of generality, we can assume that $\rho$ is an integer.
We will show that \cala{} can be used to efficiently solve
the CNF Satisfiability problem (SAT) which is known 
to be \cnp-hard \citep{GJ-1979}.

Let $g$ be a CNF formula representing an instance of SAT.
Let $X$ = $\{x_1, x_2, \ldots, x_n\}$ denote the set of
Boolean variables used in $g$.
We create another CNF formula $f$ as follows.
Let $Y = \{y_1, y_2, \ldots, y_{\rho+1}\}$ and
$Z = \{z_1, z_2, \ldots, z_{\rho+1}\}$ be two 
new sets of variables, with each set containing $\rho+1$ variables.
The expression for $f$, which is a function of $n+2\rho+2$
variables, namely $x_1, \ldots, x_n$, $y_1, \ldots, y_{\rho+1}$,
$z_1, \ldots, z_{\rho+1}$, is the following:
\[
    g(x_1, \ldots x_n) \wedge (y_1 \vee \overline{z_1}) 
                       \wedge (y_2 \vee \overline{z_2}) 
                       \wedge \ldots \wedge
                              (y_{\rho+1} \vee \overline{z_{\rho+1}}).
\]
Since $g$ is a CNF formula, so is $f$.
We have the following claims.

\medskip

\noindent
\textbf{Claim 1:}~ If $g$ is \emph{not} satisfiable, 
the symmetry level of $f$ is 1.

\smallskip

\noindent
\textbf{Proof of Claim 1:}~ If $g$ is not satisfiable, $f$ is also
not satisfiable; that is, for all inputs, the value of $f$ is 0.
Thus, all the variables in $X \cup Y \cup Z$ can be included in one
symmetry group.
The value of $f$ is 0 regardless of how many variables in the
group have the value 1.  \hfill$\Box$

\medskip

\noindent
\textbf{Claim 2:}~ If $g$ \emph{is} satisfiable, 
the symmetry level of $f$ is at least $\rho+1$.

\smallskip

\noindent
\textbf{Proof of Claim 2:}~ Suppose $g$ is satisfiable.
We argue that if $i \neq j$, variables $y_i$ and $y_j$
cannot be in the same symmetry group for the function $f$.
To see this, consider any two variables $y_i$ and $y_j$ with $i \neq j$,
and construct an assignment $\alpha$ to
the variables of $f$ in the following manner.
\begin{enumerate}
\item For the variables in $X$, choose an assignment that
satisfies $g$.
\item Let $y_i = 0$, $z_i = 1$, $y_j = 1$ and $z_j = 0$.
\item For $1 \leq p \leq \rho+1$, $p \neq i$ and $p \neq j$,
let $y_p = 1$ and $z_p$ = 0.
\end{enumerate}
Since this assignment $\alpha$ sets the clause $(y_i \vee \overline{z_i})$ to 0,
the value of $f$ under the assignment $\alpha$ is 0.
Now, consider the assignment $\alpha'$ that is obtained by interchanging
the values of $y_i$ and $y_j$ in $\alpha$ without changing the
values assigned to the other variables. 
It can be seen that under assignment $\alpha'$, the value of $f$ is 1.
In both $\alpha$ and $\alpha'$, the number of 1-valued variables in 
the set $\{y_i, y_j\}$ is 1; however, the two assignments lead to 
different values for $f$.
Therefore, if $g$ is satisfiable, for $i \neq j$, $y_i$ and $y_j$ cannot
be in the same symmetry group for $f$.
Since there are $\rho+1$ variables in $Y$, the number of symmetry groups 
for $f$ must be at least $\rho+1$, and this 
completes our proof of Claim~2. \hfill$\Box$

\medskip
 
We now continue with the proof of Theorem~\ref{thm:approx_sym_level_hard}.
Suppose we execute the approximation algorithm \cala{} 
on the function $f$ defined above.
If $g$ is \emph{not} satisfiable, then from Claim~1, 
the symmetry level of $f$ is 1. 
Since \cala{} is a $\rho$-approximation algorithm,
it should produce at most $\rho$ symmetry groups.
On the other hand, if $g$ is satisfiable, by Claim~2, the symmetry
level of $f$ is at least $\rho+1$; so, \cala{} will produce at least
$\rho+1$ groups.
In other words, $g$ is not satisfiable iff the number
of symmetry groups produced by \cala{} is at most $\rho$.
Since \cala{} runs in polynomial time, we have an efficient 
algorithm for SAT, contradicting the assumption that 
\textbf{P} $\neq$ \cnp.  \QED

\medskip

In contrast to the above result,
we will show in the next section that the symmetry level 
of an NCF can be computed efficiently.
   %% Preliminaries and normalized representation for NCFs

\section{Symmetry of Nested Canalyzing Functions}
\label{sse:ncf_and_symmetry}

\subsection{Overview}
\label{sse:res_overview}


\subsection{Symmetric Canalyzing and Nested Canalyzing Functions}
\label{sse:sym_and_cf_ncf}

\begin{theorem}\label{thm:ncf_symmetric}
The only symmetric canalyzing functions are OR, AND, NOR, NAND, 
the constant function 0, and the constant function 1.

The only symmetric NCFs are OR, AND, NOR, and NAND.
\end{theorem}
\noindent
\textbf{Proof:}~
Suppose symmetric function $f$ is a canalyzing function. Then there
is a variable $x$, and values $a$ and $b$ such that whenever $x =
a$, function $f$ has value $b$.  Since $f$ is symmetric, $f$  has
value $b$ whenever any of its variables have value $a$.  Thus, if
at least one of its variables has value $a$, then $f$ has value
$b$.  If $f$ has value $\bar{b}$ when none of its variables has
value $a$, the four possible combinations of values for $a$ and $b$
correspond to the four functions OR, AND, NOR, and NAND.  If $f$
has value $b$ when none of its variables has value $a$, then $f$
is the constant function $b$.

Suppose that $f$ is a NCF.  Since constant functions are not NCFs,
$f$ has value $\bar{b}$ when none of its variables has value $a$.
The four possible combinations of values for $a$ and $b$ correspond
to the four functions OR, AND, NOR, and NAND, each of which is a
NCF.  \QED

\subsection{Symmetric Pairs of Variables in an NCF: A Characterization
and Its Applications}
\label{sse:ncf_strong_sym}

\begin{theorem}\label{thm:ncf_symmetric_variables}
A pair of variables of a NCF are symmetric iff
in the normalized representation of the NCF
they occur in the same layer and have the same canalyzing value.
\end{theorem}
\noindent
\textbf{Proof:}~
Suppose that a pair of variables occur in the same layer of a
canalyzing representation, and the lines containing these variables
have the same canalyzing value.  Then in any input assignment, the
values of these two variables can be interchanged, and the evaluation
of the lines in that layer will have the same effect.  Thus, the
variables are symmetric.

Now, let $f$ be the normalized NCF representation for the function.
For purposes of notational simplicity, wlog we assume that the
canalyzing variable in line $i$ of $f$ is $x_i$, $1 \leq i \leq n$.
Let  $x_j$ and $x_k$ be a pair of variables that occur in different
layers of $f$, or occur in the same layer, but with different
canalyzing values.  Wlog, assume that $j < k$, so that the line for
$x_j$ occurs above the line for $x_k$.

Suppose that line $i$ of $f$, $1 \leq i \leq n$, is 

\noindent
$(x_i : a_i \rightarrow b_i)$. \\
Consider the following input assignment $\alpha = (c_1, c_2, \ldots,  c_n)$ 
to the variables of $f$.

\noindent
For $1 \leq i < j$, set $c_i = \overline{a_i}$. \\
For $i = j$, set $c_j =a_j$. \\
For $i = k$, set $c_k =\overline{a_j}$. \\
For $j < i < k$, and for  $k < i \leq n$, 
if $b_i = b_j$, then set $c_i =  \overline{a_i}$,
else set $c_i =  a_i$.

Note that $f(\alpha) = b_j$, since the variables in all lines above
line $j$ have the complement of their canalyzing value, and variable
$x_j$ has its canalyzing value.

Now let $\alpha'$ be the assignment that is obtained from $\alpha$
by interchanging the values of variables $x_j$ and $x_k$, so that
after the interchange variable $x_j$ has value $\overline{a_j}$ and
variable $x_k$ has value $a_j$.  We will show that $f(\alpha') =
\overline{b_j}$, so variables $x_j$ and $x_k$ are not symmetric.

In line $k$ of $f$, canalyzing value $a_k$ is either the same or
the complement of canalyzing value $a_j$, and canalyzed output $b_k$
is either the same or the complement of canalyzed output $b_j$,
Thus, there are four possible cases for the form of line $k$.  We
now consider each of the four cases.

\noindent
{\bf Case 1:} $(x_k : a_j \rightarrow b_j)$. \\ 

Since line $k$ has
the same canalyzing value and canalyzed output as line $j$, line
$k$ must occur in a lower layer than that of line $j$.  Thus, there
is at least one line between line $j$ and line $k$ for which the
canalyzed output is $\overline{b_j}$.  Let line $q$ be the first
such line.  Note that for all $i$ such that $1 \leq i < q$, the
value of variable $x_i$ in assignment $\alpha'$ does not match the
canalyzing value of line $i$, but the value of variable $x_q$ does
match the canalyzing value of line $q$.  Since canalyzed output
$b_q$ equals $\overline{b_j}$, we have that $f(\alpha') =
\overline{b_j}$.

\noindent
{\bf Case 2:} $(x_k : a_j \rightarrow \overline{b_j})$. \\ 

Let line
$q$ be the first line such that $j < q \leq k$ and the canalyzed
output of line $q$ is $\overline{b_j}$.  Note that $q$ might possibly
equal $k$.  For all $i$ such that $1 \leq i < q$, the value of
variable $x_i$ in assignment $\alpha'$ does not match the canalyzing
value of line $i$, but the value of variable $x_q$ does match the
canalyzing value of line $q$.  Since canalyzed output $b_q$ equals
$\overline{b_j}$, we have that $f(\alpha') = \overline{b_j}$.

\noindent
{\bf Case 3:} $(x_k : \overline{a_j} \rightarrow b_j)$. \\ 

Suppose there is a line below line $j$ for which the canalyzed output is
$\overline{b_j}$.  Let $q$ be the first such line.  Note that for
all $i$ such that $1 \leq i < q$, the value of variable $x_i$ in
assignment $\alpha'$ does not match the canalyzing value of line
$i$, but the value of variable $x_q$ does match the canalyzing value
of line $q$.  Since canalyzed output $b_q$ equals $\overline{b_j}$,
we have that $f(\alpha') = \overline{b_j}$.

Now suppose there is no line below line $j$ for which the canalyzed
output is $\overline{b_j}$.  Then lines $j$ and $k$ both occur in
the last layer of $f$.  Thus, for every line $i$, $1 \leq i \leq
n$, the value of variable $x_i$ in assignment $\alpha'$ does not
match the canalyzing value of line $i$.  Consequently, $f(\alpha')
= \overline{b_j}$, the complement of the canalyzed output in the
last layer of $f$.

\noindent
{\bf Case 4:} $(x_k : \overline{a_j} \rightarrow \overline{b_j})$.  \\ 
Suppose there is a line, other than line $k$, below line $j$ for
which the canalyzed output is $\overline{b_j}$.  Let $q$ be the
first such line.  Note that for all $i$ such that $1 \leq i < q$,
the value of variable $x_i$ in assignment $\alpha'$ does not match
the canalyzing value of line $i$, but the value of variable $x_q$
does match the canalyzing value of line $q$.  Since canalyzed output
$b_q$ equals $\overline{b_j}$, we have that $f(\alpha') =
\overline{b_j}$.

Now suppose that line $k$ is the only line below line $j$ for which
the canalyzed output is $\overline{b_j}$.  Since line $j$ has
canalyzed output $b_j$, line $k$ is the only line in its layer.
Since $f$ is normalized, the last layer of $f$ contains at least
two lines.  Thus, there is a last layer following the layer containing
line $k$.  The canalyzed output of all the lines in this last layer
is $b_j$.  Thus, for every line $i$, $1 \leq i \leq n$, the value
of variable $x_i$ in assignment $\alpha'$ does not match the
canalyzing value of line $i$.  Consequently, $f(\alpha') =
\overline{b_j}$, the complement of the canalyzed output in the last
layer of $f$.  \QED

\begin{theorem}\label{thm:ncf_r_symmetric}
Suppose the normalized representation of a given NCF contains 
$r_1$ layers with only one distinct canalyzing value,
and $r_2$ layers with two distinct canalyzing values.
Then the function is properly $(r_1 + 2 r_2)$-symmetric.
\end{theorem}
\noindent
\textbf{Proof:}~
From Theorem~\ref{thm:ncf_symmetric_variables}, 
any pair of variables occurring in the same layer, with the same canalyzing value,
are symmetric.
Thus the function is at most $(r_1 + 2 r_2)$-symmetric.
Moreover, any pair of variables from different layers, or with different canalyzing values
are not symmetric, so there are at least $(r_1 + 2 r_2)$ symmetry groups.
\QED

\begin{corollary}\label{cor:ncf_not_rsymm}
For every $n \geq 2$, there is an $n$-variable NCF that is not $n-1$ symmetric.
\end{corollary}

\begin{corollary}\label{cor:ncf_r_symmetric_layers}
A NCF with a normalized nested canalyzing representation consisting
of $q$ layers is $2q$-symmetric, and is not $(q-1)$-symmetric.

An $r$-symmetric NCF has a normalized nested canalyzing representation
with at most $r-1$ layers, and has an unnormalized nested canalyzing
representation with at most $r$ layers.  \end{corollary}

\subsection{Strong Asymmetry and NCFs}
\label{sse:strong_asym_ncf}

Generalized forms of symmetry have been considered \cite{KS-2000}
in which a Boolean function is symmetric with respect to some permutation,
which can be more general than the permutations corresponding to symmetry groups.
We say that a Boolean function $f(x_1, x_2, \ldots, x_n)$ 
is \textbf{strongly asymmetric}
if for any permutation $\pi$ of $\{1, 2, \ldots, n\}$
\emph{except} the identity permutation,
there exists an input $(a_1, a_2, \ldots,  a_n)$ 
to $f$ such that $f(a_1, a_2, \ldots, a_n)$ $\neq$  
$f(a_{\pi(1)}, a_{\pi(2)}, \ldots, a_{\pi(n)})$.
In general, there are functions with $n$ variables that are properly
$n$-symmetric, but not strongly asymmetric \cite{KS-2000}.

We now show that any $n$ variable NCF that is properly $n$-symmetric
is also strongly asymmetric.
As a consequence, there are NCFs with $n$ variables
which are properly $n$-symmetric.

\begin{theorem}\label{thm:ncf_strong_asymmetry}
A NCF with $n$ variables is strongly asymmetric iff
it is properly $n$-symmetric.
\end{theorem}

\noindent
\textbf{Proof:}~ 
If a NCF is $(n-1)$-symmetric, then consider a permutation than
interchanges two variables from a symmetry group with at least two
members.  The value of the function is invariant under any such
permutation, so the function is not strongly asymmetric.

Now consider a $n$ variable NCF that is properly $n$-symmetric.
Let $f$ be the normalized NCF representation for the function.  For
purposes of notational simplicity, wlog we assume that the canalyzing
variable in line $i$ of $f$ is $x_i$, $1 \leq i \leq n$.  Let $\pi$
be any permutation of $\{1, 2, \ldots, n\}$ \emph{except} the
identity permutation.  We will construct an assignment $(c_1, c_2,
\ldots,  c_n)$ to the variables of $f$ such that $f(c_1, c_2, \ldots,
c_n)$ $\neq$ $f(c_{\pi(1)}, c_{\pi(2)}, \ldots, c_{\pi(n)})$.

Let $k$ be the smallest index such that $k \neq \pi(k)$.  Since
$\pi$ is a permutation, $k < n$.  Overall, for $1 \leq i < k$,
$\pi(i) = i$;
 $\pi(k) > k$; and for $k < i \leq n$, $\pi(i) \geq k$.

Suppose that line $i$ of $f$, $1 \leq i \leq n$, is \noindent $(x_i
: a_i \rightarrow b_i)$. \\ Assignment $(c_1, c_2, \ldots,  c_n)$
is constructed as follows.

\noindent
For $1 \leq i < k$, set $c_i = \overline{a_i}$. \\ For $i = k$, set
$c_k =a_k$. \\ For $i > k$, let $i' = \pi^{-1}(i)$, so that $\pi(i')
= i$.  Thus, after the permutation of values, variable $x_{i'}$
will have value $c_i$.  If $b_{i'} = b_k$, then set $c_i =
\overline{a_{i'}}$, else set $c_i =  a_{i'}$.

Since $c_k$ matches the canalyzing value of line $k$ of $f$, and
$c_i$ does not match the canalyzing value of any other earlier line
$i$, $1 \leq i < k$, we have that $f(c_1, c_2, \ldots,  c_n) = b_k$.

Now consider $f(c_{\pi(1)}, c_{\pi(2)}, \ldots, c_{\pi(n)})$.  We
first note that for $1 \leq i < k$, $c_{\pi(i)}$ does not match the
canalyzing value of line $i$.

Let $k' = \pi^{-1}(k)$, so that $\pi(k') = k$.  Canalyzing value
$a_{k'}$ is either the same or the complement of $a_k$, and canalyzed
output $b_{k'}$ is either the same or the complement of $b_k$, Thus,
there are four possible cases for the form of line $k'$ of $f$.  We
will show that in each of the four cases, $f(c_{\pi(1)}, c_{\pi(2)},
\ldots, c_{\pi(n)}) = \overline{b_k}$.

\noindent
{\bf Case 1:} $(x_{k'} : a_k \rightarrow b_k)$. \\ 

Since line $k'$
has the same canalyzing value and canalyzed output as line $k$, and
$f$ is properly $n$-symmetric, line $k'$ must occur in a lower layer
than that of line $k$.  Thus, there is at least one line between
line $k$ and line $k'$ for which the canalyzed output is $\overline{b_k}$.
Let line $q$ be the first such line.  Note that for all $i$ such
that $1 \leq i < q$, $c_{\pi(i)}$ does not match the canalyzing
value of line $i$, but $c_{\pi(q)}$ does match the canalyzing value
of line $q$.  Since canalyzed output $b_q$ equals $\overline{b_k}$,
we have that $f(c_{\pi(1)}, c_{\pi(2)}, \ldots, c_{\pi(n)}) =
\overline{b_k}$.

\noindent
{\bf Case 2:} $(x_{k'} : a_k \rightarrow \overline{b_k})$. \\

Let line $q$ be the first line such that $k < q \leq k'$ and
the canalyzed output of line $q$ is $\overline{b_k}$.
Note that for all $i$ such that $1 \leq i < q$, 
$c_{\pi(i)}$ does not match the canalyzing value of line $i$,
so $f(c_{\pi(1)}, c_{\pi(2)}, \ldots, c_{\pi(n)}) = \overline{b_k}$.

\noindent
{\bf Case 3:} $(x_{k'} : \overline{a_k} \rightarrow b_k)$. \\

Suppose there is a line below line $k$ for which the canalyzed
output is $\overline{b_k}$.  Let $q$ be the first such line.  Note
that for all $i$ such that $1 \leq i < q$, $c_{\pi(i)}$ does not
match the canalyzing value of line $i$, but $c_{\pi(q)}$ does match
the canalyzing value of line $q$.  Thus, $f(c_{\pi(1)}, c_{\pi(2)},
\ldots, c_{\pi(n)}) = \overline{b_k}$.

Now suppose there is no line below line $k$ for which the canalyzed
output is $\overline{b_k}$.  Then line $k$ occurs in the last layer
of $f$.  Since $f$ is both normalized and properly $n$-symmetric,
this last layer contains exactly two lines.  Thus $k = n-1$ and $k'
= n$.  Note that for all $i$, $c_{\pi(i)}$ does not match the
canalyzing value of line $i$.  Thus, $f(c_{\pi(1)}, c_{\pi(2)},
\ldots, c_{\pi(n)}) = \overline{b_k}$.

\noindent
{\bf Case 4:} $(x_{k'} : \overline{a_k} \rightarrow \overline{b_k})$. \\

Suppose there is a line, other than line $k'$, below line $k$ for
which the canalyzed output is $\overline{b_k}$.  Let $q$ be the
first such line.  Note that for all $i$ such that $1 \leq i < q$,
$c_{\pi(i)}$ does not match the canalyzing value of line $i$, but
$c_{\pi(q)}$ does match the canalyzing value of line $q$.  Thus,
$f(c_{\pi(1)}, c_{\pi(2)}, \ldots, c_{\pi(n)}) = \overline{b_k}$.

Now suppose that line $k'$ is the only line below line $k$ for which
the canalyzed output is $\overline{b_k}$.  Since line $k$ has
canalyzed output $b_k$, line $k'$ is the only line in its layer.
Since $f$ is normalized, there is a last layer following the layer
containing line $k'$.  Thus, for all $i$, $c_{\pi(i)}$ does not
match the canalyzing value of line $i$, so $f(c_{\pi(1)}, c_{\pi(2)},
\ldots, c_{\pi(n)}) = \overline{b_k}$.  \QED


Examples of properly $n$-symmetric $n$-variable NCFs satisfying the
conditions of Theorem \ref{thm:ncf_r_symmetric} are the following.
For each $n > 1$, let $f_n$ be the function defined by the following
formula.

$$x_1 \vee \bar{x}_2 ( x_3 \vee \bar{x}_4 (\cdots  ) )$$

For instance,
$$f_6 = x_1 \vee \bar{x}_2 ( x_3 \vee \bar{x}_4 ( x_5 \vee \bar{x}_6  ) ),$$
and
$$f_7 = x_1 \vee \bar{x}_2 ( x_3 \vee \bar{x}_4 ( x_5 \vee \bar{x}_6   x_7) ).$$

Function $f_n(x_1, x_2, \ldots, x_n)$ is the NCF corresponding to
the following NCF representation:

\medskip
\noindent
\hspace*{0.5in}
$x_1:~ (1 \rightarrow 1)$ \\
\hspace*{0.5in}
$x_2:~ (1 \rightarrow 0)$ \\
\hspace*{0.5in}
$x_3:~ (1 \rightarrow 1)$ \\
\hspace*{0.75in}
$\vdots$ 

\noindent
If $n$ is odd, the last line is 

\noindent
\hspace*{0.5in}
$x_n:~ (1 \rightarrow 1)$ 

\noindent
and if $n$ is even, the last line is 

\noindent
\hspace*{0.5in}
$x_n:~ (1 \rightarrow 0)$. \\

Note that if $x_i = 0$ for all $i$, $1 \leq i \leq n$, then the
value of $f_n$ is 0 if $n$ is odd and 1 if $n$ is even.  Also, note
that the above representation is unnormalized. To normalize the
representation, the last line would be changed to have canalyzing
value 0, and the same canalyzed output as the preceding line.






\iffalse
%%%%%%%%%%%%%%%%%%%%%%%%%%%%%%%%%%%%%%%%%%%%%\begin{theorem}\label{thm:ncf_asymmetry}
\begin{theorem}\label{thm:ncf_asymmetry}.
Consider the set of variables $x_i$ for $1 \leq i \leq n$ and 
let $f(x_1, x_2, \ldots, x_n)$ be the NCF defined by the following sequence
of lines:

\medskip
\noindent
\hspace*{0.5in}
$x_1:~ (1 \rightarrow 1)$ \\
\hspace*{0.5in}
$x_2:~ (1 \rightarrow 0)$ \\
\hspace*{0.5in}
$x_3:~ (1 \rightarrow 1)$ \\
\hspace*{0.75in}
$\vdots$ 

\noindent
If $n$ is odd, the last line is 

\noindent
\hspace*{0.5in}
$x_n:~ (1 \rightarrow 1)$ 

\noindent
and if $n$ is even, the last line is 

\noindent
\hspace*{0.5in}
$x_n:~ (1 \rightarrow 0)$. \\

\noindent
Let $\pi$ be any permutation of $\{1, 2, \ldots, n\}$
\emph{except} the identity permutation.
Then there exists an input $(a_1, a_2, \ldots,  a_n)$ 
to $f$ such that $f(a_1, a_2, \ldots, ,a_n)$ $\neq$  
$f(a_{\pi(1)}, a_{\pi(2)}, \ldots, a_{\pi(n)})$.
\end{theorem}

\noindent
\textbf{Proof:}~ 
Note that if $x_i = 0$ for all $i$, $1 \leq i \leq n$, then the
value of the function $f$ is 0 if $n$ is odd and 1 if $n$ is even.

First assume there is an odd $i$ such that $\pi(i)$ is even. 
Consider the input where $a_i$ is 1 and all others are 0. 
It can be verified that $f(a_1, a_2, \ldots, a_n)$ = 1 while
$f(a_{\pi(1)}, a_{\pi(2)}, \ldots, a_{\pi(n)})$ = 0.

\smallskip
Now assume the permutation $\pi$ is such that for each odd $i$, $\pi(i)$ is odd 
and for each even $i$, $\pi(i)$ is even.
Let $k$ be the smallest index such that $k \neq \pi(k)$.
Assume that $k$  is odd. 
Construct the input $(a_1, a_2, \ldots,  a_n)$ to $f$ as follows: 
\begin{enumerate}
\item Set $a_k$ to 1 and $a_i$ to 0 for all other odd $i$. 
\item For even $i$, set $a_i$ to 0 if $i < k$ and to 1 otherwise. 
\end{enumerate}
We have the following claim.

\smallskip
\noindent
\textbf{Claim:}~ For the input $(a_1, a_2, \ldots, a_n)$, 
$f(a_1, a_2, \ldots, a_n)$ = 1 and
$f(a_{\pi(1)}, a_{\pi(2)}, \ldots, a_{\pi(n)})$ = 0.

\smallskip
\noindent
\textbf{Proof of claim:}~ Note that the permutation
on the even indices maps 1's to 1's and leaves 0's in place.  
Before the permutation,  
$x_j = 0$ for $1 \leq j < k$ and $x_k = 1$,
so the canalyzing variable is $x_k$ and the function value is 1.
After the permutation, 
$x_j = a_j = 0$ for $1 \leq j < k$,
$x_k = 0$, and $x_{k+1} = 1$;
so the canalyzing variable is $x_{k+1}$ and the function value is 0.
This establishes the claim. 

A similar proof can be given when $k$ is even and this completes
the proof of Theorem~\ref{thm:ncf_asymmetry}.
\QED

\noindent
The following result is a consequence of the above theorem.

\begin{corollary}\label{cor:ncf_not_rsymm}
For every $n \geq 2$, there is an $n$-variable NCF that is not $n-1$ symmetric.
\end{corollary}

\noindent
\textbf{Proof:}~ 
Let $f$ be the Boolean function defined in Theorem~\ref{thm:ncf_asymmetry}.
We will show by contradiction that $f$ is not $r$-symmetric for any $r < n$.

Suppose $f$ is $r$-symmetric for some $r < n$.
Then, by definition, there is a partition of the set $X$ =
$\{x_1, x_2, \ldots, x_n\}$ of variables into at most $r$ subsets such that
the value of $f$ depends only on the number of variables with value 1
in each subset.
Since $r < n$, this partition has a subset, say $X_j$, 
such that $|X_j| \geq 2$.
Let $X_j = \{x_{j_1}, x_{j_2}, \ldots, x_{j_k}\}$, where $k \geq 2$.
Assume without loss of generality that $j_1 < j_2 < \ldots < j_k$.
Define a permutation $\pi$ of $\{1, 2, \ldots, n\}$ as follows. 
\begin{enumerate}
\item For all $i$ such that $x_i \not\in X_j$, $\pi(i) = i$.
\item $\pi(j_{\ell})$ = $j_{\ell+1}$ for $1 \leq \ell_t \leq k-1$
      and $\pi(j_{k})$ = $j_1$. 
\end{enumerate}
Note that $\pi$ is not the identity permutation.
Thus, by Theorem~\ref{thm:ncf_asymmetry},
there is an input $(a_1, a_2, \ldots, a_n)$ to $f$ such that
such that
\begin{equation}\label{eqn:f_asymmetry}
f(a_1, a_2, \ldots, a_n) ~\neq~  f(a_{\pi(1)}, a_{\pi(2)}, \ldots, a_{\pi(n)})
\end{equation}
However, from the definition of permutation $\pi$, it follows
that the number of 1's in any subset of the partition 
is the same in both the inputs
$(a_1, a_2, \ldots, a_n)$ and
$(a_{\pi(1)}, a_{\pi(2)}, \ldots, a_{\pi(n)})$.
Therefore, since $f$ is $r$-symmetric, we must have
\begin{equation}\label{eqn:f_contra}
f(a_1, a_2, \ldots, a_n) ~=~ f(a_{\pi(1)}, a_{\pi(2)}, \ldots, a_{\pi(n)}).
\end{equation}
Equation~(\ref{eqn:f_contra}) contradicts 
Equation~(\ref{eqn:f_asymmetry}), and the result follows. \QED
%%%%%%%%%%%%%%%%%%%%%%%%%%%%%%%%%%%%%%%%%%%%%
\fi


\iffalse
%%%%%%%%%%%%%%%%%%%%%%%%%%%%%%%%%%%%%%%%%%%%%
$$x_1 \vee \bar{x}_2 ( x_3 \vee \bar{x}_4 (\cdots  ) )$$

For instance,
$$f_6 = x_1 \vee \bar{x}_2 ( x_3 \vee \bar{x}_4 ( x_5 \vee \bar{x}_6  ) ),$$
and
$$f_7 = x_1 \vee \bar{x}_2 ( x_3 \vee \bar{x}_4 ( x_5 \vee \bar{x}_6   x_7) ).$$

Function $f_n$ is a NCF, 
and can be representing as a sequence where the variables appear in lexicographic order,
each $a_i$ is 1, and the $b_i$'s alternate between 1 and 0.
For instance, $f_6$ can be represented as follows:

\noindent
$(x_1: 1 \rightarrow 1 )$ \\
$(x_2: 1 \rightarrow 0 )$ \\
$(x_3: 1 \rightarrow 1 )$ \\
$(x_4: 1 \rightarrow 0 )$ \\
$(x_5: 1 \rightarrow 1 )$ \\
$(x_6: 1 \rightarrow 0 )$ \\

We now argue that $f_n$ is not $n-1$ symmetric.
Consider the classes of $f_n$, represented as an $r$-symmetric function for some $r$.
If we compare the assignment where a given odd-numbered variable $x_{2i+1}$ is the only variable equal to 1,
and the assignment where a given even-numbered variable $x_{2j}$ is the only variable equal to 1,
we can see that these assignments produce different values for $f_n$,
so variables $x_{2i+1}$ and $x_{2j}$ must belong to different classes.
If we compare the assignment where an odd-numbered variable $x_{2i+1}$  
and the even-numbered variable $x_{2i+2}$ 
are the only variables equal to 1,
and the assignment where an odd-numbered variable of the form $x_{2k+1}$, $k >i$,
and $x_{2i+2}$  are the only variables equal to 1,
we can see that these assignments produce different values for $f_n$,
so $x_{2i+1}$ and $x_{2k+1}$ must belong to different classes.
Thus, each odd-numbered variable is the only member of its class.
If we compare the assignment where an even-numbered variable $x_{2i}$  
and the odd-numbered variable $x_{2i+1}$ 
are the only variables equal to 1,
and the assignment where an even-numbered variable of the form $x_{2k}$, $k >i$,
and $x_{2i+1}$  are the only variables equal to 1,
we can see that these assignments produce different values for $f_n$,
so $x_{2i}$ and $x_{2k}$ must belong to different classes.
Thus, each even-numbered variable is the only member of its class.
\QED

%%The following result shows another form of asymmetry for the function
%%defined in the above theorem.  (This form of asymmetry may be stronger
%%than the one established in the above theorem.)
%%%%%%%%%%%%%%%%%%%%%%%%%%%%%%%%%%%%%%%%%%%%%
\fi



   %% Main results. 

\section{Summary}
\label{sec:summary}

We presented a characterization of when two variables
of an NCF are symmetric and used that characterization
to show that the symmetry level of an NCF can be easily computed.
We also showed that an $n$-variable NCF is strongly asymmetric
iff its symmetry level is $n$.
We presented several corollaries of this result, including
a closed form expression for the number of NCFs which are also
strongly asymmetric.
Thus, our results bring out several interesting relationships 
between NCFs and symmetric Boolean functions.

  %% Brief summary to end the paper.


\iffalse
%%%%%%%%%%%%%%%%%%%%%%%%%%%%%%%%%%%%%
% use section* for acknowledgment
\ifCLASSOPTIONcompsoc
  % The Computer Society usually uses the plural form
  \section*{Acknowledgments}
\else
  % regular IEEE prefers the singular form
  \section*{Acknowledgment}
\fi
%%%%%%%%%%%%%%%%%%%%%%%%%%%%%%%%%%%%%
\fi

\section*{Acknowledgments}
This work has been partially supported by
DTRA CNIMS (Contract HDTRA1-11-D-0016-0001),
NSF DIBBS Grant ACI-1443054 and
NSF BIG DATA Grant IIS-1633028.
The U.S. Government is authorized to reproduce and
distribute reprints for Governmental purposes notwithstanding
any copyright annotation thereon.

%%\bigskip\bigskip

\medskip

\noindent
\textbf{Disclaimer:}~ The views and conclusions contained
herein are those of the authors and should
not be interpreted as necessarily representing the
official policies or endorsements, either expressed
or implied, of the U.S. Government.


% Can use something like this to put references on a page
% by themselves when using endfloat and the captionsoff option.

%% \ifCLASSOPTIONcaptionsoff
%%   \newpage
%% \fi


% trigger a \newpage just before the given reference
% number - used to balance the columns on the last page
% adjust value as needed - may need to be readjusted if
% the document is modified later
%\IEEEtriggeratref{8}
% The "triggered" command can be changed if desired:
%\IEEEtriggercmd{\enlargethispage{-5in}}

% references section
\bibliographystyle{IEEEtran}
\bibliography{refs}

% can use a bibliography generated by BibTeX as a .bbl file
% BibTeX documentation can be easily obtained at:
% http://mirror.ctan.org/biblio/bibtex/contrib/doc/
% The IEEEtran BibTeX style support page is at:
% http://www.michaelshell.org/tex/ieeetran/bibtex/
%\bibliographystyle{IEEEtran}
% argument is your BibTeX string definitions and bibliography database(s)
%\bibliography{IEEEabrv,../bib/paper}
%
% <OR> manually copy in the resultant .bbl file
% set second argument of \begin to the number of references
% (used to reserve space for the reference number labels box)

% biography section
% 
% If you have an EPS/PDF photo (graphicx package needed) extra braces are
% needed around the contents of the optional argument to biography to prevent
% the LaTeX parser from getting confused when it sees the complicated
% \includegraphics command within an optional argument. (You could create
% your own custom macro containing the \includegraphics command to make things
% simpler here.)
%\begin{IEEEbiography}[{\includegraphics[width=1in,height=1.25in,clip,keepaspectratio]{mshell}}]{Michael Shell}
% or if you just want to reserve a space for a photo:

\iffalse
%%%%%%%%%%%%%%%%%%%%%%%%%%%%%%%%%%%%%%%%%%%%%%%%%%%
\begin{IEEEbiography}{Michael Shell}
Biography text here.
\end{IEEEbiography}

% if you will not have a photo at all:
\begin{IEEEbiographynophoto}{John Doe}
Biography text here.
\end{IEEEbiographynophoto}

% insert where needed to balance the two columns on the last page with
% biographies
%\newpage

\begin{IEEEbiographynophoto}{Jane Doe}
Biography text here.
\end{IEEEbiographynophoto}
%%%%%%%%%%%%%%%%%%%%%%%%%%%%%%%%%%%%%%%%%%%%%%%%%%%
\fi

% You can push biographies down or up by placing
% a \vfill before or after them. The appropriate
% use of \vfill depends on what kind of text is
% on the last page and whether or not the columns
% are being equalized.

%\vfill

% Can be used to pull up biographies so that the bottom of the last one
% is flush with the other column.
%\enlargethispage{-5in}

% that's all folks
\end{document}
