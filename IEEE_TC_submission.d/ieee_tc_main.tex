%\documentclass[10pt,journal,compsoc]{IEEEtran}
\documentclass[10pt,journal,compsoc]{IEEEtran}
%
% If IEEEtran.cls has not been installed into the LaTeX system files,
% manually specify the path to it like:
% \documentclass[10pt,journal,compsoc]{../sty/IEEEtran}

% Some very useful LaTeX packages include:
% (uncomment the ones you want to load)


% *** CITATION PACKAGES ***
%
\ifCLASSOPTIONcompsoc
  % IEEE Computer Society needs nocompress option
  % requires cite.sty v4.0 or later (November 2003)
  \usepackage[nocompress]{cite}
\else
  % normal IEEE
  \usepackage{cite}
\fi

% *** GRAPHICS RELATED PACKAGES ***
%
\ifCLASSINFOpdf
  % \usepackage[pdftex]{graphicx}
  % declare the path(s) where your graphic files are
  % \graphicspath{{../pdf/}{../jpeg/}}
  % and their extensions so you won't have to specify these with
  % every instance of \includegraphics
  % \DeclareGraphicsExtensions{.pdf,.jpeg,.png}
\else
  % or other class option (dvipsone, dvipdf, if not using dvips). graphicx
  % will default to the driver specified in the system graphics.cfg if no
  % driver is specified.
  % \usepackage[dvips]{graphicx}
  % declare the path(s) where your graphic files are
  % \graphicspath{{../eps/}}
  % and their extensions so you won't have to specify these with
  % every instance of \includegraphics
  % \DeclareGraphicsExtensions{.eps}
\fi


% *** MATH PACKAGES ***
%
\usepackage{amsmath}
\usepackage{amsfonts}
\usepackage{latexsym}
\usepackage{amsmath}
%\usepackage[latin1]{inputenc}
%\usepackage{varioref}
%\usepackage{cite}
% \usepackage{times}


% *** PDF, URL AND HYPERLINK PACKAGES ***
%
\usepackage{url}
% url.sty was written by Donald Arseneau. It provides better support for
% handling and breaking URLs. url.sty is already installed on most LaTeX
% systems. The latest version and documentation can be obtained at:
% http://www.ctan.org/pkg/url
% Basically, \url{my_url_here}.


% correct bad hyphenation here
\hyphenation{op-tical net-works semi-conduc-tor}

%% Some theorem-like environments needed.

\newtheorem{theorem}{Theorem}[section]
\newtheorem{lemma}[theorem]{Lemma}
\newtheorem{corollary}[theorem]{Corollary}
\newtheorem{fact}[theorem]{Fact}
\newtheorem{claim}[theorem]{Claim}
\newtheorem{observation}[theorem]{Observation}
\newtheorem{definition}[theorem]{Definition}
\newtheorem{proposition}[theorem]{Proposition}
\newtheorem{example}[theorem]{Example}

%%% The following command controls the indent for description
%%% environmet.

%\renewcommand{\IEEEiedlistdecl}{\IEEEsetlabelindent{xx}}
\setlength{\IEEElabelindent}{-5pt}
\setlength{\labelsep}{-2pt}
%\setlength{\IEEElabelindent}{0pt}

\begin{document}
%
% paper title
% Titles are generally capitalized except for words such as a, an, and, as,
% at, but, by, for, in, nor, of, on, or, the, to and up, which are usually
% not capitalized unless they are the first or last word of the title.
% Linebreaks \\ can be used within to get better formatting as desired.
% Do not put math or special symbols in the title.
%\begin{titlepage}
\title{Symmetry Properties of Nested Canalyzing Functions}
%\end{titlepage}


%
%
% author names and IEEE memberships
% note positions of commas and nonbreaking spaces ( ~ ) LaTeX will not break
% a structure at a ~ so this keeps an author's name from being broken across
% two lines.
% use \thanks{} to gain access to the first footnote area
% a separate \thanks must be used for each paragraph as LaTeX2e's \thanks
% was not built to handle multiple paragraphs
%
%
%\IEEEcompsocitemizethanks is a special \thanks that produces the bulleted
% lists the Computer Society journals use for "first footnote" author
% affiliations. Use \IEEEcompsocthanksitem which works much like \item
% for each affiliation group. When not in compsoc mode,
% \IEEEcompsocitemizethanks becomes like \thanks and
% \IEEEcompsocthanksitem becomes a line break with idention. This
% facilitates dual compilation, although admittedly the differences in the
% desired content of \author between the different types of papers makes a
% one-size-fits-all approach a daunting prospect. For instance, compsoc 
% journal papers have the author affiliations above the "Manuscript
% received ..."  text while in non-compsoc journals this is reversed. Sigh.

\iffalse
%%%%%%%%%%%%%%%%%%%%%%%%%%%%%%%%%%%%%%%%%%%%%%%%%%%%%%%%%%%%%
\author{Daniel J. Rosenkrantz,~\IEEEmembership{}
        Madhav V. Marathe,~\IEEEmembership{Fellow,~IEEE,}
        S. S. Ravi,~\IEEEmembership{}
        and~Richard E. Stearns,~\IEEEmembership{}% <-this % stops a space
\IEEEcompsocitemizethanks{
\IEEEcompsocthanksitem D. J. Rosenkrantz is with the Department
of Computer Science, University at Albany -- SUNY, Albany, NY, 12222.\protect\\
% note need leading \protect in front of \\ to get a newline within \thanks as
% \\ is fragile and will error, could use \hfil\break instead.
E-mail: drosenkrantz@gmail.com
\IEEEcompsocthanksitem M. V. Marathe is with the Biocomplexity Institute \& Initiative
and the Department of Computer Science at the  University of Virginia, Charlottesville,
VA, 22904.\protect\\ E-mail: marathe@virginia.edu
\IEEEcompsocthanksitem S. S. Ravi is with the Biocomplexity Institute \& Initiative
at the  University of Virginia, Charlottesville,
VA, 22904 and the Department of Computer Science, University at Albany -- SUNY,
Albany, NY, 12222.\protect\\ E-mail: ssravi0@gmail.com
\IEEEcompsocthanksitem R. E. Stearns is with the Department
of Computer Science, University at Albany -- SUNY, Albany, NY, 12222.\protect\\
% note need leading \protect in front of \\ to get a newline within \thanks as
% \\ is fragile and will error, could use \hfil\break instead.
E-mail: thestearns2@gmail.com
}
}% <-this % stops an unwanted space
%%%%%%%%%%%%%%%%%%%%%%%%%%%%%%%%%%%%%%%%%%%%%%%%%%%%%%%%%%%%%
\fi
\author{Daniel J. Rosenkrantz,~ Madhav V. Marathe,~ S.~S. Ravi~ and~
        Richard E. Stearns% 
   \thanks{D. J. Rosenkrantz and R. E. Stearns are with the Computer
           Science Department, University at Albany -- SUNY, Albany,
           NY 12222. Email: \{drosenkrantz,thestearns2\}@gmail.com}
   \thanks{M. V. Marathe is with the Biocomplexity Institute and Initiative
           and the Department of Computer Science, University of Virginia,
           Charlottesville, VA, 22904. Email: marathe@virgnia.edu.}
   \thanks{S. S. Ravi is with the Biocomplexity Institute and Initiative
           at the University of Virginia,
           Charlottesville, VA, 22904 and the Computer
           Science Department, University at Albany -- SUNY, Albany, NY 12222.
           Email: ssravi0@gmail.com}
}
\maketitle

\noindent
\thanks{Manuscript received xx, xxxx; revised xx, xxxx.}

% note the % following the last \IEEEmembership and also \thanks - 
% these prevent an unwanted space from occurring between the last author name
% and the end of the author line. i.e., if you had this:
% 
% \author{....lastname \thanks{...} \thanks{...} }
%                     ^------------^------------^----Do not want these spaces!
%
% a space would be appended to the last name and could cause every name on that
% line to be shifted left slightly. This is one of those "LaTeX things". For
% instance, "\textbf{A} \textbf{B}" will typeset as "A B" not "AB". To get
% "AB" then you have to do: "\textbf{A}\textbf{B}"
% \thanks is no different in this regard, so shield the last } of each \thanks
% that ends a line with a % and do not let a space in before the next \thanks.
% Spaces after \IEEEmembership other than the last one are OK (and needed) as
% you are supposed to have spaces between the names. For what it is worth,
% this is a minor point as most people would not even notice if the said evil
% space somehow managed to creep in.



% The paper headers
\markboth{IEEE Transactions on Computers,~Vol.~xx, No.~x, February~2019}%
{Rosenkrantz \MakeLowercase{\textit{et al.}}: Symmetry Properties of Nested 
Canalyzing Functions} 

% The only time the second header will appear is for the odd numbered pages
% after the title page when using the twoside option.
% 
% *** Note that you probably will NOT want to include the author's ***
% *** name in the headers of peer review papers.                   ***
% You can use \ifCLASSOPTIONpeerreview for conditional compilation here if
% you desire.


% The publisher's ID mark at the bottom of the page is less important with
% Computer Society journal papers as those publications place the marks
% outside of the main text columns and, therefore, unlike regular IEEE
% journals, the available text space is not reduced by their presence.
% If you want to put a publisher's ID mark on the page you can do it like
% this:
%\IEEEpubid{0000--0000/00\$00.00~\copyright~2015 IEEE}
% or like this to get the Computer Society new two part style.
%\IEEEpubid{\makebox[\columnwidth]{\hfill 0000--0000/00/\$00.00~\copyright~2015 IEEE}%
%\hspace{\columnsep}\makebox[\columnwidth]{Published by the IEEE Computer Society\hfill}}
% Remember, if you use this you must call \IEEEpubidadjcol in the second
% column for its text to clear the IEEEpubid mark (Computer Society jorunal
% papers don't need this extra clearance.)

% use for special paper notices
%\IEEEspecialpapernotice{(Invited Paper)}



% for Computer Society papers, we must declare the abstract and index terms
% PRIOR to the title within the \IEEEtitleabstractindextext IEEEtran
% command as these need to go into the title area created by \maketitle.
% As a general rule, do not put math, special symbols or citations
% in the abstract or keywords.
%% \IEEEtitleabstractindextext{%


\newcommand{\QED}{\hfill\rule{2mm}{2mm}}
\newcommand{\qed}{\hfill$\Box$}

\newcommand{\cpoly}{\textbf{P}}
\newcommand{\cnp}{\textbf{NP}}
\newcommand{\cnump}{\textbf{\#P}}
\newcommand{\wtg}{\mbox{$\mathcal{G}$}}

\newcommand{\arr}{\mbox{$\:\longrightarrow\:$}}

\smallskip

\begin{abstract}
%%The abstract goes here.
A class of Boolean functions, 
called \textbf{nested canalyzing functions} (NCFs),
has been used to model certain biological phenomena.
Researchers have studied some properties of these functions
such as sensitivity and stability.
In this note, we examine the relationships between NCFs, symmetric
Boolean functions and a more general form of symmetric Boolean functions,
called $r$-symmetric functions (where $r$ is the symmetry level).
Using a standard representation for NCFs, we develop a
characterization of when two variables of an NCF are symmetric.
Using this characterization, we show
that the symmetry level of an NCF $f$
can be easily computed given a standard representation of $f$.
(In contrast, we prove that even approximating the symmetry level of
a general Boolean function to within any factor $\geq 1$ is \cnp-hard.)
We also show that for any NCF $f$ with $n$ variables, the notion of
strong asymmetry considered in the literature is equivalent to
the property that $f$ is $n$-symmetric. 
We use this result to derive a closed form expression for the
number of $n$-variable Boolean functions
that are NCFs and strongly asymmetric.
In addition, we identify all the Boolean functions that are NCFs
and symmetric. 
\end{abstract}

% Note that keywords are not normally used for peerreview papers.
\begin{IEEEkeywords}
Boolean functions, Nested canalyzing functions, Symmetry.
\end{IEEEkeywords}
%%}

% make the title area
%% \maketitle  <--- Using peerreview mode.



% To allow for easy dual compilation without having to reenter the
% abstract/keywords data, the \IEEEtitleabstractindextext text will
% not be used in maketitle, but will appear (i.e., to be "transported")
% here as \IEEEdisplaynontitleabstractindextext when the compsoc 
% or transmag modes are not selected <OR> if conference mode is selected 
% - because all conference papers position the abstract like regular
% papers do.

%% \IEEEdisplaynontitleabstractindextext

% \IEEEdisplaynontitleabstractindextext has no effect when using
% compsoc or transmag under a non-conference mode.



% For peer review papers, you can put extra information on the cover
% page as needed:
% \ifCLASSOPTIONpeerreview
% \begin{center} \bfseries EDICS Category: 3-BBND \end{center}
% \fi
%
% For peerreview papers, this IEEEtran command inserts a page break and
% creates the second title. It will be ignored for other modes.

%% \IEEEpeerreviewmaketitle

%%% Newcommands needed for the paper.




%%\IEEEraisesectionheading{\section{Introduction}\label{sec:intro}}

% Computer Society journal (but not conference!) papers do something unusual
% with the very first section heading (almost always called "Introduction").
% They place it ABOVE the main text! IEEEtran.cls does not automatically do
% this for you, but you can achieve this effect with the provided
% \IEEEraisesectionheading{} command. Note the need to keep any \label that
% is to refer to the section immediately after \section in the above as
% \IEEEraisesectionheading puts \section within a raised box.

% The very first letter is a 2 line initial drop letter followed
% by the rest of the first word in caps (small caps for compsoc).
% 
% form to use if the first word consists of a single letter:
% \IEEEPARstart{A}{demo} file is ....
% 
% form to use if you need the single drop letter followed by
% normal text (unknown if ever used by the IEEE):
% \IEEEPARstart{A}{}demo file is ....
% 
% Some journals put the first two words in caps:
% \IEEEPARstart{T}{his demo} file is ....
% 
% Here we have the typical use of a "T" for an initial drop letter
% and "HIS" in caps to complete the first word.

%% \IEEEPARstart{T}{his} demo file is intended to serve as a ``starter file''
%% for IEEE Computer Society journal papers produced under \LaTeX\ using
%% IEEEtran.cls version 1.8b and later.

% You must have at least 2 lines in the paragraph with the drop letter
% (should never be an issue)

\section{Introduction} 
\label{sec:intro}

\subsection{Canalyzing and Nested Canalyzing Functions}
\label{sse:ncf_def}

\IEEEPARstart{T}{he} class of \textbf{canalyzing} Boolean functions, introduced 
in \cite{Kauffman-1969}, is defined as follows.

\begin{definition}\label{def:canalyzing}
Given a set $X = $ $\{x_1, x_2, \ldots, x_n\}$ of $n$  Boolean variables,
a Boolean function $f(x_1, x_2, \ldots, x_n)$ over $X$ is \textbf{canalyzing}
if there is a variable $x_i \in X$ and values $a$, $b$ $\in \{0,1\}$ such that
\[
f(x_1, \ldots, x_{i-1}, a, x_{i+1}, \ldots, x_n) ~=~ b, 
\]
regardless of the values assigned to the variables in $X - \{x_i\}$.
\end{definition}

\medskip

\noindent
\textbf{Example 1:}~ Consider the function 
$f(x_1, x_2, x_3) ~=~ \overline{x_1} \wedge (x_2 \vee \overline{x_3})$.
This function is canalyzing since if we set to $x_1 = 1$,
$f(1, x_2, x_3) ~=~ 0$, for all combinations of values for 
$x_2$ and $x_3$. %\qed

\medskip

Another class of Boolean functions, called \textbf{nested canalyzing functions} (NCFs),
was introduced later in \cite{Kauffman-etal-2003} to carry out a detailed
analysis of the behavior of certain biological systems.
We follow the presentation in \cite{Layne-2011} in defining NCFs.
(For a Boolean value $b$,~ the complement is denoted by $\overline{b}$.)

\begin{definition}\label{def:nested_canalyzing}
Let $X = $ $\{x_1, x_2, \ldots, x_n\}$ denote a set of $n$  Boolean variables.
Let $\pi$ be a permutation of $\{1, 2, \ldots, n\}$.
A Boolean function $f(x_1, x_2, \ldots, x_n)$ over $X$ is \textbf{nested canalyzing}
in the variable order $x_{\pi(1)}, x_{\pi(2)}, \ldots, x_{\pi(n)}$ with
\textbf{canalyzing values} $a_1, a_2, \ldots, a_n$ and 
\textbf{canalyzed values} $b_1, b_2, \ldots, b_n$ 
if $f$ can be expressed in the following form:
\[
f(x_1, x_2, \ldots, x_n) ~=~ 
   \begin{cases}
       \:b_1 & \mathrm{if~~} x_{\pi(1)} ~=~ a_1 \\
       \:b_2 & \mathrm{if~~} x_{\pi(1)} ~\neq~ a_1 \mathrm{~~and~~}
            x_{\pi(2)} ~=~ a_2 \\
       \:\vdots & \vdots \\
       \:b_n & \mathrm{if~~} x_{\pi(1)} ~\neq~ a_1 \mathrm{~~and~~} \ldots~
             x_{\pi(n-1)} ~\neq~ a_{n-1} \mathrm{~~and~~} x_{\pi(n)} ~=~ a_n \\
       \:\overline{b_n} & \mathrm{if~~} x_{\pi(1)} ~\neq~ a_1 \mathrm{~~and~~} \ldots~
            x_{\pi(n)} ~\neq~ a_n \\
   \end{cases}
\]
\end{definition}
%% \clearpage
For convenience, we will use a notation introduced in \cite{Stearns-etal-2018}
to represent NCFs.
For $1 \leq i \leq n$, line $i$ of this representation has the form

\medskip

\noindent
\hspace*{1.1in} $x_{\pi(i)}:~ a_i ~~\longrightarrow~~ b_i$

\medskip

\noindent 
with $x_{\pi(i)}$ being the \textbf{canalyzing variable} that is
\textbf{tested} in line $i$, 
and $a_i$ and $b_i$ being respectively the canalyzing and 
canalyzed values in line $i$,~ $1 \leq i \leq n$.
Each such line is called a \textbf{rule}.
When none of the conditions ``$x_{\pi(i)} ~=~ a_i$" 
is satisfied, we have line $n+1$ with the ``Default" rule
for which the canalyzed value is~ $\overline{b_n}$: 

\medskip

\noindent
\hspace*{1.1in} Default:~ $\overline{b_n}$

\medskip
\noindent
As in \cite{Stearns-etal-2018}, we will refer to the above specification
of an NCF as the \textbf{simplified representation} and assume
(without loss of generality) that each NCF is specified in this manner.
%% The simplified representation provides the following 
%% convenient computational view of an NCF.~
%% Lines defining an NCF are 
%% considered sequentially in a top-down manner.
%% The computation stops at the first line where the 
%% specified condition is satisfied, and the value of the function
%% is the canalyzed value on that line. 
We now present an example of an NCF using the two representations
mentioned above.


\medskip
\noindent
\textbf{Example 2:}~ Consider the function 
$f(x_1, x_2, x_3) ~=~ \overline{x_1} \wedge (x_2 \vee \overline{x_3})$
used in Example~1.
This function is nested canalyzing using the identity permutation $\pi$ on $\{1,2,3\}$
with canalyzing values $1,1,0$ and canalyzed values $0, 1, 1$.
We first show how this function can be expressed using the syntax of
Definition~\ref{def:nested_canalyzing}.

\[
f(x_1, x_2, x_3) ~=~ 
   \begin{cases}
       \:0 & \mathrm{if~~} x_{1} ~=~ 1 \\
       \:1 & \mathrm{if~~} x_{1} ~\neq~ 1 \mathrm{~~and~~}
            x_{2} ~=~ 1 \\
       \:1 & \mathrm{if~~} x_{1} ~\neq~ 1 \mathrm{~~and~~}
            x_{2} ~\neq~ 1 \mathrm{~~and~~} x_{3} = 0 \\
       \:0 & \mathrm{if~~} x_{1} ~\neq~ 1 \mathrm{~~and~~}
            x_{2} ~\neq~ 1 \mathrm{~~and~~} x_{3} ~\neq~ 0 \\
   \end{cases}
\]

\medskip
\noindent
A simplified representation of the same function is as follows.

\bigskip

\noindent
\begin{tabular}{ll}
\hspace*{1.1in} & $x_1:~$  $1 ~\longrightarrow~ 0$ \\ [1ex]
\hspace*{1.1in} & $x_2:~$  $1 ~\longrightarrow~ 1$ \\ [1ex]
\hspace*{1.1in} & $x_3:~$  $0 ~\longrightarrow~ 1$ \\ [1ex]
\hspace*{1.1in} & Default:~ $0$ \\
\end{tabular}

\noindent
%\qed

\iffalse
%%%%%%%%%%%%%%%%%%%%%%%%%%%%%%%%%%%%%%%%%%%%%%%
\noindent
We will also consider a more general form of NCFs defined below.

\begin{definition}\label{def:generalized ncf}
A {\bf generalized NCF} is a function represented as either a constant
or an NCF representation of a subset (not necessarily proper) 
of the function's variables.
\end{definition}

\noindent
\textbf{Example 3:}~ For $b \in \{0,1\}$, the constant function which takes 
on the value $b$ for every input is a generalized NCF since 
it can be represented by the following rule: 

\smallskip

\hspace*{1.1in} Default:~ $b$ 

\smallskip

\noindent
The following generalized NCF specification for 
a function $f(x_1, x_2, x_3, x_4)$ indicates that the function
depends only on variables $x_1$ and $x_3$.

\medskip

\noindent
\begin{tabular}{ll}
\hspace*{1.1in} & $x_1:~$  $0 ~\longrightarrow~ 1$ \\ [1ex]
\hspace*{1.1in} & $x_3:~$  $1 ~\longrightarrow~ 0$ \\ [1ex]
\hspace*{1.1in} & Default:~ $1$ \\
\end{tabular}

\noindent
%\qed

%% \bigskip

%% \noindent
%% \textbf{Note:} If a generalized NCF is not a constant, 
%% there is an assignment to its variables
%% such that the value of the function is 0,
%% and there is an assignment to its variables
%% such that the value of the function is 1.
%%%%%%%%%%%%%%%%%%%%%%%%%%%%%%%%%%%%%%%%%%%%%%%
\fi

\subsection{Symmetric Boolean Functions}
\label{sse:symmetry}

A pair of variables of a Boolean function $f(x_1, x_2, \ldots, x_n)$ is
said to be \textbf{symmetric} if their values can be interchanged without
affecting the value of the function.
As a simple example, each pair of variables in the function 
$f_2(x_1, x_2, x_3)$ = $x_1 \oplus x_2 \oplus x_3$ is symmetric.
This form of interchange symmetry partitions the set of variables into a set of
\textbf{symmetry groups}, where the members of each group are pairwise symmetric.
A Boolean function $f$ is 
$r$-\textbf{symmetric} if it has at most $r$ symmetry groups.
In this case, the value of $f$
depends only on how many of the variables in each symmetry group have the value 1.
We say that $f$ is \textbf{properly} $r$-\textbf{symmetric} if
it is $r$-symmetric, but not $(r-1)$-symmetric.
For example, the function $f_3(x_1, x_2, x_3) = (x_1 \wedge x_2) \vee\, \overline{x_3}$~
is not 1-symmetric since $f_3(1, 0, 1) \neq f_3(1, 1, 0)$; however, 
it is 2-symmetric with the symmetric groups being $\{x_1, x_2\}$ and $\{x_3\}$.
For a Boolean function $f$, the integer $r$ such that $f$ is properly $r$-symmetric
will be referred to as the \textbf{symmetry level} of $f$.
Thus, the symmetry level of $f$ is the smallest integer $r$ such that
$f$ is $r$-symmetric.

In the literature (see e.g., \cite{Crama-Hammer-2011,HT-2016,Toth-etal-1977}),
a 1-symmetric function $f$ is referred to simply
as a \textbf{symmetric} function.
Thus, in any symmetric function, each pair of variables is symmetric.
As a simple example, the function $f_2(x_1, x_2, x_3)$ = 
$x_1 \oplus x_2 \oplus x_3$ (defined above) is symmetric.
If $f$ is a symmetric function, then for any
input $(a_1, a_2, \ldots, a_n)$ to $f$, where $a_i \in \{0,1\}$ for
$1 \leq i \leq n$, and any permutation $\pi$ of $\{1, 2, \ldots, n\}$,
$f(a_1, a_2, \ldots, a_n)$ = $f(a_{\pi(1)}, a_{\pi(2)}, \ldots, a_{\pi(n)})$.
The class of $r$-symmetric functions have also been
studied in the literature on discrete dynamical systems (see e.g., 
\cite{Barrett-etal-2007,Rosenkrantz-etal-2015,MR-2007}).

Other forms of symmetry which can be more general than the permutations 
corresponding to symmetry groups have been considered \cite{KS-2000}.
A Boolean function $f(x_1, x_2, \ldots, x_n)$
is \textbf{strongly asymmetric}
if for any permutation $\pi$ of $\{1, 2, \ldots, n\}$
\emph{except} the identity permutation,
there exists an input $(a_1, a_2, \ldots,  a_n)$
to $f$ such that $f(a_1, a_2, \ldots, a_n)$ $\neq$
$f(a_{\pi(1)}, a_{\pi(2)}, \ldots, a_{\pi(n)})$.
In general, there are functions with $n$ variables that are properly
$n$-symmetric, but not strongly asymmetric \cite{KS-2000}.
However, in the case of NCFs with $n$ variables, we will show 
(see Section~\ref{sec:ncf_and_symmetry}) that the notion of 
strong asymmetry coincides with that 
of being properly $n$-symmetric.

%\noindent

\subsection{Summary of Results}
\label{sse:contrib}

Our focus is on the relationships between 
NCFs and symmetric Boolean functions.
Using a standard representation for NCFs, we develop a
characterization of when two variables of an NCF are symmetric.
Using this characterization, we show
that the symmetry level of an NCF $f$
can be easily computed given a standard representation of $f$.
(In contrast, we show that 
one cannot even efficiently approximate the symmetry level of
a general Boolean function to within any factor $\geq 1$, 
unless \textbf{P} = \cnp.)
We also show that for any NCF $f$ with $n$ variables, the notion of
strong asymmetry considered in the literature is equivalent to
the property that $f$ is $n$-symmetric.
We use this result to derive a closed form expression for the
number of $n$-variable Boolean functions
that are NCFs and strongly asymmetric.
In addition, we identify all the Boolean functions that are 
canalyzing and symmetric as well as those that 
are NCFs and symmetric.

\subsection{Related Work}
\label{sse:related}

The term \textbf{canalization}, coined by
Waddington \cite{Waddington-1942}, is generally used to describe
the stability of a biological system with changes
in external conditions.
In 1969, Kauffman \cite{Kauffman-1969} introduced canalyzing Boolean functions
to explain the stability of gene regulatory networks.
The subclass of NCFs was
introduced later by Kauffman et al. \cite{Kauffman-etal-2003} 
to facilitate a rigorous analysis of the Boolean network model
for gene regulatory networks.
It is known that the class of NCFs coincides with that
of unate cascade Boolean functions \cite{Jarrah-etal-2007}.
In the literature on computational learning theory,
NCFs are referred to as $1$-\textbf{decision lists} \cite{KV-1994}.
Many researchers have pointed out the usefulness of NCFs 
in modeling biological phenomena 
(e.g., \cite{Layne-2011,
Layne-etal-2012,Li-etal-2011,Li-etal-2012,Li-etal-2013}).
Properties of NCFs such as sensitivity and 
stability have also been studied in the 
literature \cite{Kauffman-etal-2004, Layne-2011,Layne-etal-2012,
Li-etal-2011,Li-etal-2013,Klotz-etal-2013, Stearns-etal-2018}. 

Over the years, a considerable amount of research on detecting symmetry properties 
of Boolean functions has been reported in the literature
(e.g., \cite{BS-1968,Biswas-1970,TM-1996,KS-2000,Darga-etal-2008,Maurer-2015}).
Several references have observed the usefulness of detecting symmetries
in automated logic synthesis (e.g., \cite{AP-2008,Hu-etal-2008,Darga-etal-2008}).
Reference \cite{Maurer-2015} provides a thorough discussion of known
symmetry detection algorithms and presents a new universal algorithm 
for detecting any form of permutation-based symmetry in Boolean functions. 
To our knowledge, relationships between NCFs and symmetric
Boolean functions have not been addressed in the literature.
   %% Introduction (definitions, summary of results and related work)

\section{Other Definitions and Preliminary Results}
\label{sec:prelim}

\subsection{Overview}

Here, we present some definitions and preliminary results which enable
us to define a normalized representation for NCFs.
This representation plays an important role in several
problems considered in later sections.

\medskip

Recall that any NCF $f$ with $n$ variables,
denoted by $x_1$, $x_2$, $\ldots$, $x_n$,~
is specified using a variable ordering 
$x_{\pi(1)}$, $x_{\pi(2)}$, $\ldots$,  $x_{\pi(n)}$,
where $\pi$ is a permutation of $\{1, 2, \ldots, n\}$.
We will refer to this as the \textbf{lexicographic} ordering
of the variables in the specification.

\medskip

Throughout this section, we assume that an NCF $f$ with $n$
variables is specified using the simplified representation with $n$ lines. 
Unless otherwise mentioned, the ``Default" line (which is assigned the number $n+1$)
in the simplified representation of an NCF is \emph{not} considered
in the results stated in this section.

\subsection{Layers and Normalized Representation}
\label{sse:ncf_layer}

We begin with a two-part observation from \cite{Stearns-etal-2018}.

\begin{observation} \label{obs:ncf_transformations}
Let $f$ be an NCF with $n$ variables specified using $n$ lines
in the simplified representation. 
\begin{description}
\item{(a)} 
For any $q \geq 2$ and for any $i$, $1 \leq i \leq n-q+1$,
if the $q$ consecutive lines $i$, $i+1$, $\ldots$, $i+q-1$~
have the same canalyzed value, then the function remains
unchanged if these $q$ lines are permuted in any order
without changing the other lines.

\item{(b)} 
Suppose lines $n-1$ and $n$ in the specification of $f$ 
have complementary canalyzed outputs.
Then, the function remains unchanged 
if the canalyzing value and canalyzed output in line $n$
are both complemented. 
(Here, the value on the ``Default" line is also complemented.)
\QED
\end{description}
\end{observation}

\noindent
Li et al. \cite{Li-etal-2013} have defined the concept of a 
{\bf layer} of an NCF in terms of
an algebraic representation of the NCF as an extended monomial.
For our purposes, we use the following definition based on the
simplified representation of NCFs given in Section~\ref{sec:intro}.

\begin{definition}
\label{def:layer}
A {\bf layer} of a NCF representation is a maximal length sequence 
of lines with the same canalyzed output.
\end{definition}
%%Note that the ``Default" line in the representation of an NCF is
%%\emph{not} considered in the definition of a layer.

\noindent
\textbf{Example 4:}~ Consider the following function $f$ of six variables
$x_1$, $x_2$, $\ldots$, $x_6$. 

\bigskip

\noindent
\begin{tabular}{ll}
\hspace*{1.1in} & $x_1:~$  $1 ~\longrightarrow~ 1$ \\ [0.5ex]
\hspace*{1.1in} & $x_2:~$  $0 ~\longrightarrow~ 1$ \\ [0.5ex]
\hspace*{1.1in} & $x_3:~$  $0 ~\longrightarrow~ 0$ \\ [0.5ex]
\hspace*{1.1in} & $x_4:~$  $1 ~\longrightarrow~ 0$ \\ [0.5ex]
\hspace*{1.1in} & $x_5:~$  $1 ~\longrightarrow~ 1$ \\ [0.5ex]
\hspace*{1.1in} & $x_6:~$  $1 ~\longrightarrow~ 0$ \\ [0.5ex]
\hspace*{1.1in} & Default:~ $1$ \\
\end{tabular}

\medskip
\noindent
This function has 4 layers, with 
Layer~1 consisting of the lines with $x_1$ and $x_2$, 
Layer~2 consisting of the lines with $x_3$ and $x_4$, 
Layer~3 consisting of the line with $x_5$ and 
Layer~4 consisting of the line with $x_6$. \qed

\medskip
From Part~(a) of Observation~\ref{obs:ncf_transformations},
it follows that the lines within the same layer of an NCF
representation can be permuted without changing the function.
Part~(b) of Observation~\ref{obs:ncf_transformations} points
out that for any NCF $f$ with $n \geq 2$ variables, there is a simplified 
representation in which lines $n-1$ and $n$ are in the same layer.

%% It can be verified that the number of layers in the 
%% extended monomial formula for a NCF (as defined in \cite{Li-etal-2013})
%% is the same as the number of layers in the normalized representation.

\medskip
We can now define the notion of a \textbf{normalized} 
representation for NCFs.

\begin{definition}
\label{def:normalized}
Let $f$ be an NCF with $n \geq 2$ variables specified using
the simplified representation with 
lexicographic ordering given by  
$\langle x_{\pi(1)}, x_{\pi(2)}, x_{\pi(n)}\rangle$.
We say that the representation is {\bf normalized} if
both of the following conditions hold.
\begin{enumerate}
  \item Lines $n-1$ and $n$ have the same canalyzed output ~and
  \item each pair of consecutive lines
        with the same canalyzed output occur in the lexicographic order 
        of their canalyzing variables; that is, if two consecutive lines are \\
%\medskip
\noindent
\hspace*{1.1in} $x_i$:~ $a_i \longrightarrow b$ \\
\hspace*{1.1in} $x_j$:~ $a_j \longrightarrow b$

%\medskip
\noindent
then $x_i$ precedes $x_j$ in the lexicographic ordering.
\end{enumerate}
\end{definition}

\noindent
\textbf{Example 5:}~ 

\medskip
\noindent
\textbf{Part 1:}~
Consider the NCF $f$ shown in Example~4 with
the variable ordering $x_1$, $x_2$, $x_3$, $x_4$, $x_5$, $x_6$.
This representation is \emph{not} normalized since lines 5 and 6
have different canalyzed values.
If we change line 6 to ``$x_6$:~ 0 $\longrightarrow$ 1" and the
value on the ``Default" line to 0, then we obtain 
a normalized representation. 

\medskip
\noindent
\textbf{Part 2:}~ Consider the normalized version of Example~4 
produced in Part~1 above. 
In that representation, if we exchange Lines 1 and 2, 
by Part~(a) of Observation~\ref{obs:ncf_transformations},
the function does not change. 
However, the resulting representation is \emph{not} normalized
since the line with $x_2$ precedes the one 
with $x_1$. \qed

\medskip
From Observation \ref{obs:ncf_transformations},
we note that in polynomial time, for any NCF representation with more than one line,
we can first modify the last line if necessary so that it has the same canalyzed output as the preceding line,
and then we can lexicographically sort
each maximal sequence of consecutive lines with the same canalyzed output,
thereby obtaining an equivalent normalized NCF representation.
We state this formally below.

\begin{observation}\label{obs:normalization_poly}
Given an NCF representation for a function $f$, a normalized
representation for $f$ can be obtained in polynomial time. \QED
\end{observation}

\subsection{Complexity of Estimating the Symmetry Level of a i
Boolean Function}
\label{sse:symmetry_level_hardness}

Recall that the symmetry level of a Boolean function $f$ is the smallest
integer $r$ such that the following condition holds:
the  inputs to $f$ can be partitioned into $r$ subsets 
(symmetry groups) so that the value of $f$ depends only on 
how many of the inputs in each group have the value 1.
The following result shows that given a Boolean function $f$
in the form of a Boolean expression, it is \cnp-hard to 
approximate\footnote{An algorithm approximates the symmetry
level of a Boolean function $f$ within the factor $\rho$ if it finds a
partition of the inputs to $f$ into at most $\rho \timex r^*$ symmetry
groups, where $r^*$ is the symmetry level of $f$.}
the symmetry level of $f$ to within any factor $\rho \geq 1$.

\newcommand{\cala}{\mbox{$\mathcal{A}$}}

\begin{proposition}\label{pro:approx_sym_level_hard}
For any $\rho \geq 1$, it is \cnp-hard to approximate the
symmetry level of $f$ to within the factor $\rho$.
\end{proposition}

\noindent
\textbf{Proof:}~ Suppose there is an efficient algorithm \cala{} 
that provides a $\rho$ approximation for the symmetry level
of a Boolean function for some $\rho \geq 1$.
We will show that \cala{} can be used to efficiently solve
the CNF Satisfiability problem (SAT) which is known 
to be \cnp-hard \cite{GJ-1979}.

Let $f$ be a CNF formula representing an instance of SAT.
Let $X$ = $\{x_1, x_2, \ldots, x_n\}$ denote the set of
Boolean variables used in $f$.


\QED

In contrast, we will show that the symmetry level of an NCF (or that of a
generalized NCF) can be computed efficiently.
   %% Preliminaries and normalized representation for NCFs

\section{Symmetry of Nested Canalyzing Functions}
\label{sec:ncf_and_symmetry}

\subsection{Overview}
\label{sse:res_overview}
Our first result (Theorem~\ref{thm:ncf_symmetric_variables})
provides a characterization of pairs of variables of an NCF
that are part of the same symmetry group. 
This characterization allows us to give a simple closed form
expression (Theorem~\ref{thm:ncf_r_symmetric}) 
for the symmetry level of an NCF specified using
its default-normalized representation.
We also present an efficient algorithm (Theorem~\ref{thm:rsym_canalyzing})
for the converse problem, that is,
testing whether a given $r$-symmetric function $f$ is an NCF;
if so, our algorithm constructs a default-normalized 
representation for $f$.
Next, we show that for any NCF $f$ with $n$ variables,
the two statements ``$f$ is strongly asymmetric" ~and~
``the symmetry level of $f$ is $n$"~ are equivalent 
(Theorem~\ref{thm:ncf_strong_asymmetry}).
We use this result to derive a closed form expression 
for the number of $n$-variable NCFs that are 
also strongly asymmetric (Theorem~\ref{thm:count_strongly_asymmetric}).
Our final result (Proposition~\ref{pro:ncf_symmetric})
identifies all the functions that are symmetric and
canalyzing as well as those that are symmetric and NCFs.

\subsection{Symmetric Pairs of Variables in an NCF:
A Characterization and\newline its Applications}
\label{sse:ncf_strong_sym}

\begin{theorem}\label{thm:ncf_symmetric_variables}
Two variables of an NCF $f$ are symmetric iff
in the default-normalized representation of $f$,
they occur in the same layer and have the same canalyzing value.
\end{theorem}
\noindent
\textbf{Proof:}~
Suppose that two variables of an NCF $f$ occur in the same layer of a
default-normalized representation, and the lines containing these variables
have the same canalyzing value.  Then in any input assignment, the
values of these two variables can be interchanged, and the evaluation
of the lines in that layer will have the same effect.  Thus, the
variables are symmetric.

For the converse, consider the default-normalized NCF representation of $f$. 
As mentioned earlier, we assume without loss of generality that the
canalyzing variable in line $i$ of $f$ is $x_i$, $1 \leq i \leq n$.
Let  $x_j$ and $x_k$ be two variables that occur in different
layers of $f$, or occur in the same layer, but with different
canalyzing values.  Without loss of generality, 
assume that $j < k$, so that the line for
$x_j$ occurs above the line for $x_k$.
Suppose that line $i$ of $f$, $1 \leq i \leq n$, is 

\smallskip

\hspace*{0.25in}
$x_i : a_i ~\longrightarrow~ b_i$. 

\smallskip

\noindent
Consider the following assignment $\alpha$ = $(c_1, c_2, \ldots,  c_n)$ 
to the variables $x_1$, $x_2$, $\ldots$, $x_n$ of $f$.
\begin{enumerate}
\item For $1 \leq i < j$, set $c_i = \overline{a_i}$. 
\item For $i = j$, set $c_j =a_j$. 
\item For $i = k$, set $c_k =\overline{a_j}$. 
\item For $j < i < k$, and for  $k < i \leq n$, 
if $b_i = b_j$,\\ then set $c_i =  \overline{a_i}$,
else set $c_i =  a_i$.
\end{enumerate}
Note that $f(\alpha) = b_j$, since the variables in all lines above
line $j$ have the complement of their canalyzing value, and variable
$x_j$ has its canalyzing value.

\medskip

Now, let $\alpha'$ be the assignment that is obtained from $\alpha$
by interchanging the values of variables $x_j$ and $x_k$, so that
after the interchange, variable $x_j$ has value $\overline{a_j}$ and
variable $x_k$ has value $a_j$.  We will show that $f(\alpha') =
\overline{b_j}$, so variables $x_j$ and $x_k$ are not symmetric.

In line $k$ of $f$, canalyzing value $a_k$ is either the same or
the complement of canalyzing value $a_j$, and canalyzed value $b_k$
is either the same or the complement of canalyzed value $b_j$,
Thus, there are four possible cases for the form of line $k$.  We
now consider each of the four cases.

\medskip

\noindent
{\bf Case 1:} Line $k$ has the form $x_k : a_j ~\longrightarrow~ b_j$.  

\smallskip

Since line $k$ has
the same canalyzing value and canalyzed value as line $j$, line
$k$ must occur in a lower layer than that of line $j$.  Thus, there
is at least one line between line $j$ and line $k$ for which the
canalyzed value is $\overline{b_j}$.  Let line $q$ be the first
such line.  Note that for all $i$ such that $1 \leq i < q$, the
value of variable $x_i$ in assignment $\alpha'$ does not match the
canalyzing value of line $i$, but the value of variable $x_q$ does
match the canalyzing value of line $q$.  Since canalyzed value
$b_q$ equals $\overline{b_j}$, we have that $f(\alpha') =
\overline{b_j}$.

\medskip

\noindent
{\bf Case 2:} Line $k$ has the form  $x_k : a_j ~\longrightarrow~ \overline{b_j}$.  

\smallskip

Let line $q$ be the first line such that $j < q \leq k$ and the canalyzed
value of line $q$ is $\overline{b_j}$.  Note that $q$ might possibly
equal $k$.  For all $i$ such that $1 \leq i < q$, the value of
variable $x_i$ in assignment $\alpha'$ does not match the canalyzing
value of line $i$, but the value of variable $x_q$ does match the
canalyzing value of line $q$.  Since canalyzed value $b_q$ equals
$\overline{b_j}$, we have that $f(\alpha') = \overline{b_j}$.

\medskip

\noindent
{\bf Case 3:} Line $k$ has the form $x_k : \overline{a_j} ~\longrightarrow~ b_j$.  

\smallskip

Suppose there is a line below line $j$ for which the canalyzed value is
$\overline{b_j}$.  Let $q$ be the first such line.  Note that for
all $i$ such that $1 \leq i < q$, the value of variable $x_i$ in
assignment $\alpha'$ does not match the canalyzing value of line
$i$, but the value of variable $x_q$ does match the canalyzing value
of line $q$.  Since canalyzed value $b_q$ equals $\overline{b_j}$,
we have that $f(\alpha') = \overline{b_j}$.

\smallskip

Now suppose there is no line below line $j$ for which the canalyzed
value is $\overline{b_j}$.  Then lines $j$ and $k$ both occur in
the last layer of $f$.  Thus, for every line $i$, $1 \leq i \leq
n$, the value of variable $x_i$ in assignment $\alpha'$ does not
match the canalyzing value of line $i$.  Consequently, $f(\alpha')
= \overline{b_j}$, the complement of the canalyzed value in the
last layer of $f$.

\medskip

\noindent
{\bf Case 4:} Line $k$ has the form  $x_k : \overline{a_j} ~\longrightarrow~ \overline{b_j}$. 

\smallskip

Suppose there is a line, other than line $k$, below line $j$ for
which the canalyzed value is $\overline{b_j}$.  Let $q$ be the
first such line.  Note that for all $i$ such that $1 \leq i < q$,
the value of variable $x_i$ in assignment $\alpha'$ does not match
the canalyzing value of line $i$, but the value of variable $x_q$
does match the canalyzing value of line $q$.  Since canalyzed value
$b_q$ equals $\overline{b_j}$, we have that $f(\alpha') =
\overline{b_j}$.

Now suppose that line $k$ is the only line below line $j$ for which
the canalyzed value is $\overline{b_j}$.  Since line $j$ has
canalyzed value $b_j$, line $k$ is the only line in its layer.
Since $f$ is default-normalized, the last layer of $f$ contains at least
two lines.  Thus, there is a last layer following the layer containing
line $k$.  The canalyzed value of all the lines in this last layer
is $b_j$.  Thus, for every line $i$, $1 \leq i \leq n$, the value
of variable $x_i$ in assignment $\alpha'$ does not match the
canalyzing value of line $i$.  Consequently, $f(\alpha') =
\overline{b_j}$, the complement of the canalyzed value in the last
layer of $f$.  \QED

\begin{theorem}\label{thm:ncf_r_symmetric}
Suppose the default-normalized representation of a given NCF contains 
$r_1$ layers with only one distinct canalyzing value,
and $r_2$ layers with two distinct canalyzing values.
Then the function is properly $(r_1 + 2 r_2)$-symmetric.
\end{theorem}

\noindent
\textbf{Proof:}~
From Theorem~\ref{thm:ncf_symmetric_variables}, 
any pair of variables occurring in the same layer, 
with the same canalyzing value, are symmetric.
Thus the function is at most $(r_1 + 2 r_2)$-symmetric.
Moreover, any pair of variables from different layers, 
or with different canalyzing values
are not symmetric, so there are at least $(r_1 + 2 r_2)$ symmetry groups.
\QED

\begin{corollary}\label{cor:ncf_not_rsymm}
For every $n \geq 2$, there is an $n$-variable NCF that is not $n-1$ symmetric. \QED
\end{corollary}

\begin{corollary}\label{cor:ncf_r_symmetric_layers}
(i) An NCF with a default-normalized nested canalyzing representation consisting
of $q$ layers is $2q$-symmetric, and is not $(q-1)$-symmetric.
(ii) An $r$-symmetric NCF has a default-normalized 
nested canalyzing representation with at most $r$ layers. \QED
\end{corollary}

\subsection{Testing Whether an $r$-Symmetric Function is an NCF}
\label{sse:rsym_to_ncf}

Theorem~\ref{thm:ncf_r_symmetric} shows that given an NCF $f$,
the problem of finding the smallest integer $r$ such that
$f$ is $r$-symmetric can be solved efficiently.
We now show that the converse problem, that is, testing whether a given
$r$-symmetric Boolean function is an NCF,
can also be solved efficiently.
When the answer is ``yes", a
default-normalized representation (defined in Section~\ref{sec:prelim})
of $f$ can also be constructed efficiently.

Let $f(x_1, x_2, \ldots, x_n)$ be an $r$-symmetric 
Boolean function of $n$ variables.
We assume that the symmetry groups 
of $f$, denoted by $g_1$, $g_2$, $\ldots$, $g_r$,
are given.
Let $m_i = |g_i|$, $1 \leq i \leq r$.
Thus, in group $g_i$, the number
of variables which can take on the value 1 varies from 0
to $m_i$, $1 \leq i \leq r$.
We also assume that $f$ is
given as a table $T$ of the following form:
each row of $T$ specifies an $r$-tuple $(c_1, c_2, \ldots, c_r)$,
where $c_i$ is the number of variables of group $g_i$ which
have the value 1, along with the \{0,1\} value of $f$ for that $r$-tuple.
Thus, $\mu$, the number of rows in $T$, 
is given by  $\mu = \prod_{i=1}^r (m_i+1)$.
Since there are $\mu$ rows and each row is an $r$-tuple,
the size of the table $T$ is $O(r\mu)$.
As will be seen, our algorithm for determining whether $f$ is
an NCF runs in $O(r\mu)$ time.
The algorithm relies on the following observation.

\begin{observation}\label{obs:rsym_canalyzing}
Suppose $f(x_1, x_2, \ldots, x_n)$ is an $r$-symmetric Boolean function.
Consider any variable $x_i$ and suppose
the number of variables in the group
containing $x_i$ is $\nu_i$, $1 \leq i \leq n$.
\begin{enumerate}
\item Function $f$ is canalyzing in variable $x_i$
with canalyzing value 1 iff all the table entries where the number
of 1's in the group containing $x_i$ is nonzero have the same value
of $f$. (This value of $f$ is the canalyzed value when $x_i = 1$.)
\item Function $f$ is canalyzing in variable $x_i$ with
canalyzing value 0 iff all the table entries where the number of
1's in the group containing $x_i$ is less than $\nu_i$ have the same
value of $f$. (This value of $f$ is the canalyzed value when $x_i = 0$.)
\QED
\end{enumerate}
\end{observation}
We now explain how Observation~\ref{obs:rsym_canalyzing} can be
used to develop an iterative algorithm for determining whether 
$f$ is an NCF; if so, the algorithm also constructs a default-normalized 
representation for $f$.
We use the following notation.
At beginning of iteration $j$,~
$r_j$ denotes the number of remaining groups,
$X_j$ denotes a set of variables with
exactly one variable from each remaining group,
$T_j$ denotes the table which provides values for the function and
$\mu_j$ denotes the number of rows of $T_j$.
Initially (i.e., $j = 1$),~ $r_1 = r$ (the number of symmetry groups of $f$),
$X_1$ is constructed by choosing one 
variable (arbitrarily) from each of the $r$ groups,
$T_1 = T$ (the given table $T$ for $f$) and 
$\mu_1 = \mu$ (the number of rows of $T$);
further, the NCF representation for $f$ is empty.
The algorithm carries out iteration $j$ if $r_j \geq 1$; 
in that iteration, the algorithm performs Steps I, II and III
as described below.

\smallskip
\noindent
I. Use Observation~\ref{obs:rsym_canalyzing} to determine whether there is
a canalyzing variable $x_i \in X_j$. 

\smallskip
\noindent
II. If yes, perform the following steps. 
\begin{enumerate}
\item Let $\alpha$ and $\beta$ be the respective canalyzing and 
canalyzed values for $x_i$ found in Step~I. 
\item For each variable $x_p$ in the group containing $x_i$, append the rule~
``$x_p : \alpha ~\longrightarrow~ \beta$"~ to the NCF representation of $f$.
\item Set $X_{j+1}$ = $X_j - \{x_i\}$.  
\item Obtain $T_{j+1}$ by retaining only those rows of $T_j$ where all the
variables in the group containing $x_i$ are set to $\overline{\alpha}$. 
\item Set $r_{j+1} = r_j - 1$.
\item If $r_{j+1} \geq 1$, start the next iteration; 
      otherwise, \textbf{stop}.
\end{enumerate}

\smallskip
\noindent
III. (Here, Step~I didn't find a canalyzing variable.)~ 
Output ``$f$ is not an NCF" and \textbf{stop}.

\medskip
\noindent
The correctness of the algorithm is a direct consequence of
Observation~\ref{obs:rsym_canalyzing}. If the algorithm is successful,
it produces a simplified NCF representation of $f$; this can then be
efficiently converted into the default-normalized representation as
discussed in Section~\ref{sse:ncf_layer}.

To estimate the running time, we note that in iteration $j$, we can
determine whether there is a canalyzing variable in $X_j$
in $O(r_j \mu_j)$ = $O(r \mu_j)$ time using 
Observation~\ref{obs:rsym_canalyzing}.
(This is done by considering each variable in $X_j$.)
Once a canalyzing variable is identified, the other steps in that iteration
can be completed in $O(\mu_j)$ time.
Thus, the time for iteration $j$ is $O(r \mu_j)$.
We note that in each iteration, the number of rows in the table
is reduced by a factor of at least 2; that is, $\mu_{j+1} \leq \mu_j/2$.
To see why, suppose the canalyzing variable $x_i$ found in iteration $j$
is in group $g_p$ with $m_p$ variables. 
Thus, in $T_j$, there are $m_p+1 \geq 2$ possibilities for the number of 1's 
in group $g_p$. 
As indicated in Substep~4 of Step~II above, $T_{j+1}$ 
contains only those rows of $T_j$ where all the variables 
in $g_p$ have the same value.
In other words, there is only one possibility for the number of
1's in group $g_p$.
Thus, $\mu_{j+1}$, the number of rows in $T_{j+1}$,
is at most $\mu_j/(m_p+1)$ $\leq$ $\mu_j/2$.
It follows by simple induction that $\mu_j \leq \mu/2^{j-1}$.
Thus, the running time over all the $r$ iterations is given by
$O(r [\sum_{j=1}^{r}(\mu/2^{j-1})])$ = $O(r\mu)$, since the
geometric sum is bounded by $2\mu$.
Recall that the size of input table $T$ 
(which specifies the $r$-symmetric function $f$)
is $O(r\mu)$. 
Thus, the running time of the algorithm
is linear in the size of the input.
The following theorem summarizes this result. 

\begin{theorem}\label{thm:rsym_canalyzing}
Suppose an $r$-symmetric function $f$ is given as a table where
each row has an $r$-tuple of the form $(c_1, c_2, \ldots, c_r)$,
with $c_i$ being the number of variables in group $g_i$ that have value 1,
along with the \{0,1\} value of the function for that row.
The problem of testing whether $f$ 
is an NCF can be solved in time that 
is linear in the size of the input. \QED
\end{theorem}

When an $r$-symmetric function $f$ with $n$ variables is
specified using the table representation described in the
statement of Theorem~\ref{thm:rsym_canalyzing},
the size of the table is $O(n^r)$.
When $r$ is \emph{fixed}, the size of 
this representation and the running time 
of the above algorithm are both polynomial functions of $n$. 

\subsection{Strong Asymmetry and NCFs}
\label{sse:strong_asym_ncf}

The notion of strong asymmetry of Boolean functions was defined 
in Section~\ref{sec:prelim}.
%% In general, there are functions with $n$ variables that are properly
%% $n$-symmetric, but not strongly asymmetric \cite{KS-2000}.
For general Boolean functions, the notions of ``$n$-symmetry" and
``strong asymmetry" are not equivalent. 
We illustrate this by presenting an example of a Boolean function
which is properly $n$-symmetric but not strongly asymmetric.

\medskip

\noindent
\textbf{Example 6:}
Consider the Boolean function $f$ of four variables, 
namely $a$, $b$, $c$ and $d$,
whose truth table is shown in Table~\ref{tab:not_str_ssym_ex}.
Function $f$ is symmetric under the permutation $cdab$.
For convenience, we give the truth table for $f$ in a form
that makes its symmetry under the permutation $cdab$ clear.
So, the function $f$ is not strongly asymmetric.
We now demonstrate that the function is properly $4$-symmetric
by observing that it is not symmetric with respect to any pair of variables.

\medskip

\noindent
\begin{minipage}{0.01\textwidth}
%\hspace*{0.03in}
\end{minipage}
\begin{minipage}{0.85\textwidth}
\begin{description}
\item{(i)} $f(0100) \neq f(1000)$, so $a$ and $b$ are not symmetric.
\item{(ii)} $f(0011) \neq f(1001)$, so $a$ and $c$ are not symmetric.
\item{(iii)} $f(0001) \neq f(1000)$, so $a$ and $d$ are not symmetric.
\item{(iv)} $f(0100) \neq f(0010)$, so $b$ and $c$ are not symmetric.
\item{(v)} $f(0011) \neq f(0110)$, so $b$ and $d$ are not symmetric.
\item{(vi)} $f(0001) \neq f(0010)$, so $c$ and $d$ are not symmetric.
\end{description}
\end{minipage}

%%%% Truth table for the Boolean function in Example 6.

\begin{table}[tb]
\begin{center}
\begin{tabular}{|l|l|c||l|l|c|}\hline
$a~b$ & $c~d$ & {Value of $f$} & $a~b$ & $c~d$ & {Value of $f$} \\ \hline\hline
0~0 & 0~1 & 0 & 0~1 & 1~1 & 1 \\ \hline
0~1 & 0~0 & 0 & 1~1 & 0~1 & 1  \\ \hline
0~0 & 1~0 & 1 & 1~0 & 1~1 & 1 \\ \hline
1~0 & 0~0 & 1 & 1~1 & 1~0 & 1 \\ \hline
0~0 & 1~1 & 0 & 0~0 & 0~0 & 0 \\ \hline
1~1 & 0~0 & 0 & 0~1 & 0~1 & 0 \\ \hline
0~1 & 1~0 & 1 & 1~0 & 1~0 & 0 \\ \hline
1~0 & 0~1 & 1 & 1~1 & 1~1 & 0 \\ \hline
\end{tabular}
\end{center}
\caption{An example of a Boolean function with 4 variables which is
properly $4$-symmetric but not\newline strongly asymmetric.}
\label{tab:not_str_ssym_ex}
\end{table}

%%%%% End of truth table.

We now show that any $n$ variable NCF that is properly $n$-symmetric
is also strongly asymmetric.
As a consequence, we observe that there are NCFs with $n$ variables
that are properly $n$-symmetric.

\begin{theorem}\label{thm:ncf_strong_asymmetry}
An NCF with $n$ variables is strongly asymmetric iff
it is properly $n$-symmetric.
\end{theorem}

\noindent
\textbf{Proof:}~ 
If an NCF is $(n-1)$-symmetric, then consider a permutation that
interchanges two variables from a symmetry group with at least two
members.  The value of the function is invariant under any such
permutation, so the function is not strongly asymmetric.

For the converse, consider an $n$ variable NCF that is properly $n$-symmetric.
Let $f$ be the default-normalized NCF representation for the function.  For
purposes of notational simplicity, without loss of generality,
we assume that the canalyzing
variable in line $i$ of $f$ is $x_i$, $1 \leq i \leq n$.  Let $\pi$
be any permutation of $\{1, 2, \ldots, n\}$ \emph{except} the
identity permutation.  We will construct an assignment $(c_1, c_2,
\ldots,  c_n)$ to the variables of $f$ such that $f(c_1, c_2, \ldots,
c_n)$ $\neq$ $f(c_{\pi(1)}, c_{\pi(2)}, \ldots, c_{\pi(n)})$.

Let $k$ be the smallest index such that $k \neq \pi(k)$.  Since
$\pi$ is a permutation, $k < n$.  Overall, for $1 \leq i < k$,
$\pi(i) = i$,~ $\pi(k) > k$,~ and for $k < i \leq n$,~ $\pi(i) \geq k$.

\smallskip

Suppose that line $i$ of $f$, $1 \leq i \leq n$, is 

\smallskip

\hspace*{0.25in} $x_i : a_i ~\longrightarrow~ b_i$. 

\smallskip

\noindent
Assignment $(c_1, c_2, \ldots,  c_n)$ is constructed as follows.
\begin{enumerate}
\item For $1 \leq i < k$, set $c_i = \overline{a_i}$.  
\item For $i = k$, set $c_k =a_k$. 
\item For $i > k$, let $i' = \pi^{-1}(i)$, so that $\pi(i')
= i$.  Thus, after the permutation of values, variable $x_{i'}$
will have value $c_i$.  If $b_{i'} = b_k$, then set $c_i =
\overline{a_{i'}}$, else set $c_i =  a_{i'}$.
\end{enumerate}
Since $c_k$ matches the canalyzing value of line $k$ of $f$, and
$c_i$ does not match the canalyzing value of any other earlier line
$i$, $1 \leq i < k$, we have that $f(c_1, c_2, \ldots,  c_n) = b_k$.

Now consider $f(c_{\pi(1)}, c_{\pi(2)}, \ldots, c_{\pi(n)})$.  We
first note that for $1 \leq i < k$, $c_{\pi(i)}$ does not match the
canalyzing value of line $i$.

Let $k' = \pi^{-1}(k)$, so that $\pi(k') = k$.  Canalyzing value
$a_{k'}$ is either the same or the complement of $a_k$, and canalyzed
value $b_{k'}$ is either the same or the complement of $b_k$, Thus,
there are four possible cases for the form of line $k'$ of $f$.  We
will show that in each of the four cases, $f(c_{\pi(1)}, c_{\pi(2)},
\ldots, c_{\pi(n)}) = \overline{b_k}$.

\medskip

\noindent
{\bf Case 1:} Line $k'$ has the form  $x_{k'} : a_k ~\longrightarrow~ b_k$.  

\smallskip

Since line $k'$
has the same canalyzing value and canalyzed value as line $k$, and
$f$ is properly $n$-symmetric, line $k'$ must occur in a lower layer
than that of line $k$.  Thus, there is at least one line between
line $k$ and line $k'$ for which the canalyzed value is $\overline{b_k}$.
Let line $q$ be the first such line.  Note that for all $i$ such
that $1 \leq i < q$, $c_{\pi(i)}$ does not match the canalyzing
value of line $i$, but $c_{\pi(q)}$ does match the canalyzing value
of line $q$.  Since canalyzed value $b_q$ equals $\overline{b_k}$,
we have that $f(c_{\pi(1)}, c_{\pi(2)}, \ldots, c_{\pi(n)}) =
\overline{b_k}$.

\medskip

\noindent
{\bf Case 2:} Line $k'$ has the form $x_{k'} : a_k ~\longrightarrow~ \overline{b_k}$.

\smallskip

Let line $q$ be the first line such that $k < q \leq k'$ and
the canalyzed value of line $q$ is $\overline{b_k}$.
Note that for all $i$ such that $1 \leq i < q$, 
$c_{\pi(i)}$ does not match the canalyzing value of line $i$,
so $f(c_{\pi(1)}, c_{\pi(2)}, \ldots, c_{\pi(n)}) = \overline{b_k}$.

\medskip

\noindent
{\bf Case 3:} Line $k'$ has the form $x_{k'} : \overline{a_k} ~\longrightarrow~ b_k$. 

\smallskip

Suppose there is a line below line $k$ for which the canalyzed
value is $\overline{b_k}$.  Let $q$ be the first such line.  Note
that for all $i$ such that $1 \leq i < q$, $c_{\pi(i)}$ does not
match the canalyzing value of line $i$, but $c_{\pi(q)}$ does match
the canalyzing value of line $q$.  Thus, $f(c_{\pi(1)}, c_{\pi(2)},
\ldots, c_{\pi(n)}) = \overline{b_k}$.

Now suppose there is no line below line $k$ for which the canalyzed
value is $\overline{b_k}$.  
Then line $k$ occurs in the last layer of $f$.  
Since $f$ is both default-normalized and properly $n$-symmetric,
this last layer contains exactly two lines.  Thus $k = n-1$ and $k'
= n$.  Note that for all $i$, $c_{\pi(i)}$ does not match the
canalyzing value of line $i$.  Thus, $f(c_{\pi(1)}, c_{\pi(2)},
\ldots, c_{\pi(n)}) = \overline{b_k}$.

\medskip

\noindent
{\bf Case 4:} Line $k'$ has the form $x_{k'} : \overline{a_k} ~\longrightarrow~ 
              \overline{b_k}$. 

\smallskip

Suppose there is a line, other than line $k'$, below line $k$ for
which the canalyzed value is $\overline{b_k}$.  Let $q$ be the
first such line.  Note that for all $i$ such that $1 \leq i < q$,
$c_{\pi(i)}$ does not match the canalyzing value of line $i$, but
$c_{\pi(q)}$ does match the canalyzing value of line $q$.  Thus,
$f(c_{\pi(1)}, c_{\pi(2)}, \ldots, c_{\pi(n)}) = \overline{b_k}$.

Now suppose that line $k'$ is the only line below line $k$ for which
the canalyzed value is $\overline{b_k}$.  Since line $k$ has
canalyzed value $b_k$, line $k'$ is the only line in its layer.
Since $f$ is default-normalized, there is a last layer following the layer
containing line $k'$.  Thus, for all $i$, $c_{\pi(i)}$ does not
match the canalyzing value of line $i$; therefore, 
$f(c_{\pi(1)}, c_{\pi(2)}, \ldots, c_{\pi(n)})$ ~=~ $\overline{b_k}$.  \QED

\medskip

\noindent
We now provide
examples of properly $n$-symmetric $n$-variable NCFs satisfying the
conditions of Theorem \ref{thm:ncf_r_symmetric}.
For each $n > 1$, let $f_n$ be the function of $n$ variables, namely
$x_1, \ldots, x_n$, defined by the following formula:

\smallskip

\hspace*{0.5in} $x_1 \vee (\overline{x}_2 \wedge 
    ( x_3 \vee (\overline{x}_4 \wedge (\cdots ))))$

\smallskip

\noindent
For instance,
\begin{align*}
f_6 &~=~  x_1 \vee (\overline{x}_2 \wedge (x_3 \vee (\overline{x}_4 \wedge 
      (x_5 \vee \overline{x}_6))))~~~ \mathrm{and}\\
f_7 &~=~  x_1 \vee (\overline{x}_2 \wedge (x_3 \vee (\overline{x}_4 \wedge (x_5 
          \vee (\overline{x}_6 \wedge  x_7))))).
\end{align*}
Function $f_n(x_1, x_2, \ldots, x_n)$ is the NCF corresponding to
the following NCF representation:

\medskip

\noindent
\hspace*{0.25in}
$x_1:~ 1 ~\longrightarrow~ 1$ \\
\hspace*{0.25in}
$x_2:~ 1 ~\longrightarrow~ 0$ \\
\hspace*{0.25in}
$x_3:~ 1 ~\longrightarrow~ 1$ \\
\hspace*{0.75in}
$\vdots$ 

\smallskip

\noindent
If $n$ is odd, the last line is 

\smallskip

\noindent
\hspace*{0.25in}
$x_n:~ 1 ~\longrightarrow~ 1$ 

\smallskip

\noindent
and if $n$ is even, the last line is 

\smallskip

\noindent
\hspace*{0.25in}
$x_n:~ 1 ~\longrightarrow~ 0$. 

\smallskip

\noindent
Note that if $x_i = 0$ for all $i$, $1 \leq i \leq n$, then the
value of $f_n$ is 0 if $n$ is odd and 1 if $n$ is even.  Also, note
that the above representation is not default-normalized. 
To obtain a default-normalized 
representation, the last line would be changed to have canalyzing
value 0, and the same canalyzed value as the preceding line.

\subsection{Number of Strongly Asymmetric NCFs}
\label{sse:number_strongly_asymmetric}

\begin{theorem}\label{thm:count_strongly_asymmetric}
For any $n \geq 2$, the number of Boolean functions with $n$ variables 
that are both strongly asymmetric and NCFs is~ $n! \, 2^{n-1}$.
\end{theorem}

\noindent
\textbf{Proof:}~
Consider a default-normalized representation of a strongly asymmetric NCF $f$.
The representation has $n-1$ layers.
The first $n-2$ layers each contain one line, and the last layer contains two lines.

Consider the canalyzed values in $f$. The canalyzed value in the
first layer can be either 0 or 1.  The canalyzed value in every
other layer is the complement of the canalyzed value in the preceding
layer.  Thus, there are 2 possibilities for the sequence of canalyzed
values in $f$.

The canalyzing variable in each of the first $n-2$ lines (one per
layer) can be any variable that has not yet occurred in a preceding
line.  The last layer contains the remaining two variables.  Thus,
there are $n! /[n-(n-2)]!$ = $n!/2$ possibilities for the pattern of 
canalyzing variable occurrences in $f$.

For each of the first $n-2$ variables, the canalyzing value for the line
containing that variable can be either 0 or 1.  
%% The last two lines, the canalyzed values on those two lines must be
%% the same because of the default-normalized representation. 
For the last two lines, the canalyzing values for the variables 
in those lines must be complements of each other; otherwise, by 
Theorem~\ref{thm:ncf_symmetric_variables}, those two variables 
will be symmetric.
Thus, the number of possible canalyzing values over all the $n$
lines is $2^{n-2} \times 2$ = $2^{n-1}$.

Hence, the number of Boolean functions with $n$ variables that
are both strongly asymmetric and NCFs is equal to 
$2 \times (n!/2) \times 2^{n-1}$ ~=~ $n!\,2^{n-1}$.  \QED


\subsection{Symmetric Canalyzing and Nested Canalyzing Functions}
\label{sse:sym_and_cf_ncf}

\begin{proposition}\label{pro:ncf_symmetric}
(i) The only symmetric canalyzing functions are OR, AND, NOR, NAND, 
the constant function 0 and the constant function 1.
(ii) The only symmetric NCFs are OR, AND, NOR and NAND.
\end{proposition}
%%\medskip
\noindent
\textbf{Proof:}~

\smallskip

\noindent
\textbf{Part (i):}~
Suppose symmetric function $f$ is also a canalyzing function. 
Then there is a variable $x$, and values $a$ and $b$ 
such that whenever $x = a$, function $f$ has value $b$.  
Since $f$ is symmetric, $f$  has
value $b$ whenever any of its variables has value $a$.  
Thus, if at least one of its variables has value $a$, then $f$ has value $b$.  
If $f$ has value $\overline{b}$ when none of its variables has
value $a$, the four possible combinations of values for $a$ and $b$
correspond to the four functions OR, AND, NOR, and NAND.  If $f$
has value $b$ when none of its variables has value $a$, then $f$
is the constant function $b$.

\smallskip

\noindent
\textbf{Part (ii):}~
Suppose that symmetric function $f$ is also an NCF.  
Since constant functions are not NCFs,
$f$ has value $\overline{b}$ when none of its variables has value $a$.
The four possible combinations of values for $a$ and $b$ correspond
to the four functions OR, AND, NOR and NAND, each of which is an
NCF.  \QED








   %% Main results. 

%% \section{Conclusion}
%% The conclusion goes here.


\iffalse
%%%%%%%%%%%%%%%%%%%%%%%%%%%%%%%%%%%%%
% use section* for acknowledgment
\ifCLASSOPTIONcompsoc
  % The Computer Society usually uses the plural form
  \section*{Acknowledgments}
\else
  % regular IEEE prefers the singular form
  \section*{Acknowledgment}
\fi
%%%%%%%%%%%%%%%%%%%%%%%%%%%%%%%%%%%%%
\fi

\section*{Acknowledgments}
This work has been partially supported by
DTRA CNIMS (Contract HDTRA1-11-D-0016-0001),
NSF DIBBS Grant ACI-1443054 and
NSF BIG DATA Grant IIS-1633028.
The U.S. Government is authorized to reproduce and
distribute reprints for Governmental purposes notwithstanding
any copyright annotation thereon.

%%\bigskip\bigskip
\smallskip

\noindent
\textbf{Disclaimer:}~ The views and conclusions contained
herein are those of the authors and should
not be interpreted as necessarily representing the
official policies or endorsements, either expressed
or implied, of the U.S. Government.


% Can use something like this to put references on a page
% by themselves when using endfloat and the captionsoff option.

%% \ifCLASSOPTIONcaptionsoff
%%   \newpage
%% \fi


% trigger a \newpage just before the given reference
% number - used to balance the columns on the last page
% adjust value as needed - may need to be readjusted if
% the document is modified later
%\IEEEtriggeratref{8}
% The "triggered" command can be changed if desired:
%\IEEEtriggercmd{\enlargethispage{-5in}}

% references section
\bibliographystyle{IEEEtran}
\bibliography{refs}

% can use a bibliography generated by BibTeX as a .bbl file
% BibTeX documentation can be easily obtained at:
% http://mirror.ctan.org/biblio/bibtex/contrib/doc/
% The IEEEtran BibTeX style support page is at:
% http://www.michaelshell.org/tex/ieeetran/bibtex/
%\bibliographystyle{IEEEtran}
% argument is your BibTeX string definitions and bibliography database(s)
%\bibliography{IEEEabrv,../bib/paper}
%
% <OR> manually copy in the resultant .bbl file
% set second argument of \begin to the number of references
% (used to reserve space for the reference number labels box)

% biography section
% 
% If you have an EPS/PDF photo (graphicx package needed) extra braces are
% needed around the contents of the optional argument to biography to prevent
% the LaTeX parser from getting confused when it sees the complicated
% \includegraphics command within an optional argument. (You could create
% your own custom macro containing the \includegraphics command to make things
% simpler here.)
%\begin{IEEEbiography}[{\includegraphics[width=1in,height=1.25in,clip,keepaspectratio]{mshell}}]{Michael Shell}
% or if you just want to reserve a space for a photo:

\iffalse
%%%%%%%%%%%%%%%%%%%%%%%%%%%%%%%%%%%%%%%%%%%%%%%%%%%
\begin{IEEEbiography}{Michael Shell}
Biography text here.
\end{IEEEbiography}

% if you will not have a photo at all:
\begin{IEEEbiographynophoto}{John Doe}
Biography text here.
\end{IEEEbiographynophoto}

% insert where needed to balance the two columns on the last page with
% biographies
%\newpage

\begin{IEEEbiographynophoto}{Jane Doe}
Biography text here.
\end{IEEEbiographynophoto}
%%%%%%%%%%%%%%%%%%%%%%%%%%%%%%%%%%%%%%%%%%%%%%%%%%%
\fi

% You can push biographies down or up by placing
% a \vfill before or after them. The appropriate
% use of \vfill depends on what kind of text is
% on the last page and whether or not the columns
% are being equalized.

%\vfill

% Can be used to pull up biographies so that the bottom of the last one
% is flush with the other column.
%\enlargethispage{-5in}

% that's all folks
\end{document}
