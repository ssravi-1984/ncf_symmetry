\section{Symmetry of Nested Canalyzing\newline Functions}
\label{sec:ncf_and_symmetry}

\subsection{Overview}
\label{sse:res_overview}
Our first result (Theorem~\ref{thm:ncf_symmetric_variables})
provides a characterization of pairs of variables of an NCF
that are part of the same symmetry group. 
This characterization allows us to give a simple closed form
expression (Theorem~\ref{thm:ncf_r_symmetric}) 
for the symmetry level of an NCF specified using
its default-normalized representation.
We also present an efficient algorithm (Theorem~\ref{thm:rsym_canalyzing})
for the converse problem, that is,
testing whether a given $r$-symmetric function $f$ is an NCF;
if so, our algorithm constructs a default-normalized 
representation for $f$.
Next, we show that for any NCF $f$ with $n$ variables,
the two statements ``$f$ is strongly asymmetric" ~and~
``the symmetry level of $f$ is $n$"~ are equivalent 
(Theorem~\ref{thm:ncf_strong_asymmetry}).
We use this result to derive a closed form expression 
for the number of $n$-variable NCFs that are 
also strongly asymmetric (Theorem~\ref{thm:count_strongly_asymmetric}).
Our final result (Proposition~\ref{pro:ncf_symmetric})
identifies all the functions that are symmetric and
canalyzing as well as those that are symmetric and NCFs.

\subsection{Symmetric Pairs of Variables in an NCF: A Characterization
and Its Applications}
\label{sse:ncf_strong_sym}

\begin{theorem}\label{thm:ncf_symmetric_variables}
Two variables of an NCF $f$ are symmetric iff
in the default-normalized representation of $f$,
they occur in the same layer and have the same canalyzing value.
\end{theorem}
\noindent
\textbf{Proof:}~
Suppose that two variables of an NCF $f$ occur in the same layer of a
default-normalized representation, and the lines containing these variables
have the same canalyzing value.  Then in any input assignment, the
values of these two variables can be interchanged, and the evaluation
of the lines in that layer will have the same effect.  Thus, the
variables are symmetric.

For the converse, consider the default-normalized NCF representation of $f$. 
As mentioned earlier, we assume without loss of generality that the
canalyzing variable in line $i$ of $f$ is $x_i$, $1 \leq i \leq n$.
Let  $x_j$ and $x_k$ be two variables that occur in different
layers of $f$, or occur in the same layer, but with different
canalyzing values.  Without loss of generality, 
assume that $j < k$, so that the line for
$x_j$ occurs above the line for $x_k$.
Suppose that line $i$ of $f$, $1 \leq i \leq n$, is 

\smallskip

\hspace*{0.25in}
$x_i : a_i ~\longrightarrow~ b_i$. 

\smallskip

\noindent
Consider the following assignment $\alpha$ = $(c_1, c_2, \ldots,  c_n)$ 
to the variables $x_1$, $x_2$, $\ldots$, $x_n$ of $f$.
\begin{enumerate}
\item For $1 \leq i < j$, set $c_i = \overline{a_i}$. 
\item For $i = j$, set $c_j =a_j$. 
\item For $i = k$, set $c_k =\overline{a_j}$. 
\item For $j < i < k$, and for  $k < i \leq n$, 
if $b_i = b_j$,\\ then set $c_i =  \overline{a_i}$,
else set $c_i =  a_i$.
\end{enumerate}
Note that $f(\alpha) = b_j$, since the variables in all lines above
line $j$ have the complement of their canalyzing value, and variable
$x_j$ has its canalyzing value.

\medskip

Now, let $\alpha'$ be the assignment that is obtained from $\alpha$
by interchanging the values of variables $x_j$ and $x_k$, so that
after the interchange, variable $x_j$ has value $\overline{a_j}$ and
variable $x_k$ has value $a_j$.  We will show that $f(\alpha') =
\overline{b_j}$, so variables $x_j$ and $x_k$ are not symmetric.

In line $k$ of $f$, canalyzing value $a_k$ is either the same or
the complement of canalyzing value $a_j$, and canalyzed value $b_k$
is either the same or the complement of canalyzed value $b_j$,
Thus, there are four possible cases for the form of line $k$.  We
now consider each of the four cases.

\medskip

\noindent
{\bf Case 1:} Line $k$ has the form $x_k : a_j ~\longrightarrow~ b_j$.  

\smallskip

Since line $k$ has
the same canalyzing value and canalyzed value as line $j$, line
$k$ must occur in a lower layer than that of line $j$.  Thus, there
is at least one line between line $j$ and line $k$ for which the
canalyzed value is $\overline{b_j}$.  Let line $q$ be the first
such line.  Note that for all $i$ such that $1 \leq i < q$, the
value of variable $x_i$ in assignment $\alpha'$ does not match the
canalyzing value of line $i$, but the value of variable $x_q$ does
match the canalyzing value of line $q$.  Since canalyzed value
$b_q$ equals $\overline{b_j}$, we have that $f(\alpha') =
\overline{b_j}$.

\medskip

\noindent
{\bf Case 2:} Line $k$ has the form  $x_k : a_j ~\longrightarrow~ \overline{b_j}$.  

\smallskip

Let line $q$ be the first line such that $j < q \leq k$ and the canalyzed
value of line $q$ is $\overline{b_j}$.  Note that $q$ might possibly
equal $k$.  For all $i$ such that $1 \leq i < q$, the value of
variable $x_i$ in assignment $\alpha'$ does not match the canalyzing
value of line $i$, but the value of variable $x_q$ does match the
canalyzing value of line $q$.  Since canalyzed value $b_q$ equals
$\overline{b_j}$, we have that $f(\alpha') = \overline{b_j}$.

\medskip

\noindent
{\bf Case 3:} Line $k$ has the form $x_k : \overline{a_j} ~\longrightarrow~ b_j$.  

\smallskip

Suppose there is a line below line $j$ for which the canalyzed value is
$\overline{b_j}$.  Let $q$ be the first such line.  Note that for
all $i$ such that $1 \leq i < q$, the value of variable $x_i$ in
assignment $\alpha'$ does not match the canalyzing value of line
$i$, but the value of variable $x_q$ does match the canalyzing value
of line $q$.  Since canalyzed value $b_q$ equals $\overline{b_j}$,
we have that $f(\alpha') = \overline{b_j}$.

\smallskip

Now suppose there is no line below line $j$ for which the canalyzed
value is $\overline{b_j}$.  Then lines $j$ and $k$ both occur in
the last layer of $f$.  Thus, for every line $i$, $1 \leq i \leq
n$, the value of variable $x_i$ in assignment $\alpha'$ does not
match the canalyzing value of line $i$.  Consequently, $f(\alpha')
= \overline{b_j}$, the complement of the canalyzed value in the
last layer of $f$.

\medskip

\noindent
{\bf Case 4:} Line $k$ has the form  $x_k : \overline{a_j} ~\longrightarrow~ \overline{b_j}$. 

\smallskip

Suppose there is a line, other than line $k$, below line $j$ for
which the canalyzed value is $\overline{b_j}$.  Let $q$ be the
first such line.  Note that for all $i$ such that $1 \leq i < q$,
the value of variable $x_i$ in assignment $\alpha'$ does not match
the canalyzing value of line $i$, but the value of variable $x_q$
does match the canalyzing value of line $q$.  Since canalyzed value
$b_q$ equals $\overline{b_j}$, we have that $f(\alpha') =
\overline{b_j}$.

Now suppose that line $k$ is the only line below line $j$ for which
the canalyzed value is $\overline{b_j}$.  Since line $j$ has
canalyzed value $b_j$, line $k$ is the only line in its layer.
Since $f$ is default-normalized, the last layer of $f$ contains at least
two lines.  Thus, there is a last layer following the layer containing
line $k$.  The canalyzed value of all the lines in this last layer
is $b_j$.  Thus, for every line $i$, $1 \leq i \leq n$, the value
of variable $x_i$ in assignment $\alpha'$ does not match the
canalyzing value of line $i$.  Consequently, $f(\alpha') =
\overline{b_j}$, the complement of the canalyzed value in the last
layer of $f$.  \QED

\begin{theorem}\label{thm:ncf_r_symmetric}
Suppose the default-normalized representation of a given NCF contains 
$r_1$ layers with only one distinct canalyzing value,
and $r_2$ layers with two distinct canalyzing values.
Then the function is properly $(r_1 + 2 r_2)$-symmetric.
\end{theorem}

\noindent
\textbf{Proof:}~
From Theorem~\ref{thm:ncf_symmetric_variables}, 
any pair of variables occurring in the same layer, 
with the same canalyzing value, are symmetric.
Thus the function is at most $(r_1 + 2 r_2)$-symmetric.
Moreover, any pair of variables from different layers, or with different canalyzing values
are not symmetric, so there are at least $(r_1 + 2 r_2)$ symmetry groups.
\QED

\begin{corollary}\label{cor:ncf_not_rsymm}
For every $n \geq 2$, there is an $n$-variable NCF that is not $n-1$ symmetric. \QED
\end{corollary}

\begin{corollary}\label{cor:ncf_r_symmetric_layers}
(i) An NCF with a default-normalized nested canalyzing representation consisting
of $q$ layers is $2q$-symmetric, and is not $(q-1)$-symmetric.
(ii) An $r$-symmetric NCF has a default-normalized nested canalyzing representation
with at most $r-1$ layers, and has an unnormalized nested canalyzing
representation with at most $r$ layers.   \QED
\end{corollary}

\subsection{Testing Whether an $r$-Symmetric Function is an NCF}
\label{sse:rsym_to_ncf}

Theorem~\ref{thm:ncf_r_symmetric} shows that given an NCF $f$,
the problem of finding the smallest integer $r$ such that
$f$ is $r$-symmetric can be solved efficiently.
We now show that the converse problem, that is, testing whether a given
$r$-symmetric Boolean function is an NCF,
can also be solved efficiently.
When the answer is ``yes", a
default-normalized representation (defined in Section~\ref{sec:prelim}
of $f$ can also be constructed efficiently.

Let $f(x_1, x_2, \ldots, x_n)$ be an $r$-symmetric 
Boolean function of $n$ variables.
We assume that the symmetry groups $g_i$, $1 \leq i \leq r$,
of $f$ are given.
Let $m_i = |g_i|$, $1 \leq i \leq m$.
Thus, in group $g_i$, the number
of variables which can take on the value 1 varies from 0
to $m_i$, $1 \leq i \leq r$.
We also assume that $f$ is
given as a table $T$ of the following form:
each row of $T$ specifies an $r$-tuple $(c_1, c_2, \ldots, c_r)$,
where $c_i$ is the number of variables of group $g_i$ which
have the value 1, along with the \{0,1\} value of $f$ for that $r$-tuple.
Thus, $\mu$, the number of rows in $T$, 
is given by  $\mu = \prod_{i=1}^r (m_i+1)$.
As will be seen, our algorithm for determining whether $f$ is
an NCF runs in $O(r\mu)$ time.
The algorithm relies on the following observation.

\begin{observation}\label{obs:rsym_canalyzing}
Suppose $f(x_1, x_2, \ldots, x_n)$ is an $r$-symmetric Boolean function.
Consider any variable $x_i$ and suppose
the number of variables in the group
containing $x_i$ is $\alpha_i$, $1 \leq i \leq n$.
\begin{enumerate}
\item Function $f$ is canalyzing in variable $x_i$
with canalyzing value 1 iff all the table entries where the number
of 1's in the group containing $x_i$ is nonzero have the same value
of $f$. (This value of $f$ is the canalyzed value when $x_i = 1$.)
\item Function $f$ is canalyzing in variable $x_i$ with
canalyzing value 0 iff all the table entries where the number of
1's in the group containing $x_i$ is less than $\alpha_i$ have the same
value of $f$. (This value of $f$ is the canalyzed value when $x_i = 0$.)
\QED
\end{enumerate}
\end{observation}
We now explain how Observation~\ref{obs:rsym_canalyzing} can be
used to develop an iterative algorithm for determining whether 
$f$ is an NCF; if so, the algorithm also constructs a default-normalized 
representation for $f$.
At beginning of iteration $j$, the algorithm has the set $X_j$ of remaining variables,
table $T_j$ with $\mu_j$ rows and $\ell_j$ remaining groups.
Initially (i.e., $j = 1$), $X_1$ = $X$ = $\{x_1, \ldots, x_n\}$,
$T_1 = T$ (the given table $T$ for $f$), 
$\mu_1 = \mu$ (the number of rows of $T$), $\ell_j = r$ (the number of
symmetry groups of $f$) and the NCF representation for $f$ is empty.
The algorithm carries out iteration $j$ if $\ell_j \geq 1$; 
in that iteration, the algorithm performs Steps I, II and III
as described below.

\smallskip
\noindent
I. Use Observation~\ref{obs:rsym_canalyzing} to determine whether there is
a canalyzing variable $x_i \in X_j$. 

\smallskip
\noindent
II. If yes, perform the following steps. 
\begin{enumerate}
\item Let $\alpha$ and $\beta$ be the respective canalyzing and 
canalyzed values for $x_i$ found in Step~I. 
\item For each variable $x_p$ in the group containing $x_i$, append the rule~
``$x_p\,:\,\alpha ~\longrightarrow~ \beta$"~ to the NCF representation of $f$.
\item Obtain $X_{j+1}$ by deleting from $X_j$ all the variables which are 
in the same group as $x_i$.
\item Set $\ell_{j+1} = \ell_j - 1$.
\item Obtain $T_{j+1}$ by retaining only those rows of $T_j$ where all the
variables in the group containing $x_i$ are set to $\overline{\alpha}$. 
\item Start the next iteration.
\end{enumerate}

\smallskip
\noindent
III. (Here, Step~I didn't find a canalyzing variable.)~ 
Output ``$f$ is not an NCF" and \textbf{stop}.

\medskip
\noindent
The correctness of the algorithm is a direct consequence of
Observation~\ref{obs:rsym_canalyzing}. If the algorithm is successful,
it produces a simplified NCF representation of $f$; this can then be
efficiently converted into the default-normalized representation as
discussed in Section~\ref{sse:ncf_layer}.

To estimate the running time, we note that in iteration $j$, we can
determine whether there is a canalyzing variable in $X_j$
in $O(\ell_j \mu_j)$ = $O(r \mu_j)$ time using 
Observation~\ref{obs:rsym_canalyzing}.
(This can be done by considering an arbitrary variable 
from each group.)
Once a canalyzing variable is identified, the other steps in that iteration
can be completed in $O(\mu_j)$ time.
Thus, the time for iteration $j$ is $O(r \mu_j)$.
We note that in each iteration, the number of rows in the table
is reduced by at least a factor of 2; that is, $\mu_{j+1} \leq \mu_j/2$.
To see why, suppose the canalyzing variable $x_i$ found in iteration $j$
is in group $g_p$ with $m_p$ variables. 
Thus, in $T_j$, there are $m_p+1 \geq 2$ possibilities for the number of 1's 
in group $g_p$. 
As indicated in Substep~5 of Step~II above, $T_{j+1}$ 
contains only those rows of $T_j$ where all the variables 
in $g_p$ have the same value.
In other words, there is only one possibility for the number of
1's in group $g_p$.
Thus, $\mu_{j+1}$, the number of rows in $T_{j+1}$,
is at most $\mu_j/(m_p+1)$ $\leq$ $\mu_j/2$.
It follows by simple induction that $\mu_j \leq \mu/2^{j-1}$.
Thus, the running time over all the $r$ iterations is given by
$O(r [\sum_{j=1}^{r}(\mu/2^{j-1})])$ = $O(r\mu)$, since the
geometric sum is bounded by $2\mu$.
The following theorem summarizes this result. 

\begin{theorem}\label{thm:rsym_canalyzing}
The problem of testing whether a given 
$r$-symmetric function $f$ is an NCF can 
be solved efficiently. \QED
\end{theorem}

\subsection{Strong Asymmetry and NCFs}
\label{sse:strong_asym_ncf}

The notion of strong asymmetry of Boolean functions was defined 
in Section~\ref{sec:prelim}.
In general, there are functions with $n$ variables that are properly
$n$-symmetric, but not strongly asymmetric \cite{KS-2000}.
We now show that any $n$ variable NCF that is properly $n$-symmetric
is also strongly asymmetric.
As a consequence, we observe that there are NCFs with $n$ variables
that are properly $n$-symmetric.

\begin{theorem}\label{thm:ncf_strong_asymmetry}
An NCF with $n$ variables is strongly asymmetric iff
it is properly $n$-symmetric.
\end{theorem}

\noindent
\textbf{Proof:}~ 
If an NCF is $(n-1)$-symmetric, then consider a permutation that
interchanges two variables from a symmetry group with at least two
members.  The value of the function is invariant under any such
permutation, so the function is not strongly asymmetric.

For the converse, consider an $n$ variable NCF that is properly $n$-symmetric.
Let $f$ be the default-normalized NCF representation for the function.  For
purposes of notational simplicity, without loss of generality,
we assume that the canalyzing
variable in line $i$ of $f$ is $x_i$, $1 \leq i \leq n$.  Let $\pi$
be any permutation of $\{1, 2, \ldots, n\}$ \emph{except} the
identity permutation.  We will construct an assignment $(c_1, c_2,
\ldots,  c_n)$ to the variables of $f$ such that $f(c_1, c_2, \ldots,
c_n)$ $\neq$ $f(c_{\pi(1)}, c_{\pi(2)}, \ldots, c_{\pi(n)})$.

Let $k$ be the smallest index such that $k \neq \pi(k)$.  Since
$\pi$ is a permutation, $k < n$.  Overall, for $1 \leq i < k$,
$\pi(i) = i$,~ $\pi(k) > k$,~ and for $k < i \leq n$,~ $\pi(i) \geq k$.

\smallskip

Suppose that line $i$ of $f$, $1 \leq i \leq n$, is 

\smallskip

\hspace*{0.25in} $x_i : a_i ~\longrightarrow~ b_i$. 

\smallskip

\noindent
Assignment $(c_1, c_2, \ldots,  c_n)$ is constructed as follows.
\begin{enumerate}
\item For $1 \leq i < k$, set $c_i = \overline{a_i}$.  
\item For $i = k$, set $c_k =a_k$. 
\item For $i > k$, let $i' = \pi^{-1}(i)$, so that $\pi(i')
= i$.  Thus, after the permutation of values, variable $x_{i'}$
will have value $c_i$.  If $b_{i'} = b_k$, then set $c_i =
\overline{a_{i'}}$, else set $c_i =  a_{i'}$.
\end{enumerate}
Since $c_k$ matches the canalyzing value of line $k$ of $f$, and
$c_i$ does not match the canalyzing value of any other earlier line
$i$, $1 \leq i < k$, we have that $f(c_1, c_2, \ldots,  c_n) = b_k$.

Now consider $f(c_{\pi(1)}, c_{\pi(2)}, \ldots, c_{\pi(n)})$.  We
first note that for $1 \leq i < k$, $c_{\pi(i)}$ does not match the
canalyzing value of line $i$.

Let $k' = \pi^{-1}(k)$, so that $\pi(k') = k$.  Canalyzing value
$a_{k'}$ is either the same or the complement of $a_k$, and canalyzed
value $b_{k'}$ is either the same or the complement of $b_k$, Thus,
there are four possible cases for the form of line $k'$ of $f$.  We
will show that in each of the four cases, $f(c_{\pi(1)}, c_{\pi(2)},
\ldots, c_{\pi(n)}) = \overline{b_k}$.

\medskip

\noindent
{\bf Case 1:} Line $k'$ has the form  $x_{k'} : a_k ~\longrightarrow~ b_k$.  

\smallskip

Since line $k'$
has the same canalyzing value and canalyzed value as line $k$, and
$f$ is properly $n$-symmetric, line $k'$ must occur in a lower layer
than that of line $k$.  Thus, there is at least one line between
line $k$ and line $k'$ for which the canalyzed value is $\overline{b_k}$.
Let line $q$ be the first such line.  Note that for all $i$ such
that $1 \leq i < q$, $c_{\pi(i)}$ does not match the canalyzing
value of line $i$, but $c_{\pi(q)}$ does match the canalyzing value
of line $q$.  Since canalyzed value $b_q$ equals $\overline{b_k}$,
we have that $f(c_{\pi(1)}, c_{\pi(2)}, \ldots, c_{\pi(n)}) =
\overline{b_k}$.

\medskip

\noindent
{\bf Case 2:} Line $k'$ has the form $x_{k'} : a_k ~\longrightarrow~ \overline{b_k}$.

\smallskip

Let line $q$ be the first line such that $k < q \leq k'$ and
the canalyzed value of line $q$ is $\overline{b_k}$.
Note that for all $i$ such that $1 \leq i < q$, 
$c_{\pi(i)}$ does not match the canalyzing value of line $i$,
so $f(c_{\pi(1)}, c_{\pi(2)}, \ldots, c_{\pi(n)}) = \overline{b_k}$.

\medskip

\noindent
{\bf Case 3:} Line $k'$ has the form $x_{k'} : \overline{a_k} ~\longrightarrow~ b_k$. 

\smallskip

Suppose there is a line below line $k$ for which the canalyzed
value is $\overline{b_k}$.  Let $q$ be the first such line.  Note
that for all $i$ such that $1 \leq i < q$, $c_{\pi(i)}$ does not
match the canalyzing value of line $i$, but $c_{\pi(q)}$ does match
the canalyzing value of line $q$.  Thus, $f(c_{\pi(1)}, c_{\pi(2)},
\ldots, c_{\pi(n)}) = \overline{b_k}$.

Now suppose there is no line below line $k$ for which the canalyzed
value is $\overline{b_k}$.  
Then line $k$ occurs in the last layer of $f$.  
Since $f$ is both default-normalized and properly $n$-symmetric,
this last layer contains exactly two lines.  Thus $k = n-1$ and $k'
= n$.  Note that for all $i$, $c_{\pi(i)}$ does not match the
canalyzing value of line $i$.  Thus, $f(c_{\pi(1)}, c_{\pi(2)},
\ldots, c_{\pi(n)}) = \overline{b_k}$.

\medskip

\noindent
{\bf Case 4:} Line $k'$ has the form $x_{k'} : \overline{a_k} ~\longrightarrow~ 
              \overline{b_k}$. 

\smallskip

Suppose there is a line, other than line $k'$, below line $k$ for
which the canalyzed value is $\overline{b_k}$.  Let $q$ be the
first such line.  Note that for all $i$ such that $1 \leq i < q$,
$c_{\pi(i)}$ does not match the canalyzing value of line $i$, but
$c_{\pi(q)}$ does match the canalyzing value of line $q$.  Thus,
$f(c_{\pi(1)}, c_{\pi(2)}, \ldots, c_{\pi(n)}) = \overline{b_k}$.

Now suppose that line $k'$ is the only line below line $k$ for which
the canalyzed value is $\overline{b_k}$.  Since line $k$ has
canalyzed value $b_k$, line $k'$ is the only line in its layer.
Since $f$ is default-normalized, there is a last layer following the layer
containing line $k'$.  Thus, for all $i$, $c_{\pi(i)}$ does not
match the canalyzing value of line $i$; therefore, 
$f(c_{\pi(1)}, c_{\pi(2)}, \ldots, c_{\pi(n)})$ ~=~ $\overline{b_k}$.  \QED

\medskip

\noindent
We now provide
examples of properly $n$-symmetric $n$-variable NCFs satisfying the
conditions of Theorem \ref{thm:ncf_r_symmetric}.
For each $n > 1$, let $f_n$ be the function of $n$ variables, namely
$x_1, \ldots, x_n$, defined by the following formula:

\smallskip

\hspace*{0.5in} $x_1 \vee (\overline{x}_2 \wedge ( x_3 \vee (\overline{x}_4 \wedge (\cdots ))$

\smallskip

\noindent
For instance,
\begin{align*}
f_6 &~=~  x_1 \vee (\overline{x}_2 \wedge (x_3 \vee (\overline{x}_4 \wedge 
      (x_5 \vee \overline{x}_6)) \\
f_7 &~=~  x_1 \vee (\overline{x}_2 \wedge (x_3 \vee (\overline{x}_4 \wedge (x_5 
          \vee (\overline{x}_6 \wedge  x_7)).
\end{align*}
Function $f_n(x_1, x_2, \ldots, x_n)$ is the NCF corresponding to
the following NCF representation:

\medskip

\noindent
\hspace*{0.25in}
$x_1:~ 1 ~\longrightarrow~ 1$ \\
\hspace*{0.25in}
$x_2:~ 1 ~\longrightarrow~ 0$ \\
\hspace*{0.25in}
$x_3:~ 1 ~\longrightarrow~ 1$ \\
\hspace*{0.75in}
$\vdots$ 

\smallskip

\noindent
If $n$ is odd, the last line is 

\smallskip

\noindent
\hspace*{0.25in}
$x_n:~ 1 ~\longrightarrow~ 1$ 

\smallskip

\noindent
and if $n$ is even, the last line is 

\smallskip

\noindent
\hspace*{0.25in}
$x_n:~ 1 ~\longrightarrow~ 0$. 

\smallskip

\noindent
Note that if $x_i = 0$ for all $i$, $1 \leq i \leq n$, then the
value of $f_n$ is 0 if $n$ is odd and 1 if $n$ is even.  Also, note
that the above representation is unnormalized. To normalize the
representation, the last line would be changed to have canalyzing
value 0, and the same canalyzed value as the preceding line.

\subsection{Number of Strongly Asymmetric NCFs}
\label{sse:number_strongly_asymmetric}

\begin{theorem}\label{thm:count_strongly_asymmetric}
For any $n \geq 2$, the number of Boolean functions with $n$ variables 
that are both strongly asymmetric and NCFs is $n! \, 2^{n-1}$.
\end{theorem}

\noindent
\textbf{Proof:}~
Consider a default-normalized representation of a strongly asymmetric NCF $f$.
The representation has $n-1$ layers.
The first $n-2$ layers each contain one line, and the last layer contains two lines.

Consider the canalyzed values in $f$. The canalyzed value in the
first layer can be either 0 or 1.  The canalyzed value in every
other layer is the complement of the canalyzed value in the preceding
layer.  Thus, there are 2 possibilities for the sequence of canalyzed
values in $f$.

The canalyzing variable in each of the first $n-2$ lines (one per
layer) can be any variable that has not yet occurred in a preceding
line.  The last layer contains the remaining two variables.  Thus,
there are $n! /[n-(n-2)]!$ = $n!/2$ possibilities for the pattern of 
canalyzing variable occurrences in $f$.

For each of the first $n-2$ variables, the canalyzing value for the line
containing that variable can be either 0 or 1.  
%% The last two lines, the canalyzed values on those two lines must be
%% the same because of the default-normalized representation. 
For the last two lines, the canalyzing values for the variables 
in those lines must be complements of each other; otherwise, by 
Theorem~\ref{thm:ncf_symmetric_variables}, those two variables 
will be symmetric.
Thus, the number of possible canalyzing values over all the $n$
lines is $2^{n-2} \times 2$ = $2^{n-1}$.

Hence, the number of Boolean functions with $n$ variables that
are both strongly asymmetric and NCFs is equal to \\ 
$2 \times (n!/2) \times 2^{n-1}$ ~=~ $n!\,2^{n-1}$.  \QED


\subsection{Symmetric Canalyzing and Nested Canalyzing Functions}
\label{sse:sym_and_cf_ncf}

\begin{proposition}\label{pro:ncf_symmetric}
(i) The only symmetric canalyzing functions are OR, AND, NOR, NAND, 
the constant function 0 and the constant function 1.
(ii) The only symmetric NCFs are OR, AND, NOR and NAND.
\end{proposition}
%%\medskip
\noindent
\textbf{Proof:}~

\smallskip

\noindent
\textbf{Part (i):}~
Suppose symmetric function $f$ is also a canalyzing function. 
Then there is a variable $x$, and values $a$ and $b$ 
such that whenever $x = a$, function $f$ has value $b$.  
Since $f$ is symmetric, $f$  has
value $b$ whenever any of its variables has value $a$.  
Thus, if at least one of its variables has value $a$, then $f$ has value $b$.  
If $f$ has value $\overline{b}$ when none of its variables has
value $a$, the four possible combinations of values for $a$ and $b$
correspond to the four functions OR, AND, NOR, and NAND.  If $f$
has value $b$ when none of its variables has value $a$, then $f$
is the constant function $b$.

\smallskip

\noindent
\textbf{Part (ii):}~
Suppose that symmetric function $f$ is also an NCF.  
Since constant functions are not NCFs,
$f$ has value $\overline{b}$ when none of its variables has value $a$.
The four possible combinations of values for $a$ and $b$ correspond
to the four functions OR, AND, NOR and NAND, each of which is an
NCF.  \QED








