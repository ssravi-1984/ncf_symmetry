\vspace*{-0.05in}
\begin{center}
\textbf{Abstract}~~ \textcolor{red}{\textbf{(To be revised)}}
\end{center}

\smallskip

A class of Boolean functions, 
called \textbf{nested canalyzing functions} (NCFs),
has been used to model certain biological phenomena.
Rearchers have studied some properties of these functions
such as sensitivity and stability.
In this note, we examine the relationships between NCFs and symmetric 
Boolean functions. 
We identify all NCFs that are also symmetric. 
A more general form of symmetric Boolean functions,
called $r$-symmetric functions (where $r$ is the symmetry level),
have also been studies in the literature.
We establish lower and upper bounds on the symmetry
level of an NCF $f$ as functions of the number of layers
in a standard respresentation of $f$.  


