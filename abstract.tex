\vspace*{-0.05in}
\begin{center}
\textbf{Abstract}  %% ~~ \textcolor{red}{\textbf{(To be revised)}}
\end{center}

\smallskip

A class of Boolean functions, 
called \textbf{nested canalyzing functions} (NCFs),
has been used to model certain biological phenomena.
Researchers have studied some properties of these functions
such as sensitivity and stability.
In this note, we examine the relationships between NCFs and symmetric 
Boolean functions. 
We identify all the Boolean functions that NCFs as well as symmetric. 
A more general form of symmetric Boolean functions,
called $r$-symmetric functions (where $r$ is the symmetry level),
has also been studied in the literature.
We observe that, in general,
one cannot efficiently approximate the symmetry level of
a Boolean function to within any factor, unless \textbf{P} = \cnp. 
In contrast, we show that the symmetry level of an NCF $f$
can be easily computed given a standard representation of $f$.
We also that for any NCF with $n$ variables, the notion of
strong asymmetry considered in the literature is equivalent to
the property that the symmetry level of the function is $n$.
