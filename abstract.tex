\vspace*{-0.05in}
\begin{center}
\textbf{Abstract}  %% ~~ \textcolor{red}{\textbf{(To be revised)}}
\end{center}

\smallskip

A class of Boolean functions, 
called \textbf{nested canalyzing functions} (NCFs),
has been used to model certain biological phenomena.
Researchers have studied some properties of these functions
such as sensitivity and stability.
In this note, we examine the relationships between NCFs and symmetric 
Boolean functions. 
We identify all the Boolean functions that both  NCFs and symmetric. 
A more general form of symmetric Boolean functions,
called $r$-symmetric functions (where $r$ is the symmetry level),
have also been studied in the literature.
%% We establish lower and upper bounds on the symmetry
%% level of an NCF $f$ as functions of the number of layers
%% in a standard respresentation of $f$.  
We show that, in general,
one cannot efficiently approximate the symmetry level of
a Boolean function to within any factor, unless \textbf{P} = \cnp. 
In contrast, we show that the symmetry level of an NCF $f$
can be easily computed from a standard (and efficiently
computable) representation of $f$.
