\vspace*{-0.05in}
\begin{center}
\textbf{Abstract}  %% ~~ \textcolor{red}{\textbf{(To be revised)}}
\end{center}

\smallskip

A class of Boolean functions, 
called \textbf{nested canalyzing functions} (NCFs),
has been used to model certain biological phenomena.
Researchers have studied some properties of these functions
such as sensitivity and stability.
In this note, we examine the relationships between NCFs, symmetric 
Boolean functions and a more generl form of symmetric Boolean functions,
called $r$-symmetric functions (where $r$ is the symmetry level).
Using a standard representation for NCFs, we develop a 
characterization of when two variables of an NCF are symmetric.
Using this characterization, we show 
that the symmetry level of an NCF $f$
can be easily computed given a standard representation of $f$.
(In general, we show that the symmetry level of
a Boolean function cannot be efficiently approximated
to within any factor, unless \textbf{P} = \cnp.)
We also show that for any NCF $f$ with $n$ variables, the notion of
strong asymmetry considered in the literature is equivalent to
the property that $f$ is $n$-symmetric. 
We use this result to derive a closed form expression for the
number of $n$-variable Boolean functions 
that are NCFs and strongly asymmetric.
In addition, we identify all the Boolean functions that are NCFs 
and symmetric. 
