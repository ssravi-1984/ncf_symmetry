\section{Introduction} 
\label{sec:intro}

\subsection{Canalyzing and Nested Canalyzing Functions}
\label{sse:ncf_def}

The class of \textbf{canalyzing} Boolean functions, introduced 
in \cite{Kauffman-1969}, is defined as follows.

\begin{definition}\label{def:canalyzing}
Given a set $X = $ $\{x_1, x_2, \ldots, x_n\}$ of $n$  Boolean variables,
a Boolean function $f(x_1, x_2, \ldots, x_n)$ over $X$ is \textbf{canalyzing}
if there is a variable $x_i \in X$ and values $a$, $b$ $\in \{0,1\}$ such that
\[
f(x_1, \ldots, x_{i-1}, a, x_{i+1}, \ldots, x_n) ~=~ b, 
\]
regardless of the values assigned to the variables in $X - \{x_i\}$.
\end{definition}

\medskip

\noindent
\textbf{Example 1:}~ Consider the function 
$f(x_1, x_2, x_3) ~=~ \overline{x_1} \wedge (x_2 \vee \overline{x_3})$.
This function is canalyzing since if we set to $x_1 = 1$,
$f(1, x_2, x_3) ~=~ 0$, for all combinations of values for 
$x_2$ and $x_3$. \qed

\medskip

Another class of Boolean functions, called \textbf{nested canalyzing functions} (NCFs),
was introduced later in \cite{Kauffman-etal-2003} to carry out a detailed
analysis of the behavior of certain biological systems.
We follow the presentation in \cite{Layne-2011} in defining NCFs.
(For a Boolean value $b$,~ the complement is denoted by $\overline{b}$.)

\begin{definition}\label{def:nested_canalyzing}
Let $X = $ $\{x_1, x_2, \ldots, x_n\}$ denote a set of $n$  Boolean variables.
Let $\pi$ be a permutation of $\{1, 2, \ldots, n\}$.
A Boolean function $f(x_1, x_2, \ldots, x_n)$ over $X$ is \textbf{nested canalyzing}
in the variable order $x_{\pi(1)}, x_{\pi(2)}, \ldots, x_{\pi(n)}$ with
\textbf{canalyzing values} $a_1, a_2, \ldots, a_n$ and 
\textbf{canalyzed values} $b_1, b_2, \ldots, b_n$ 
if $f$ can be expressed in the following form:
\[
f(x_1, x_2, \ldots, x_n) ~=~ 
   \begin{cases}
       \:b_1 & \mathrm{if~~} x_{\pi(1)} ~=~ a_1 \\
       \:b_2 & \mathrm{if~~} x_{\pi(1)} ~\neq~ a_1 \mathrm{~~and~~}
            x_{\pi(2)} ~=~ a_2 \\
       \:\vdots & \vdots \\
       \:b_n & \mathrm{if~~} x_{\pi(1)} ~\neq~ a_1 \mathrm{~~and~~} \ldots~
             x_{\pi(n-1)} ~\neq~ a_{n-1} \mathrm{~~and~~} x_{\pi(n)} ~=~ a_n \\
       \:\overline{b_n} & \mathrm{if~~} x_{\pi(1)} ~\neq~ a_1 \mathrm{~~and~~} \ldots~
            x_{\pi(n)} ~\neq~ a_n \\
   \end{cases}
\]
\end{definition}
%% \clearpage
For convenience, we will use a notation introduced in \cite{Stearns-etal-2018}
to represent NCFs.
For $1 \leq i \leq n$, line $i$ of this representation has the form

\medskip

\noindent
\hspace*{1.1in} $x_{\pi(i)}:~ a_i ~~\longrightarrow~~ b_i$

\medskip

\noindent 
with $x_{\pi(i)}$ being the \textbf{canalyzing variable} that is
\textbf{tested} in line $i$, 
and $a_i$ and $b_i$ being respectively the canalyzing and 
canalyzed values in line $i$,~ $1 \leq i \leq n$.
Each such line is called a \textbf{rule}.
When none of the conditions ``$x_{\pi(i)} ~=~ a_i$" 
is satisfied, we have line $n+1$ with the ``Default" rule
for which the canalyzed value is~ $\overline{b_n}$: 

\medskip

\noindent
\hspace*{1.1in} Default:~ $\overline{b_n}$

\medskip
\noindent
As in \cite{Stearns-etal-2018}, we will refer to the above specification
of an NCF as the \textbf{simplified representation} and assume
(without loss of generality) that each NCF is specified in this manner.
%% The simplified representation provides the following 
%% convenient computational view of an NCF.~
%% Lines defining an NCF are 
%% considered sequentially in a top-down manner.
%% The computation stops at the first line where the 
%% specified condition is satisfied, and the value of the function
%% is the canalyzed value on that line. 
We now present an example of an NCF using the two representations
mentioned above.


\medskip
\noindent
\textbf{Example 2:}~ Consider the function 
$f(x_1, x_2, x_3) ~=~ \overline{x_1} \wedge (x_2 \vee \overline{x_3})$
used in Example~1.
This function is nested canalyzing using the identity permutation $\pi$ on $\{1,2,3\}$
with canalyzing values $1,1,0$ and canalyzed values $0, 1, 1$.
We first show how this function can be expressed using the syntax of
Definition~\ref{def:nested_canalyzing}.

\[
f(x_1, x_2, x_3) ~=~ 
   \begin{cases}
       \:0 & \mathrm{if~~} x_{1} ~=~ 1 \\
       \:1 & \mathrm{if~~} x_{1} ~\neq~ 1 \mathrm{~~and~~}
            x_{2} ~=~ 1 \\
       \:1 & \mathrm{if~~} x_{1} ~\neq~ 1 \mathrm{~~and~~}
            x_{2} ~\neq~ 1 \mathrm{~~and~~} x_{3} = 0 \\
       \:0 & \mathrm{if~~} x_{1} ~\neq~ 1 \mathrm{~~and~~}
            x_{2} ~\neq~ 1 \mathrm{~~and~~} x_{3} ~\neq~ 0 \\
   \end{cases}
\]

\medskip
\noindent
A simplified representation of the same function is as follows.

\bigskip

\noindent
\begin{tabular}{ll}
\hspace*{1.1in} & $x_1:~$  $1 ~\longrightarrow~ 0$ \\ [1ex]
\hspace*{1.1in} & $x_2:~$  $1 ~\longrightarrow~ 1$ \\ [1ex]
\hspace*{1.1in} & $x_3:~$  $0 ~\longrightarrow~ 1$ \\ [1ex]
\hspace*{1.1in} & Default:~ $0$ \\
\end{tabular}

\noindent
\qed

\noindent
We will also consider a more general form of NCFs defined below.

\begin{definition}\label{def:generalized ncf}
A {\bf generalized NCF} is a function represented as either a constant
or an NCF representation of a subset (not necessarily proper) 
of the function's variables.
\end{definition}

\noindent
\textbf{Example 3:}~ For $b \in \{0,1\}$, the constant function which takes 
on the value $b$ for every input is a generalized NCF since 
it can be represented by the following rule: 

\smallskip

\hspace*{1.1in} Default:~ $b$ 

\smallskip

\noindent
The following generalized NCF specification for 
a function $f(x_1, x_2, x_3, x_4)$ indicates that the function
depends only on variables $x_1$ and $x_3$.

\medskip

\noindent
\begin{tabular}{ll}
\hspace*{1.1in} & $x_1:~$  $0 ~\longrightarrow~ 1$ \\ [1ex]
\hspace*{1.1in} & $x_3:~$  $1 ~\longrightarrow~ 0$ \\ [1ex]
\hspace*{1.1in} & Default:~ $1$ \\
\end{tabular}

\noindent
\qed

%% \bigskip

%% \noindent
%% \textbf{Note:} If a generalized NCF is not a constant, 
%% there is an assignment to its variables
%% such that the value of the function is 0,
%% and there is an assignment to its variables
%% such that the value of the function is 1.

\subsection{Symmetric Boolean Functions}
\label{sse:symmetry}

A pair of variables of a Boolean function $f(x_1, x_2, \ldots, x_n)$ is
said to be \textbf{symmetric} if their values can be interchanged without
affecting the value of the function.
As a simple example, each pair of variables in the function 
$f_2(x_1, x_2, x_3)$ = $x_1 \oplus x_2 \oplus x_3$ is symmetric.
This form of interchange symmetry partitions the set of variables into a set of
\textbf{symmetry groups}, where the members of each group are pairwise symmetric.
A Boolean function $f$ is 
$r$-\textbf{symmetric} if it has at most $r$ symmetry groups.
In this case, the value of $f$
depends only on how many of the variables in each symmetry group have the value 1.
We say that $f$ is \textbf{properly} $r$-\textbf{symmetric} if
it is $r$-symmetric, but not $(r-1)$-symmetric.
For example, the function $f_3(x_1, x_2, x_3) = (x_1 \wedge x_2) \vee\, \overline{x_3}$
is not 1-symmetric since $f_3(1, 0, 1) \neq f_3(1, 1, 0)$; however, 
it is 2-symmetric with the symmetric groups being $\{x_1, x_2\}$ and $\{x_3\}$.
For a Boolean function $f$, the integer $r$ such that $f$ is properly $r$-symmetric
will be referred to as the \textbf{symmetry level} of $f$.
Thus, the symmetry level of $f$ is the smallest integer $r$ such that
$f$ is $r$-symmetric.

In the literature (see, e.g., \cite{Crama-Hammer-2011,HT-2016,Toth-etal-1977}),
a 1-symmetric function $f$ is referred to simply
as a \textbf{symmetric} function.
Thus, in any symmetric function, each pair of variables is symmetric.
As a simple example, the function $f_2(x_1, x_2, x_3)$ = 
$x_1 \oplus x_2 \oplus x_3$ (defined above) is symmetric.
If $f$ is a symmetric function, then for any
input $(a_1, a_2, \ldots, a_n)$ to $f$, where $a_i \in \{0,1\}$ for
$1 \leq i \leq n$, and any permutation $\pi$ of $\{1, 2, \ldots, n\}$,
$f(a_1, a_2, \ldots, a_n)$ = $f(a_{\pi(1)}, a_{\pi(2)}, \ldots, a_{\pi(n)})$.
The class of $r$-symmetric functions have also been
studied in the literature on discrete dynamical systems (see, e.g., 
\cite{Barrett-etal-2007,Rosenkrantz-etal-2015,MR-2007}).

Other forms of symmetry which can be more general than the permutations 
corresponding to symmetry groups have been considered \cite{KS-2000}.
A Boolean function $f(x_1, x_2, \ldots, x_n)$
is \textbf{strongly asymmetric}
if for any permutation $\pi$ of $\{1, 2, \ldots, n\}$
\emph{except} the identity permutation,
there exists an input $(a_1, a_2, \ldots,  a_n)$
to $f$ such that $f(a_1, a_2, \ldots, a_n)$ $\neq$
$f(a_{\pi(1)}, a_{\pi(2)}, \ldots, a_{\pi(n)})$.
In general, there are functions with $n$ variables that are properly
$n$-symmetric, but not strongly asymmetric \cite{KS-2000}.
However, in the case of NCFs with $n$ variables, we will show 
(see Section~\ref{sec:ncf_and_symmetry}) that the notion of 
strong asymmetry coincides with that 
of being properly $n$-symmetric.

%\noindent

\subsection{Summary of Results}
\label{sse:contrib}

In this note, our focus is on the relationship between 
NCFs and symmetric Boolean functions.
We observe that, in general,
one cannot efficiently approximate the symmetry level of
a Boolean function to within any factor, unless \textbf{P} = \cnp.
In contrast, we show that the symmetry level of an NCF $f$
can be easily computed given a standard representation of $f$.
We also show that for any NCF $f$ with $n$ variables, the notion of
strong asymmetry considered in the literature is equivalent to
the property that the symmetry level of $f$ is $n$.
Finally, we identify all the Boolean functions that NCFs 
as well as symmetric.

\subsection{Related Work}
\label{sse:related}

The term \textbf{canalization}, coined by
Waddington \cite{Waddington-1942}, is generally used to describe
the stability of a biological system with changes
in external conditions.
In 1969, Kauffman \cite{Kauffman-1969} introduced canalyzing Boolean functions
to explain the stability of gene regulatory networks.
The subclass of NCFs was
introduced later by Kauffman et al. \cite{Kauffman-etal-2003} 
to facilitate a rigorous analysis of the Boolean network model
for gene regulatory networks.
It is known that the class of NCFs coincides with that
of unate cascade Boolean functions \cite{Jarrah-etal-2007}.
In the literature on computational learning theory,
generalized NCFs are referred to as $1$-\textbf{decision lists} \cite{KV-1994}.
Many researchers have pointed out the usefulness of NCFs 
in modeling biological phenomena 
(e.g., \cite{Layne-2011,
Layne-etal-2012,Li-etal-2011,Li-etal-2012,Li-etal-2013}).
Properties of NCFs such as sensitivity and 
stability have also been studied in the 
literature \cite{Kauffman-etal-2004, Layne-2011,Layne-etal-2012,
Li-etal-2011,Li-etal-2013,Klotz-etal-2013, Stearns-etal-2018}. 
To our knowledge, the relationship between NCFs and symmetric
Boolean functions has not been addressed in the literature.
