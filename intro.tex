\section{Introduction} 
\label{sec:intro}

\subsection{Canalyzing and Nested Canalyzing Functions}
\label{sse:ncf_def}


\begin{definition}\label{def:canalyzing}
Let $X = $ $\{x_1, x_2, \ldots, x_n\}$ be a set of $n$  Boolean variables.
A Boolean function $f(x_1, x_2, \ldots, x_n)$ over $X$ is \textbf{canalyzing}
if there is a variable $x_i \in X$ and values $a$, $b$ $\in \{0,1\}$ such that 
$f(x_1, \ldots, x_{i-1}, a, x_{i+1}, \ldots, x_n)$ $~=~ b$, regardless of the values
assigned to the variables in $X - \{x_i\}$.
\end{definition}

\medskip

\noindent
\textbf{Example 1:}~ Consider the function 
$f(x_1, x_2, x_3) ~=~ x_1 \vee (x_2 \wedge x_3)$.
This function is canalyzing since if we set to $x_1 = 1$,
$f(1, x_2, x_3) ~=~ 1$, regardless of the values assigned to $x_2$ and $x_3$. \qed

\medskip

A class of Boolean functions, called \textbf{nested canalyzing functions} (NCFs),
was introduced in \cite{Kauffman-etal-2003} to model the behavior of
certain biological systems.
We follow the presentation in \cite{Layne-2011} in defining 
such a Boolean function. 
(For a Boolean value $b$,~ the complement is denoted by $\overline{b}$.)

\begin{definition}\label{def:nested_canalyzing}
Let $X = $ $\{x_1, x_2, \ldots, x_n\}$ denote a set of $n$  Boolean variables.
Let $\pi$ be a permutation of $\{1, 2, \ldots, n\}$.
A Boolean function $f(x_1, x_2, \ldots, x_n)$ over $X$ is \textbf{nested canalyzing}
in the variable order $x_{\pi(1)}, x_{\pi(2)}, \ldots, x_{\pi(n)}$ with
\textbf{canalyzing values} $a_1, a_2, \ldots, a_n$ and 
\textbf{canalyzed values} $b_1, b_2, \ldots, b_n$ 
if $f$ can be expressed in the following form:
\[
f(x_1, x_2, \ldots, x_n) ~=~ 
   \begin{cases}
       \:b_1 & \mathrm{if~~} x_{\pi(1)} ~=~ a_1 \\
       \:b_2 & \mathrm{if~~} x_{\pi(1)} ~\neq~ a_1 \mathrm{~~and~~}
            x_{\pi(2)} ~=~ a_2 \\
       \:\vdots & \vdots \\
       \:b_n & \mathrm{if~~} x_{\pi(1)} ~\neq~ a_1 \mathrm{~~and~~} \ldots~
             x_{\pi(n-1)} ~\neq~ a_{n-1} \mathrm{~~and~~} x_{\pi(n)} ~=~ a_n \\
       \:\overline{b_n} & \mathrm{if~~} x_{\pi(1)} ~\neq~ a_1 \mathrm{~~and~~} \ldots~
            x_{\pi(n)} ~\neq~ a_n \\
   \end{cases}
\]
\end{definition}

\medskip
%% \clearpage
For convenience, we will use a computational notation to represent NCFs.
For $1 \leq i \leq n$, line $i$ of our representation has the following form:

\medskip

\noindent
\hspace*{1.1in} $x_{\pi(i)}:~ a_i ~~\longrightarrow~~ b_i$

\medskip

\noindent 
We say that $x_{\pi(i)}$ is the \textbf{canalyzing variable} that is
\textbf{tested} in line $i$, 
with $a_i$ and $b_i$ denoting respectively the canalyzing and 
canalyzed values in line $i$ as before,~ $1 \leq i \leq n$.
We refer to each such line as a \textbf{rule}.
When none of the conditions ``$x_{\pi(i)} ~=~ a_i$" 
is satisfied, we have line $n+1$ with the ``Default" rule
for which the canalyzed value is~ $\overline{b_n}$: 

\medskip

\noindent
\hspace*{1.1in} Default:~ $\overline{b_n}$

\medskip
\noindent
In the remainder of this paper, we will refer to the above specification
of an NCF as the \textbf{simplified representation} and assume
(without loss of generality) that each NCF is specified in this manner.
The simplified representation provides the following 
convenient computational view of an NCF.~
Lines defining an NCF are 
considered sequentially in a top-down manner.
The computation stops at the first line where the 
specified condition is satisfied, and the value of the function
is the canalyzed value on that line. 
We now present an example of an NCF using the two representations
mentioned above.


\medskip
\noindent
\textbf{Example 2:}~ Consider the function 
$f(x_1, x_2, x_3) ~=~ x_1 \vee (x_2 \wedge x_3)$.
This function is nested canalyzing using the identity permutation $\pi$ on $\{1,2,3\}$
with canalyzing values $1,0,0$ and canalyzed values $1, 0, 0$.
We first show how this function can be expressed using the syntax of
Definition~\ref{def:nested_canalyzing}.

\[
f(x_1, x_2, x_3) ~=~ 
   \begin{cases}
       \:1 & \mathrm{if~~} x_{1} ~=~ 1 \\
       \:0 & \mathrm{if~~} x_{1} ~\neq~ 1 \mathrm{~~and~~}
            x_{2} ~=~ 0 \\
       \:0 & \mathrm{if~~} x_{1} ~\neq~ 1 \mathrm{~~and~~}
            x_{2} ~\neq~ 0 \mathrm{~~and~~} x_{3} = 0 \\
       \:1 & \mathrm{if~~} x_{1} ~\neq~ 1 \mathrm{~~and~~}
            x_{2} ~\neq~ 0 \mathrm{~~and~~} x_{3} ~\neq~ 0 \\
   \end{cases}
\]

\medskip
\noindent
A simplified representation of the same function is as follows.

\bigskip

\noindent
\begin{tabular}{ll}
\hspace*{1.1in} & $x_1:~$  $1 ~\longrightarrow~ 1$ \\ [1ex]
\hspace*{1.1in} & $x_2:~$  $0 ~\longrightarrow~ 0$ \\ [1ex]
\hspace*{1.1in} & $x_3:~$  $0 ~\longrightarrow~ 0$ \\ [1ex]
\hspace*{1.1in} & Default:~ $1$ \\
\end{tabular}

\noindent
\qed


\noindent
\underline{\textsf{Generalized NCFs:}}~
For some problems, we will also consider a more general form
of NCFs defined below.

\begin{definition}\label{def:generalized ncf}
A {\bf generalized NCF} is a function represented as either a constant
or an NCF representation of a subset (not necessarily proper) 
of the function's variables.
\end{definition}

\noindent
\textbf{Example 3:}~ For $b \in \{0,1\}$, the constant function which takes 
on the value $b$ for every input is a generalized NCF since 
it can be represented by the following rule: 

\smallskip

\hspace*{1.1in} Default:~ $b$ 

\smallskip

\noindent
The following generalized NCF specification for 
a function $f(x_1, x_2, x_3, x_4)$ indicates that the function
depends only on variables $x_1$ and $x_3$.

\medskip

\noindent
\begin{tabular}{ll}
\hspace*{1.1in} & $x_1:~$  $0 ~\longrightarrow~ 1$ \\ [1ex]
\hspace*{1.1in} & $x_3:~$  $1 ~\longrightarrow~ 0$ \\ [1ex]
\hspace*{1.1in} & Default:~ $1$ \\
\end{tabular}

\noindent
\qed

\bigskip

\noindent
\textbf{Note:} If a generalized NCF is not a constant, 
there is an assignment to its variables
such that the value of the function is 0,
and there is an assignment to its variables
such that the value of the function is 1.



%\noindent

\subsection{Summary of Results}
\label{sse:contrib}

Our focus is on the relationship between NCFs and symmetric
Boolean functions.


\subsection{Related Work}


Many researchers have studied mathematical properties of NCFs and
pointed out the importance of NCFs in modeling biological
phenomena (e.g., \cite{Kauffman-etal-2003,Kauffman-etal-2004,Layne-2011,
Layne-etal-2012,Li-etal-2011,Li-etal-2012,Li-etal-2013}).

\medskip

\noindent
\textbf{Equivalence and Implication Problems.}~
We note that for general Boolean formulas, the implication and equivalence problems
are \cnp-hard \cite{GJ-1979}.

\bigskip

\noindent
\textbf{Sensitivity of Boolean Functions.}~
There is extensive literature on the sensitivity of various classes
of Boolean functions
(e.g., \cite{Buhrman-etal-2002,Nisan-etal-1994,Zhang-2011}).
For a discussion on how the sensitivity of a Boolean function provides
an indication of its stability, the reader is referred to \cite{Kauffman-etal-2004,
Layne-2011,Layne-etal-2012}.
Observations regarding relationships between sensitivities of Boolean functions
and their computational complexity are presented
in \cite{Buhrman-etal-2002,Nisan-etal-1994}.
Li and Adeyeye \cite{Li-etal-2012} present lower and upper bounds on the
sensitivity of any NCF.
Li et al. \cite{Li-etal-2011,Li-etal-2013} conjectured that the average
sensitivity of any NCF is strictly less than  $4/3$.
This conjecture was proved by Klotz et al. \cite{Klotz-etal-2013} by establishing
a tight upper bound on the average sensitivity of any NCF.
Their methods rely on Fourier analysis of Boolean functions
\cite{Odonnell-2014}.
In \cite{Stearns-etal-2018}, 
we presented a characterization of NCFs with maximum average sensitivity.
This characterization allowed us to present an alternative proof 
of the upper bound in \cite{Klotz-etal-2013} and derive a closed
form expression for the number of such NCFs.
