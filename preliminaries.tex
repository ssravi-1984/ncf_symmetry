\section{Other Definitions and Preliminary Results}
\label{sec:prelim}

\subsection{Overview}

Here, we present some definitions and preliminary results which enable
us to define a normalized representation for NCFs.
This representation plays an important role in several
problems considered in later sections.

\medskip

Recall that any NCF $f$ with $n$ variables,
denoted by $x_1$, $x_2$, $\ldots$, $x_n$,~
is specified using a variable ordering 
$x_{\pi(1)}$, $x_{\pi(2)}$, $\ldots$,  $x_{\pi(n)}$,
where $\pi$ is a permutation of $\{1, 2, \ldots, n\}$.
We will refer to this as the \textbf{lexicographic} ordering
of the variables in the specification.

\medskip

Throughout this section, we assume that an NCF $f$ with $n$
variables is specified using the simplified representation with $n$ lines. 
Unless otherwise mentioned, the ``Default" line (which is assigned the number $n+1$)
in the simplified representation of an NCF is \emph{not} considered
in the results stated in this section.

\subsection{Layers and Normalized Representation}
\label{sse:ncf_layer}

We begin with a two-part observation from \cite{Stearns-etal-2018}.

\begin{observation} \label{obs:ncf_transformations}
Let $f$ be an NCF with $n$ variables specified using $n$ lines
in the simplified representation. 
\begin{description}
\item{(a)} 
For any $q \geq 2$ and for any $i$, $1 \leq i \leq n-q+1$,
if the $q$ consecutive lines $i$, $i+1$, $\ldots$, $i+q-1$~
have the same canalyzed value, then the function remains
unchanged if these $q$ lines are permuted in any order
without changing the other lines.

\item{(b)} 
Suppose lines $n-1$ and $n$ in the specification of $f$ 
have complementary canalyzed outputs.
Then, the function remains unchanged 
if the canalyzing value and canalyzed output in line $n$
are both complemented. 
(Here, the value on the ``Default" line is also complemented.)
\QED
\end{description}
\end{observation}

\noindent
Li et al. \cite{Li-etal-2013} have defined the concept of a 
{\bf layer} of an NCF in terms of
an algebraic representation of the NCF as an extended monomial.
For our purposes, we use the following definition based on the
simplified representation of NCFs given in Section~\ref{sec:intro}.

\begin{definition}
\label{def:layer}
A {\bf layer} of a NCF representation is a maximal length sequence 
of lines with the same canalyzed output.
\end{definition}
%%Note that the ``Default" line in the representation of an NCF is
%%\emph{not} considered in the definition of a layer.

\noindent
\textbf{Example 4:}~ Consider the following function $f$ of six variables
$x_1$, $x_2$, $\ldots$, $x_6$. 

\bigskip

\noindent
\begin{tabular}{ll}
\hspace*{1.1in} & $x_1:~$  $1 ~\longrightarrow~ 1$ \\ [0.5ex]
\hspace*{1.1in} & $x_2:~$  $0 ~\longrightarrow~ 1$ \\ [0.5ex]
\hspace*{1.1in} & $x_3:~$  $0 ~\longrightarrow~ 0$ \\ [0.5ex]
\hspace*{1.1in} & $x_4:~$  $1 ~\longrightarrow~ 0$ \\ [0.5ex]
\hspace*{1.1in} & $x_5:~$  $1 ~\longrightarrow~ 1$ \\ [0.5ex]
\hspace*{1.1in} & $x_6:~$  $1 ~\longrightarrow~ 0$ \\ [0.5ex]
\hspace*{1.1in} & Default:~ $1$ \\
\end{tabular}

\medskip
\noindent
This function has 4 layers, with 
Layer~1 consisting of the lines with $x_1$ and $x_2$, 
Layer~2 consisting of the lines with $x_3$ and $x_4$, 
Layer~3 consisting of the line with $x_5$ and 
Layer~4 consisting of the line with $x_6$. \qed

\medskip
From Part~(a) of Observation~\ref{obs:ncf_transformations},
it follows that the lines within the same layer of an NCF
representation can be permuted without changing the function.
Part~(b) of Observation~\ref{obs:ncf_transformations} points
out that for any NCF $f$ with $n \geq 2$ variables, there is a simplified 
representation in which lines $n-1$ and $n$ are in the same layer.

%% It can be verified that the number of layers in the 
%% extended monomial formula for a NCF (as defined in \cite{Li-etal-2013})
%% is the same as the number of layers in the normalized representation.

\medskip
We can now define the notion of a \textbf{normalized} 
representation for NCFs.

\begin{definition}
\label{def:normalized}
Let $f$ be an NCF with $n \geq 2$ variables specified using
the simplified representation with 
lexicographic ordering given by  
$\langle x_{\pi(1)}, x_{\pi(2)}, x_{\pi(n)}\rangle$.
We say that the representation is {\bf normalized} if
both of the following conditions hold.
\begin{enumerate}
  \item Lines $n-1$ and $n$ have the same canalyzed output ~and
  \item each pair of consecutive lines
        with the same canalyzed output occur in the lexicographic order 
        of their canalyzing variables; that is, if two consecutive lines are \\
%\medskip
\noindent
\hspace*{1.1in} $x_i$:~ $a_i \longrightarrow b$ \\
\hspace*{1.1in} $x_j$:~ $a_j \longrightarrow b$

%\medskip
\noindent
then $x_i$ precedes $x_j$ in the lexicographic ordering.
\end{enumerate}
\end{definition}

\noindent
\textbf{Example 5:}~ 

\medskip
\noindent
\textbf{Part 1:}~
Consider the NCF $f$ shown in Example~4 with
the variable ordering $x_1$, $x_2$, $x_3$, $x_4$, $x_5$, $x_6$.
This representation is \emph{not} normalized since lines 5 and 6
have different canalyzed values.
If we change line 6 to ``$x_6$:~ 0 $\longrightarrow$ 1" and the
value on the ``Default" line to 0, then we obtain 
a normalized representation. 

\medskip
\noindent
\textbf{Part 2:}~ Consider the normalized version of Example~4 
produced in Part~1 above. 
In that representation, if we exchange Lines 1 and 2, 
by Part~(a) of Observation~\ref{obs:ncf_transformations},
the function does not change. 
However, the resulting representation is \emph{not} normalized
since the line with $x_2$ precedes the one 
with $x_1$. \qed

\medskip
From Observation \ref{obs:ncf_transformations},
we note that in polynomial time, for any NCF representation with more than one line,
we can first modify the last line if necessary so that it has the same canalyzed output as the preceding line,
and then we can lexicographically sort
each maximal sequence of consecutive lines with the same canalyzed output,
thereby obtaining an equivalent normalized NCF representation.
We state this formally below.

\begin{observation}\label{obs:normalization_poly}
Given an NCF representation for a function $f$, a normalized
representation for $f$ can be obtained in polynomial time. \QED
\end{observation}

\subsection{Closure Properties of NCFs}

\begin{observation}\label{complement}
The set of NCFs and the set of generalized NCFs
are each closed under complement.
Moreover, given an NCF or a generalized NCF representation $f$, 
an NCF or generalized NCF $f'$ representing the complement of $f$ 
can be constructed in linear time.
\end{observation}

\noindent
\textbf{Proof:}~ We construct
representation $f'$ by complementing 
every canalyzed output in representation $f$.  \QED

\medskip
We also observe that the set of NCFs is not closed under OR,~ XOR~ or AND.
To see this for OR and XOR,  note that the functions 
$f_1 = x_1 \, \bar{x}_2$ and $f_2 = \bar{x}_1 \, x_2$
are each NCFs, but their OR and XOR is the same Boolean function,
and this Boolean function is not  a canalyzing function of either variable.
To see this for AND, note that the functions 
$f_1 = x_1 \vee \bar{x}_2$ and $f_2 = \bar{x}_1 \vee x_2$
are each NCFs, but their AND is not  a canalyzing function of either variable.
We now define a new operation (projection) on Boolean functions 
and show through an explicit construction that
NCFs are closed under this operation.
This construction is used in Section~\ref{sse:gen_ncf_implication}
where we present an efficient algorithm for the implication problem
for generalized NCFs. 

\begin{definition}\label{def:projection}
Given a Boolean function $f$, a variable $x$, and a Boolean value $a$,
the {\bf projection} of $f$ on $x = a$,
denoted by $f_{x = a}$, is the function on the remaining variables
whose value on any assignment $\alpha$ to these variables
is the value of $f$ when $\alpha$ is extended to a complete assignment for $f$
by setting the variable $x$ to the value $a$.
\end{definition}

\noindent
We now show that generalized NCFs are closed under projection.

\begin{lemma}\label{lem:projection}
Given a generalized NCF representation $f$, a variable $x$, and a
Boolean value $a$, a generalized NCF representing the projection
$f_{x = a}$ can be constructed in linear time.
\end{lemma}

\noindent
\textbf{Proof:}~ We use a simple case analysis.
We will assume that the layers of the representation of $f$ have
been identified and that they are numbered 
from top to bottom, starting with 1.

\smallskip

\noindent
{\bf Case 1}:~ Variable $x$ does not occur in $f$.
Then  $f_{x = a}$ is the same as $f$.

\smallskip

\noindent
{\bf Case 2}:~ For some value $b$, $f$ contains the line~ 
``$x:~ a \longrightarrow b$".~
We have two subcases.
\begin{description}
\item {(i)} If this line occurs in the first layer of $f$, 
then $f_{x = a}$ is the constant $b$.

\item{(ii)} If this line occurs in a layer of $f$ other than the first layer, 
$f_{x = a}$ is obtained from $f$ by deleting 
the layer containing this line and all subsequent layers.
(Note that in this case, the preceding layer has canalyzed output $\bar{b}$.)
\end{description}

%%\smallskip

\noindent
{\bf Case 3}:~ For some value $b$, $f$ contains the line~
``$x:~ \bar{a} \longrightarrow b$".~
We have three subcases here.
\begin{description}
\item{(i)}
If this line occurs in a layer with more than one line, 
or if this line occurs in a layer that is not the last layer,
then $f_{x = a}$ is obtained from $f$ by deleting this line.
\item{(ii)}
If this line is the last line in $f$, 
and there is no other line in $f$ whose canalyzed output is $b$,
then $f_{x = a}$ is  the constant $\bar{b}$.
\item{(iii)}
If this line is the last line in $f$, is the only line is its layer, 
and there is at least one other line in $f$ whose canalyzed output is $b$,
then $f_{x = a}$ is obtained from $f$ by deleting the last two layers.
\end{description}
\QED
