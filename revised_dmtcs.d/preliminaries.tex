\section{Other Definitions and Preliminary Results}
\label{sec:prelim}

\subsection{Overview}

We present some definitions and recall
a known result regarding NCFs which allow us
to define a normalized representation for NCFs.
This representation plays an important role in several
problems considered in later sections.
We also show (Theorem~\ref{thm:approx_sym_level_hard}) 
that, in general, the symmetry level
of a Boolean function cannot be efficiently approximated to within any
factor $\rho \geq 1$, unless \textbf{P} = \cnp.

\medskip

\subsection{Layers and Normalized Representation}
\label{sse:ncf_layer}

Recall that any NCF $f$ with $n$ variables,
denoted by $x_1$, $x_2$, $\ldots$, $x_n$,~
is specified using a variable ordering 
$x_{\pi(1)}$, $x_{\pi(2)}$, $\ldots$,  $x_{\pi(n)}$,
where $\pi$ is a permutation of $\{1, 2, \ldots, n\}$.
Throughout this paper, we will assume without loss of generality
that the variable ordering uses the identity permutation 
so that $x_i$ is the variable tested in line $i$, $1 \leq i \leq n$. 
We assume that an NCF $f$ with $n$
variables is specified using the simplified representation with $n$ lines. 
Unless otherwise mentioned, the ``Default" line (which is assigned the number $n+1$)
in the simplified representation of an NCF is \emph{not} considered
in the results stated in this section.
We will use the following result from \cite{Stearns-etal-2018}.

\begin{observation} \label{obs:ncf_transformations}
Let $f$ be an NCF with $n$ variables specified using $n$ lines
in the simplified representation. 
%\begin{description}
\begin{enumerate}
\item 
For any $q \geq 2$ and for any $i$, $1 \leq i \leq n-q+1$,
if the $q$ consecutive lines $i$, $i+1$, $\ldots$, $i+q-1$~
have the same canalyzed value, then the function remains
unchanged if these $q$ lines are permuted in any order
without changing the other lines.

\item 
Suppose lines $n-1$ and $n$ in the specification of $f$ 
have complementary canalyzed values.
Then, the function remains unchanged 
if the canalyzing value and canalyzed value in line $n$
are both complemented. 
(Here, the value on the ``Default" line is also complemented.)
\QED
\end{enumerate}
%\end{description}
\end{observation}

%%\noindent
\citet{Li-etal-2013} defined the concept of a 
{\bf layer} of an NCF in terms of
an algebraic representation of the NCF as an extended monomial.
For our purposes, we use the following definition based on the
simplified representation of NCFs.

\begin{definition}
\label{def:layer}
A {\bf layer} of an NCF representation is a maximal length sequence 
of lines with the same canalyzed value.
\end{definition}
%%Note that the ``Default" line in the representation of an NCF is
%%\emph{not} considered in the definition of a layer.

\noindent
\textbf{Example 4:}~ Consider the following function $f$ of six variables
$x_1$, $x_2$, $\ldots$, $x_6$. 

\bigskip

\noindent
\begin{tabular}{ll}
\hspace*{0.25in} & $x_1:~$  $1 ~\longrightarrow~ 0$ \\ [0.5ex]
\hspace*{0.25in} & $x_2:~$  $0 ~\longrightarrow~ 0$ \\ [0.5ex]
\hspace*{0.25in} & $x_3:~$  $0 ~\longrightarrow~ 0$ \\ [0.5ex]
\hspace*{0.25in} & $x_4:~$  $1 ~\longrightarrow~ 1$ \\ [0.5ex]
\hspace*{0.25in} & $x_5:~$  $1 ~\longrightarrow~ 1$ \\ [0.5ex]
\hspace*{0.25in} & $x_6:~$  $1 ~\longrightarrow~ 0$ \\ [0.5ex]
\hspace*{0.25in} & Default:~ $1$ \\
\end{tabular}

\medskip
\noindent
This function has 3 layers, with 
Layer~1 consisting of the lines with $x_1$, $x_2$ and $x_3$, 
Layer~2 consisting of the lines with $x_4$ and $x_5$, and
Layer~3 consisting of the line with $x_6$.  %\qed

\medskip
From Part~1 of Observation~\ref{obs:ncf_transformations},
it follows that the lines within the same layer of an NCF
representation can be permuted without changing the function.
Part~2 of Observation~\ref{obs:ncf_transformations} points
out that for any NCF $f$ with $n \geq 2$ variables, there is a simplified 
representation in which lines $n-1$ and $n$ are in the same layer.
We can now define the notion of a \textbf{default-normalized} 
representation for NCFs.

\begin{definition}
\label{def:normalized}
Let $f$ be an NCF with $n \geq 2$ variables specified using
the simplified representation with 
the variable ordering $\langle x_1, x_2, \ldots, x_n\rangle$.
We say that the representation is {\bf default-normalized} if
lines $n-1$ and $n$ have the same canalyzed value.
\end{definition}

\noindent
\textbf{Example 5:}~ 
Consider the NCF $f$ shown in Example~4. 
This representation is \emph{not} default-normalized since lines 5 and 6
have different canalyzed values.
If we change line 6 to ``$x_6$:~ 0 $\longrightarrow$ 1" and the
value on the ``Default" line to 0, then we obtain 
a default-normalized representation. 

\medskip
From Observation \ref{obs:ncf_transformations},
we note that in polynomial time, for any NCF representation with more than one line,
we can first modify the last line if necessary 
so that it has the same canalyzed value as the preceding line,
thereby obtaining an equivalent default-normalized NCF representation.
We state this formally below.

\begin{observation}\label{obs:normalization_poly}
Given an NCF representation for a Boolean function $f$, a default-normalized
representation for $f$ can be obtained in polynomial time. \QED
\end{observation}

\noindent
In view of Observation~\ref{obs:normalization_poly},
we assume henceforth that any NCF is specified in default-normalized form.

\subsection{Complexity of Approximating the Symmetry Level of a 
General Boolean Function}
\label{sse:symmetry_level_hardness}

Recall that the symmetry level of a Boolean function $f$ is the smallest
integer $r$ such that the following condition holds:
the  inputs to $f$ can be partitioned into $r$ subsets 
(symmetry groups) so that the value of $f$ depends only on 
how many of the inputs in each group have the value 1.
We now show that given a Boolean function $f$
in the form of a Boolean expression, it is \cnp-hard to 
approximate\footnote{An algorithm approximates the symmetry
level of a Boolean function $f$ within the factor $\rho$ if it finds a
partition of the inputs to $f$ into at most $\rho \times r^*$ symmetry
groups, where $r^*$ is the symmetry level of $f$.}
the symmetry level of $f$ to within any factor $\rho \geq 1$.
This result holds even when the function is given as an expression
in conjunctive normal form (CNF), that is, in the form of a conjunction
of clauses where each clause is a disjunction of literals. 

\newcommand{\cala}{\mbox{$\mathcal{A}$}}

\begin{theorem}\label{thm:approx_sym_level_hard}
Unless \textbf{P} = \cnp,
for any $\rho \geq 1$, there is no polynomial time 
algorithm for approximating the 
symmetry level of a Boolean function $f$ 
specified as a Boolean expression to within the factor $\rho$.
This result holds even when $f$ is a CNF expression.
\end{theorem}

\noindent
\textbf{Proof:}~ For some $\rho \geq 1$, suppose there is an efficient algorithm \cala{} 
that approximates the symmetry level of a Boolean function
specified as a Boolean expression to within the factor $\rho$.
Without loss of generality, we can assume that $\rho$ is an integer.
We will show that \cala{} can be used to efficiently solve
the CNF Satisfiability problem (SAT) which is known 
to be \cnp-hard \citep{GJ-1979}.

Let $g$ be a CNF formula representing an instance of SAT.
Let $X$ = $\{x_1, x_2, \ldots, x_n\}$ denote the set of
Boolean variables used in $g$.
We create another CNF formula $f$ as follows.
Let $Y = \{y_1, y_2, \ldots, y_{\rho+1}\}$ and
$Z = \{z_1, z_2, \ldots, z_{\rho+1}\}$ be two 
new sets of variables, with each set containing $\rho+1$ variables.
The expression for $f$, which is a function of $n+2\rho+2$
variables, namely $x_1, \ldots, x_n$, $y_1, \ldots, y_{\rho+1}$,
$z_1, \ldots, z_{\rho+1}$, is the following:
\[
    g(x_1, \ldots x_n) \wedge (y_1 \vee \overline{z_1}) 
                       \wedge (y_2 \vee \overline{z_2}) 
                       \wedge \ldots \wedge
                              (y_{\rho+1} \vee \overline{z_{\rho+1}}).
\]
Since $g$ is a CNF formula, so is $f$.
We have the following claims.

\medskip

\noindent
\textbf{Claim 1:}~ If $g$ is \emph{not} satisfiable, 
the symmetry level of $f$ is 1.

\smallskip

\noindent
\textbf{Proof of Claim 1:}~ If $g$ is not satisfiable, $f$ is also
not satisfiable; that is, for all inputs, the value of $f$ is 0.
Thus, all the variables in $X \cup Y \cup Z$ can be included in one
symmetry group.
The value of $f$ is 0 regardless of how many variables in the
group have the value 1.  \hfill$\Box$

\medskip

\noindent
\textbf{Claim 2:}~ If $g$ \emph{is} satisfiable, 
the symmetry level of $f$ is at least $\rho+1$.

\smallskip

\noindent
\textbf{Proof of Claim 2:}~ Suppose $g$ is satisfiable.
We argue that if $i \neq j$, variables $y_i$ and $y_j$
cannot be in the same symmetry group for the function $f$.
To see this, consider any two variables $y_i$ and $y_j$ with $i \neq j$,
and construct an assignment $\alpha$ to
the variables of $f$ in the following manner.
\begin{enumerate}
\item For the variables in $X$, choose any assignment that
satisfies $g$.
\item Let $y_i = 0$, $z_i = 1$, $y_j = 1$ and $z_j = 0$.
\item For $1 \leq p \leq \rho+1$, $p \neq i$ and $p \neq j$,
let $y_p = 1$ and $z_p$ = 0.
\end{enumerate}
Since this assignment $\alpha$ sets the clause $(y_i \vee \overline{z_i})$ to 0,
the value of $f$ under the assignment $\alpha$ is 0.
Now, consider the assignment $\alpha'$ that is obtained by interchanging
the values of $y_i$ and $y_j$ in $\alpha$ without changing the
values assigned to the other variables. 
It can be seen that under assignment $\alpha'$, the value of $f$ is 1.
In both $\alpha$ and $\alpha'$, the number of 1-valued variables in 
the set $\{y_i, y_j\}$ is 1; however, the two assignments lead to 
different values for $f$.
Therefore, if $g$ is satisfiable, for $i \neq j$, $y_i$ and $y_j$ cannot
be in the same symmetry group for $f$.
Since there are $\rho+1$ variables in $Y$, the number of symmetry groups 
for $f$ must be at least $\rho+1$, and this 
completes our proof of Claim~2. \hfill$\Box$

\medskip
 
We now continue with the proof of Theorem~\ref{thm:approx_sym_level_hard}.
Suppose we execute the approximation algorithm \cala{} 
on the function $f$ defined above.
If $g$ is \emph{not} satisfiable, then from Claim~1, 
the symmetry level of $f$ is 1. 
Since \cala{} is a $\rho$-approximation algorithm,
it should produce at most $\rho$ symmetry groups.
On the other hand, if $g$ is satisfiable, by Claim~2, the symmetry
level of $f$ is at least $\rho+1$; so, \cala{} will produce at least
$\rho+1$ groups.
In other words, $g$ is not satisfiable iff the number
of symmetry groups produced by \cala{} is at most $\rho$.
Since \cala{} runs in polynomial time, we have an efficient 
algorithm for SAT, contradicting the assumption that 
\textbf{P} $\neq$ \cnp.  \QED

\medskip

In contrast to the above result,
we will show in the next section that the symmetry level 
of an NCF can be computed efficiently.
