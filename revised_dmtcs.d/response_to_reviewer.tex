\documentclass[11pt]{article}

\setlength{\textheight}{9.0 in}
\setlength{\textwidth}{6.5 in}
\setlength{\oddsidemargin}{0 in}
\setlength{\evensidemargin}{0 in}
\setlength{\topmargin}{-0.5 in}

\setlength{\parskip}{4pt}

%% \usepackage[nofillcomment,ruled,linesnumbered]{algorithm2e}
\usepackage{amsmath}
\usepackage{amssymb}
\usepackage{color}

%% \pagestyle{empty}
\begin{document}

\newtheorem{theorem}{Theorem}[section]
\newtheorem{lemma}{Lemma}[section]
\newtheorem{corollary}{Corollary}[section]
\newtheorem{fact}{Fact}[section]
\newtheorem{definition}{Definition}[section]
\newtheorem{proposition}{Proposition}[section]
\newtheorem{observation}{Observation}[section]
\newtheorem{claim}{Claim}[section]

\newcommand{\cnp}{\textbf{NP}}
\newcommand{\true}{\texttt{True}}
\newcommand{\false}{\texttt{False}}

\newcommand{\QED}{\hfill\rule{2mm}{2mm}}

\newcommand{\irange}{\mbox{$1 \leq i \leq n$}}
\newcommand{\jrange}{\mbox{$1 \leq j \leq m$}}

\newcommand{\dunder}[1]{\underline{\underline{#1}}}

\setlength{\parskip}{3pt}

\baselineskip=1.2\normalbaselineskip

\begin{center}
\dunder{\Large{\textbf{Response to the Reviewer}}}
\end{center}

\bigskip

\hspace*{0.25in}
\mbox{
\begin{tabular}{ll}
\textsf{Title:}   & Symmetry Properties of Nested Canalyzing Functions \\ [2ex]
\textsf{Authors:} & D. J. Rosenkrantz, M. V. Marathe, S. S. Ravi 
                    and R. E. Stearns \\ [2ex]
\textsf{Journal:} & Discrete Mathematics and Theoretical Computer Science \\
\end{tabular}
}

\bigskip\bigskip

We thank the reviewer for reading the manuscript carefully and
providing very helpful suggestions. We have incorporated all the
corrections indicated in the section entitled ``Minor" in the review. 
Below, we mention how the two items mentioned in the 
``Major" section of the review have been 
addressed in the revised version.

\medskip

\begin{enumerate}
\item
The statement of Part (ii) of Corollary 3.4 has been corrected. It
now reads as follows: ``An $r$-symmetric NCF has a default-normalized
nested canalyzing representation with at most $r$ layers."

\item We have changed the statement of Theorem 3.6 to point out
that the algorithm runs in time that is linear in the \emph{size of the
input.} 
After the statement of the theorem, we have indicated that
when $r$ is fixed, the size of the input and the running time of the
algorithm are both polynomial functions of $n$, the number of variables.
If $r$ is not fixed, then the input size and the running time of the
algorithm are not necessarily polynomial functions of $n$; however,
the running time of the algorithm remains linear in the size of the input.
\end{enumerate}

\end{document}
