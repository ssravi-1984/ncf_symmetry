\section{Symmetry of Nested Canalyzing Functions}
\label{sse:ncf_and_symmetry}

\subsection{Overview}
\label{sse:res_overview}


\subsection{Symmetric Canalyzing and Nested Canalyzing Functions}
\label{sse:sym_and_cf_ncf}

\begin{theorem}\label{thm:ncf_symmetric}
The only symmetric canalyzing functions are OR, AND, NOR, NAND, 
the constant function 0, and the constant function 1.

The only symmetric NCFs are OR, AND, NOR, and NAND.
\end{theorem}
\noindent
\textbf{Proof:}~
Suppose symmetric function $f$ is a canalyzing function. Then there
is a variable $x$, and values $a$ and $b$ such that whenever $x =
a$, function $f$ has value $b$.  Since $f$ is symmetric, $f$  has
value $b$ whenever any of its variables have value $a$.  Thus, if
at least one of its variables has value $a$, then $f$ has value
$b$.  If $f$ has value $\bar{b}$ when none of its variables has
value $a$, the four possible combinations of values for $a$ and $b$
correspond to the four functions OR, AND, NOR, and NAND.  If $f$
has value $b$ when none of its variables has value $a$, then $f$
is the constant function $b$.

Suppose that $f$ is a NCF.  Since constant functions are not NCFs,
$f$ has value $\bar{b}$ when none of its variables has value $a$.
The four possible combinations of values for $a$ and $b$ correspond
to the four functions OR, AND, NOR, and NAND, each of which is a
NCF.  \QED

\subsection{Symmetric Pairs of Variables in an NCF: A Characterization
and Its Applictions}
\label{sse:ncf_strong_sym}

\begin{theorem}\label{thm:ncf_symmetric_variables}
A pair of variables of a NCF are symmetric iff
in the normalized representation of the NCF
they occur in the same layer and have the same canalyzing value.
\end{theorem}
\noindent
\textbf{Proof:}~
Suppose that a pair of variables occur in the same layer of a
canalyzing representation, and the lines containing these variables
have the same canalyzing value.  Then in any input assignment, the
values of these two variables can be interchanged, and the evaluation
of the lines in that layer will have the same effect.  Thus, the
variables are symmetric.

Now, let $f$ be the normalized NCF representation for the function.
For purposes of notational simplicity, wlog we assume that the
canalyzing variable in line $i$ of $f$ is $x_i$, $1 \leq i \leq n$.
Let  $x_j$ and $x_k$ be a pair of variables that occur in different
layers of $f$, or occur in the same layer, but with different
canalyzing values.  Wlog, assume that $j < k$, so that the line for
$x_j$ occurs above the line for $x_k$.

Suppose that line $i$ of $f$, $1 \leq i \leq n$, is 

\noindent
$(x_i : a_i \rightarrow b_i)$. \\
Consider the following input assignment $\alpha = (c_1, c_2, \ldots,  c_n)$ 
to the variables of $f$.

\noindent
For $1 \leq i < j$, set $c_i = \overline{a_i}$. \\
For $i = j$, set $c_j =a_j$. \\
For $i = k$, set $c_k =\overline{a_j}$. \\
For $j < i < k$, and for  $k < i \leq n$, 
if $b_i = b_j$, then set $c_i =  \overline{a_i}$,
else set $c_i =  a_i$.

Note that $f(\alpha) = b_j$, since the variables in all lines above
line $j$ have the complement of their canalyzing value, and variable
$x_j$ has its canalyzing value.

Now let $\alpha'$ be the assignment that is obtained from $\alpha$
by interchanging the values of variables $x_j$ and $x_k$, so that
after the interchange variable $x_j$ has value $\overline{a_j}$ and
variable $x_k$ has value $a_j$.  We will show that $f(\alpha') =
\overline{b_j}$, so variables $x_j$ and $x_k$ are not symmetric.

In line $k$ of $f$, canalyzing value $a_k$ is either the same or
the complement of canalyzing value $a_j$, and canalyzed output $b_k$
is either the same or the complement of canalyzed output $b_j$,
Thus, there are four possible cases for the form of line $k$.  We
now consider each of the four cases.

{\bf Case 1:} $(x_k : a_j \rightarrow b_j)$. \\ Since line $k$ has
the same canalyzing value and canalyzed output as line $j$, line
$k$ must occur in a lower layer than that of line $j$.  Thus, there
is at least one line between line $j$ and line $k$ for which the
canalyzed output is $\overline{b_j}$.  Let line $q$ be the first
such line.  Note that for all $i$ such that $1 \leq i < q$, the
value of variable $x_i$ in assignment $\alpha'$ does not match the
canalyzing value of line $i$, but the value of variable $x_q$ does
match the canalyzing value of line $q$.  Since canalyzed output
$b_q$ equals $\overline{b_j}$, we have that $f(\alpha') =
\overline{b_j}$.

{\bf Case 2:} $(x_k : a_j \rightarrow \overline{b_j})$. \\ Let line
$q$ be the first line such that $j < q \leq k$ and the canalyzed
output of line $q$ is $\overline{b_j}$.  Note that $q$ might possibly
equal $k$.  For all $i$ such that $1 \leq i < q$, the value of
variable $x_i$ in assignment $\alpha'$ does not match the canalyzing
value of line $i$, but the value of variable $x_q$ does match the
canalyzing value of line $q$.  Since canalyzed output $b_q$ equals
$\overline{b_j}$, we have that $f(\alpha') = \overline{b_j}$.

{\bf Case 3:} $(x_k : \overline{a_j} \rightarrow b_j)$. \\ Suppose
there is a line below line $j$ for which the canalyzed output is
$\overline{b_j}$.  Let $q$ be the first such line.  Note that for
all $i$ such that $1 \leq i < q$, the value of variable $x_i$ in
assignment $\alpha'$ does not match the canalyzing value of line
$i$, but the value of variable $x_q$ does match the canalyzing value
of line $q$.  Since canalyzed output $b_q$ equals $\overline{b_j}$,
we have that $f(\alpha') = \overline{b_j}$.

Now suppose there is no line below line $j$ for which the canalyzed
output is $\overline{b_j}$.  Then lines $j$ and $k$ both occur in
the last layer of $f$.  Thus, for every line $i$, $1 \leq i \leq
n$, the value of variable $x_i$ in assignment $\alpha'$ does not
match the canalyzing value of line $i$.  Consequently, $f(\alpha')
= \overline{b_j}$, the complement of the canalyzed output in the
last layer of $f$.

{\bf Case 4:} $(x_k : \overline{a_j} \rightarrow \overline{b_j})$.
\\ Suppose there is a line, other than line $k$, below line $j$ for
which the canalyzed output is $\overline{b_j}$.  Let $q$ be the
first such line.  Note that for all $i$ such that $1 \leq i < q$,
the value of variable $x_i$ in assignment $\alpha'$ does not match
the canalyzing value of line $i$, but the value of variable $x_q$
does match the canalyzing value of line $q$.  Since canalyzed output
$b_q$ equals $\overline{b_j}$, we have that $f(\alpha') =
\overline{b_j}$.

Now suppose that line $k$ is the only line below line $j$ for which
the canalyzed output is $\overline{b_j}$.  Since line $j$ has
canalyzed output $b_j$, line $k$ is the only line in its layer.
Since $f$ is normalized, the last layer of $f$ contains at least
two lines.  Thus, there is a last layer following the layer containing
line $k$.  The canalyzed output of all the lines in this last layer
is $b_j$.  Thus, for every line $i$, $1 \leq i \leq n$, the value
of variable $x_i$ in assignment $\alpha'$ does not match the
canalyzing value of line $i$.  Consequently, $f(\alpha') =
\overline{b_j}$, the complement of the canalyzed output in the last
layer of $f$.  \QED

\begin{theorem}\label{thm:ncf_r_symmetric}
Suppose the normalized representation of a given NCF contains 
$r_1$ layers with only one distinct canalyzing value,
and $r_2$ layers with two distinct canalyzing values.
Then the function is properly $(r_1 + 2 r_2)$-symmetric.
\end{theorem}
\noindent
\textbf{Proof:}~
From Theorem~\ref{thm:ncf_symmetric_variables}, 
any pair of variables occurring in the same layer, with the same canalyzing value,
are symmetric.
Thus the function is at most $(r_1 + 2 r_2)$-symmetric.
Moreover, any pair of variables from different layers, or with different canalyzing values
are not symmetric, so there are at least $(r_1 + 2 r_2)$ symmetry groups.
\QED

\begin{corollary}\label{cor:ncf_not_rsymm}
For every $n \geq 2$, there is an $n$-variable NCF that is not $n-1$ symmetric.
\end{corollary}

\begin{corollary}\label{cor:ncf_r_symmetric_layers}
A NCF with a normalized nested canalyzing representation consisting
of $q$ layers is $2q$-symmetric, and is not $(q-1)$-symmetric.

An $r$-symmetric NCF has a normalized nested canalyzing representation
with at most $r-1$ layers, and has an unnormalized nested canalyzing
representation with at most $r$ layers.  \end{corollary}

\subsection{Strong Asymmetry and NCFs}
\label{sse:strong_asym_ncf}

Generalized forms of symmetry have been considered \cite{KS-2000}
in which a Boolean function is symmetric with respect to some permutation,
which can be more general than the permutations corresponding to symmetry groups.
We say that a Boolean function $f(x_1, x_2, \ldots, x_n)$ 
is \textbf{strongly asymmetric}
if for any permutation $\pi$ of $\{1, 2, \ldots, n\}$
\emph{except} the identity permutation,
there exists an input $(a_1, a_2, \ldots,  a_n)$ 
to $f$ such that $f(a_1, a_2, \ldots, a_n)$ $\neq$  
$f(a_{\pi(1)}, a_{\pi(2)}, \ldots, a_{\pi(n)})$.
In general, there are functions with $n$ variables that are properly
$n$-symmetric, but not strongly asymmetric \cite{KS-2000}.

We now show that any $n$ variable NCF that is properly $n$-symmetric
is also strongly asymmetric.
As a consequence, there are NCFs with $n$ variables
which are properly $n$-symmetric.

\begin{theorem}\label{thm:ncf_strong_asymmetry}
A NCF with $n$ variables is strongly asymmetric iff
it is properly $n$-symmetric.
\end{theorem}

\noindent
\textbf{Proof:}~ 
If a NCF is $(n-1)$-symmetric, then consider a permutation than
interchanges two variables from a symmetry group with at least two
members.  The value of the function is invariant under any such
permutation, so the function is not strongly asymmetric.

Now consider a $n$ variable NCF that is properly $n$-symmetric.
Let $f$ be the normalized NCF representation for the function.  For
purposes of notational simplicity, wlog we assume that the canalyzing
variable in line $i$ of $f$ is $x_i$, $1 \leq i \leq n$.  Let $\pi$
be any permutation of $\{1, 2, \ldots, n\}$ \emph{except} the
identity permutation.  We will construct an assignment $(c_1, c_2,
\ldots,  c_n)$ to the variables of $f$ such that $f(c_1, c_2, \ldots,
c_n)$ $\neq$ $f(c_{\pi(1)}, c_{\pi(2)}, \ldots, c_{\pi(n)})$.

Let $k$ be the smallest index such that $k \neq \pi(k)$.  Since
$\pi$ is a permutation, $k < n$.  Overall, for $1 \leq i < k$,
$\pi(i) = i$;
 $\pi(k) > k$; and for $k < i \leq n$, $\pi(i) \geq k$.

Suppose that line $i$ of $f$, $1 \leq i \leq n$, is \noindent $(x_i
: a_i \rightarrow b_i)$. \\ Assignment $(c_1, c_2, \ldots,  c_n)$
is constructed as follows.

\noindent
For $1 \leq i < k$, set $c_i = \overline{a_i}$. \\ For $i = k$, set
$c_k =a_k$. \\ For $i > k$, let $i' = \pi^{-1}(i)$, so that $\pi(i')
= i$.  Thus, after the permutation of values, variable $x_{i'}$
will have value $c_i$.  If $b_{i'} = b_k$, then set $c_i =
\overline{a_{i'}}$, else set $c_i =  a_{i'}$.

Since $c_k$ matches the canalyzing value of line $k$ of $f$, and
$c_i$ does not match the canalyzing value of any other earlier line
$i$, $1 \leq i < k$, we have that $f(c_1, c_2, \ldots,  c_n) = b_k$.

Now consider $f(c_{\pi(1)}, c_{\pi(2)}, \ldots, c_{\pi(n)})$.  We
first note that for $1 \leq i < k$, $c_{\pi(i)}$ does not match the
canalyzing value of line $i$.

Let $k' = \pi^{-1}(k)$, so that $\pi(k') = k$.  Canalyzing value
$a_{k'}$ is either the same or the complement of $a_k$, and canalyzed
output $b_{k'}$ is either the same or the complement of $b_k$, Thus,
there are four possible cases for the form of line $k'$ of $f$.  We
will show that in each of the four cases, $f(c_{\pi(1)}, c_{\pi(2)},
\ldots, c_{\pi(n)}) = \overline{b_k}$.

\noindent
{\bf Case 1:} $(x_{k'} : a_k \rightarrow b_k)$. \\ 

Since line $k'$
has the same canalyzing value and canalyzed output as line $k$, and
$f$ is properly $n$-symmetric, line $k'$ must occur in a lower layer
than that of line $k$.  Thus, there is at least one line between
line $k$ and line $k'$ for which the canalyzed output is $\overline{b_k}$.
Let line $q$ be the first such line.  Note that for all $i$ such
that $1 \leq i < q$, $c_{\pi(i)}$ does not match the canalyzing
value of line $i$, but $c_{\pi(q)}$ does match the canalyzing value
of line $q$.  Since canalyzed output $b_q$ equals $\overline{b_k}$,
we have that $f(c_{\pi(1)}, c_{\pi(2)}, \ldots, c_{\pi(n)}) =
\overline{b_k}$.

\noindent
{\bf Case 2:} $(x_{k'} : a_k \rightarrow \overline{b_k})$. \\

Let line $q$ be the first line such that $k < q \leq k'$ and
the canalyzed output of line $q$ is $\overline{b_k}$.
Note that for all $i$ such that $1 \leq i < q$, 
$c_{\pi(i)}$ does not match the canalyzing value of line $i$,
so $f(c_{\pi(1)}, c_{\pi(2)}, \ldots, c_{\pi(n)}) = \overline{b_k}$.

\noindent
{\bf Case 3:} $(x_{k'} : \overline{a_k} \rightarrow b_k)$. \\

Suppose there is a line below line $k$ for which the canalyzed
output is $\overline{b_k}$.  Let $q$ be the first such line.  Note
that for all $i$ such that $1 \leq i < q$, $c_{\pi(i)}$ does not
match the canalyzing value of line $i$, but $c_{\pi(q)}$ does match
the canalyzing value of line $q$.  Thus, $f(c_{\pi(1)}, c_{\pi(2)},
\ldots, c_{\pi(n)}) = \overline{b_k}$.

Now suppose there is no line below line $k$ for which the canalyzed
output is $\overline{b_k}$.  Then line $k$ occurs in the last layer
of $f$.  Since $f$ is both normalized and properly $n$-symmetric,
this last layer contains exactly two lines.  Thus $k = n-1$ and $k'
= n$.  Note that for all $i$, $c_{\pi(i)}$ does not match the
canalyzing value of line $i$.  Thus, $f(c_{\pi(1)}, c_{\pi(2)},
\ldots, c_{\pi(n)}) = \overline{b_k}$.

\noindent
{\bf Case 4:} $(x_{k'} : \overline{a_k} \rightarrow \overline{b_k})$. \\

Suppose there is a line, other than line $k'$, below line $k$ for
which the canalyzed output is $\overline{b_k}$.  Let $q$ be the
first such line.  Note that for all $i$ such that $1 \leq i < q$,
$c_{\pi(i)}$ does not match the canalyzing value of line $i$, but
$c_{\pi(q)}$ does match the canalyzing value of line $q$.  Thus,
$f(c_{\pi(1)}, c_{\pi(2)}, \ldots, c_{\pi(n)}) = \overline{b_k}$.

Now suppose that line $k'$ is the only line below line $k$ for which
the canalyzed output is $\overline{b_k}$.  Since line $k$ has
canalyzed output $b_k$, line $k'$ is the only line in its layer.
Since $f$ is normalized, there is a last layer following the layer
containing line $k'$.  Thus, for all $i$, $c_{\pi(i)}$ does not
match the canalyzing value of line $i$, so $f(c_{\pi(1)}, c_{\pi(2)},
\ldots, c_{\pi(n)}) = \overline{b_k}$.  \QED


Examples of properly $n$-symmetric $n$-variable NCFs satisfying the
conditions of Theorem \ref{thm:ncf_r_symmetric} are the following.
For each $n > 1$, let $f_n$ be the function defined by the following
formula.

$$x_1 \vee \bar{x}_2 ( x_3 \vee \bar{x}_4 (\cdots  ) )$$

For instance,
$$f_6 = x_1 \vee \bar{x}_2 ( x_3 \vee \bar{x}_4 ( x_5 \vee \bar{x}_6  ) ),$$
and
$$f_7 = x_1 \vee \bar{x}_2 ( x_3 \vee \bar{x}_4 ( x_5 \vee \bar{x}_6   x_7) ).$$

Function $f_n(x_1, x_2, \ldots, x_n)$ is the NCF corresponding to
the following NCF representation:

\medskip
\noindent
\hspace*{0.5in}
$x_1:~ (1 \rightarrow 1)$ \\
\hspace*{0.5in}
$x_2:~ (1 \rightarrow 0)$ \\
\hspace*{0.5in}
$x_3:~ (1 \rightarrow 1)$ \\
\hspace*{0.75in}
$\vdots$ 

\noindent
If $n$ is odd, the last line is 

\noindent
\hspace*{0.5in}
$x_n:~ (1 \rightarrow 1)$ 

\noindent
and if $n$ is even, the last line is 

\noindent
\hspace*{0.5in}
$x_n:~ (1 \rightarrow 0)$. \\

Note that if $x_i = 0$ for all $i$, $1 \leq i \leq n$, then the
value of $f_n$ is 0 if $n$ is odd and 1 if $n$ is even.  Also, note
that the above representation is unnormalized. To normalize the
representation, the last line would be changed to have canalyzing
value 0, and the same canalyzed output as the preceding line.






\iffalse
%%%%%%%%%%%%%%%%%%%%%%%%%%%%%%%%%%%%%%%%%%%%%\begin{theorem}\label{thm:ncf_asymmetry}
\begin{theorem}\label{thm:ncf_asymmetry}.
Consider the set of variables $x_i$ for $1 \leq i \leq n$ and 
let $f(x_1, x_2, \ldots, x_n)$ be the NCF defined by the following sequence
of lines:

\medskip
\noindent
\hspace*{0.5in}
$x_1:~ (1 \rightarrow 1)$ \\
\hspace*{0.5in}
$x_2:~ (1 \rightarrow 0)$ \\
\hspace*{0.5in}
$x_3:~ (1 \rightarrow 1)$ \\
\hspace*{0.75in}
$\vdots$ 

\noindent
If $n$ is odd, the last line is 

\noindent
\hspace*{0.5in}
$x_n:~ (1 \rightarrow 1)$ 

\noindent
and if $n$ is even, the last line is 

\noindent
\hspace*{0.5in}
$x_n:~ (1 \rightarrow 0)$. \\

\noindent
Let $\pi$ be any permutation of $\{1, 2, \ldots, n\}$
\emph{except} the identity permutation.
Then there exists an input $(a_1, a_2, \ldots,  a_n)$ 
to $f$ such that $f(a_1, a_2, \ldots, ,a_n)$ $\neq$  
$f(a_{\pi(1)}, a_{\pi(2)}, \ldots, a_{\pi(n)})$.
\end{theorem}

\noindent
\textbf{Proof:}~ 
Note that if $x_i = 0$ for all $i$, $1 \leq i \leq n$, then the
value of the function $f$ is 0 if $n$ is odd and 1 if $n$ is even.

First assume there is an odd $i$ such that $\pi(i)$ is even. 
Consider the input where $a_i$ is 1 and all others are 0. 
It can be verified that $f(a_1, a_2, \ldots, a_n)$ = 1 while
$f(a_{\pi(1)}, a_{\pi(2)}, \ldots, a_{\pi(n)})$ = 0.

\smallskip
Now assume the permutation $\pi$ is such that for each odd $i$, $\pi(i)$ is odd 
and for each even $i$, $\pi(i)$ is even.
Let $k$ be the smallest index such that $k \neq \pi(k)$.
Assume that $k$  is odd. 
Construct the input $(a_1, a_2, \ldots,  a_n)$ to $f$ as follows: 
\begin{enumerate}
\item Set $a_k$ to 1 and $a_i$ to 0 for all other odd $i$. 
\item For even $i$, set $a_i$ to 0 if $i < k$ and to 1 otherwise. 
\end{enumerate}
We have the following claim.

\smallskip
\noindent
\textbf{Claim:}~ For the input $(a_1, a_2, \ldots, a_n)$, 
$f(a_1, a_2, \ldots, a_n)$ = 1 and
$f(a_{\pi(1)}, a_{\pi(2)}, \ldots, a_{\pi(n)})$ = 0.

\smallskip
\noindent
\textbf{Proof of claim:}~ Note that the permutation
on the even indices maps 1's to 1's and leaves 0's in place.  
Before the permutation,  
$x_j = 0$ for $1 \leq j < k$ and $x_k = 1$,
so the canalyzing variable is $x_k$ and the function value is 1.
After the permutation, 
$x_j = a_j = 0$ for $1 \leq j < k$,
$x_k = 0$, and $x_{k+1} = 1$;
so the canalyzing variable is $x_{k+1}$ and the function value is 0.
This establishes the claim. 

A similar proof can be given when $k$ is even and this completes
the proof of Theorem~\ref{thm:ncf_asymmetry}.
\QED

\noindent
The following result is a consequence of the above theorem.

\begin{corollary}\label{cor:ncf_not_rsymm}
For every $n \geq 2$, there is an $n$-variable NCF that is not $n-1$ symmetric.
\end{corollary}

\noindent
\textbf{Proof:}~ 
Let $f$ be the Boolean function defined in Theorem~\ref{thm:ncf_asymmetry}.
We will show by contradiction that $f$ is not $r$-symmetric for any $r < n$.

Suppose $f$ is $r$-symmetric for some $r < n$.
Then, by definition, there is a partition of the set $X$ =
$\{x_1, x_2, \ldots, x_n\}$ of variables into at most $r$ subsets such that
the value of $f$ depends only on the number of variables with value 1
in each subset.
Since $r < n$, this partition has a subset, say $X_j$, 
such that $|X_j| \geq 2$.
Let $X_j = \{x_{j_1}, x_{j_2}, \ldots, x_{j_k}\}$, where $k \geq 2$.
Assume without loss of generality that $j_1 < j_2 < \ldots < j_k$.
Define a permutation $\pi$ of $\{1, 2, \ldots, n\}$ as follows. 
\begin{enumerate}
\item For all $i$ such that $x_i \not\in X_j$, $\pi(i) = i$.
\item $\pi(j_{\ell})$ = $j_{\ell+1}$ for $1 \leq \ell_t \leq k-1$
      and $\pi(j_{k})$ = $j_1$. 
\end{enumerate}
Note that $\pi$ is not the identity permutation.
Thus, by Theorem~\ref{thm:ncf_asymmetry},
there is an input $(a_1, a_2, \ldots, a_n)$ to $f$ such that
such that
\begin{equation}\label{eqn:f_asymmetry}
f(a_1, a_2, \ldots, a_n) ~\neq~  f(a_{\pi(1)}, a_{\pi(2)}, \ldots, a_{\pi(n)})
\end{equation}
However, from the definition of permutation $\pi$, it follows
that the number of 1's in any subset of the partition 
is the same in both the inputs
$(a_1, a_2, \ldots, a_n)$ and
$(a_{\pi(1)}, a_{\pi(2)}, \ldots, a_{\pi(n)})$.
Therefore, since $f$ is $r$-symmetric, we must have
\begin{equation}\label{eqn:f_contra}
f(a_1, a_2, \ldots, a_n) ~=~ f(a_{\pi(1)}, a_{\pi(2)}, \ldots, a_{\pi(n)}).
\end{equation}
Equation~(\ref{eqn:f_contra}) contradicts 
Equation~(\ref{eqn:f_asymmetry}), and the result follows. \QED
%%%%%%%%%%%%%%%%%%%%%%%%%%%%%%%%%%%%%%%%%%%%%
\fi


\iffalse
%%%%%%%%%%%%%%%%%%%%%%%%%%%%%%%%%%%%%%%%%%%%%
$$x_1 \vee \bar{x}_2 ( x_3 \vee \bar{x}_4 (\cdots  ) )$$

For instance,
$$f_6 = x_1 \vee \bar{x}_2 ( x_3 \vee \bar{x}_4 ( x_5 \vee \bar{x}_6  ) ),$$
and
$$f_7 = x_1 \vee \bar{x}_2 ( x_3 \vee \bar{x}_4 ( x_5 \vee \bar{x}_6   x_7) ).$$

Function $f_n$ is a NCF, 
and can be representing as a sequence where the variables appear in lexicographic order,
each $a_i$ is 1, and the $b_i$'s alternate between 1 and 0.
For instance, $f_6$ can be represented as follows:

\noindent
$(x_1: 1 \rightarrow 1 )$ \\
$(x_2: 1 \rightarrow 0 )$ \\
$(x_3: 1 \rightarrow 1 )$ \\
$(x_4: 1 \rightarrow 0 )$ \\
$(x_5: 1 \rightarrow 1 )$ \\
$(x_6: 1 \rightarrow 0 )$ \\

We now argue that $f_n$ is not $n-1$ symmetric.
Consider the classes of $f_n$, represented as an $r$-symmetric function for some $r$.
If we compare the assignment where a given odd-numbered variable $x_{2i+1}$ is the only variable equal to 1,
and the assignment where a given even-numbered variable $x_{2j}$ is the only variable equal to 1,
we can see that these assignments produce different values for $f_n$,
so variables $x_{2i+1}$ and $x_{2j}$ must belong to different classes.
If we compare the assignment where an odd-numbered variable $x_{2i+1}$  
and the even-numbered variable $x_{2i+2}$ 
are the only variables equal to 1,
and the assignment where an odd-numbered variable of the form $x_{2k+1}$, $k >i$,
and $x_{2i+2}$  are the only variables equal to 1,
we can see that these assignments produce different values for $f_n$,
so $x_{2i+1}$ and $x_{2k+1}$ must belong to different classes.
Thus, each odd-numbered variable is the only member of its class.
If we compare the assignment where an even-numbered variable $x_{2i}$  
and the odd-numbered variable $x_{2i+1}$ 
are the only variables equal to 1,
and the assignment where an even-numbered variable of the form $x_{2k}$, $k >i$,
and $x_{2i+1}$  are the only variables equal to 1,
we can see that these assignments produce different values for $f_n$,
so $x_{2i}$ and $x_{2k}$ must belong to different classes.
Thus, each even-numbered variable is the only member of its class.
\QED

%%The following result shows another form of asymmetry for the function
%%defined in the above theorem.  (This form of asymmetry may be stronger
%%than the one established in the above theorem.)
%%%%%%%%%%%%%%%%%%%%%%%%%%%%%%%%%%%%%%%%%%%%%
\fi



